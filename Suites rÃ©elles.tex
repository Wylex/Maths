\documentclass[fleqn]{article}
\usepackage{amssymb}
\usepackage{amsthm}
\usepackage{amsmath}
\usepackage{array}

\title{Suites r\'eeles}
\date{}

\theoremstyle{definition} \newtheorem*{defi}{D\'efinition}
\theoremstyle{definition} \newtheorem*{theo}{Th\'eor\`eme}
\theoremstyle{definition} \newtheorem*{prop}{Propri\'et\'e}
\theoremstyle{definition} \newtheorem*{coro}{Corollaire}
\theoremstyle{remark} \newtheorem*{rqs}{Remarque}

\begin{document}
\maketitle

\section{G\'en\'eralit\'es}
\begin{defi}
		Soit $u \in \mathbb{R}^\mathbb{N}$. Une suite $v \in \mathbb{R}^\mathbb{N}$ est une sous suite de $u$ s'il existe
		$\varphi: \mathbb{N} \rightarrow \mathbb{N}$ strictement croissante tel que $\forall n \in \mathbb{N}, v_n = u_{\varphi(n)}$
\end{defi}
\begin{itemize}
	\item [-] $u$ est major\'ee s'il existe $M \in \mathbb{R}$ tel que $\forall n \in \mathbb{N}, u_n \leq M$
	\item [-] $u$ est minor\'ee s'il existe $m \in \mathbb{R}$ tel que $\forall n \in \mathbb{N}, u_n \geq m$
	\item [-] $u$ est born\'ee lorsqu'elle est \`a la fois major\'ee et minor\'ee
\end{itemize}

\section{Convergences}
\subsection{Suites convergentes}
\begin{defi}
	Soit $u \in \mathbb{R}^\mathbb{N}, l \in \mathbb{R}$, on dira que $u$ converge vers $l$ si:
	\[\forall \epsilon > 0, \exists n_0 \in \mathbb{N}, \forall n \geq n_0,\ |u_n - l| \leq \epsilon\]
\end{defi}

\begin{prop} $ $
	\begin{enumerate}
		\item Unicit\'e: soit $u$ une suite convergente, alors il existe un unique r\'eel $l$ tel que $u$ converge vers $l$
		\item Sous-suites: soit $u \in \mathbb{R}^\mathbb{N}$ une suite convergente de limite $l \in \mathbb{R}$. Alors toute sous-suite
		de $u$ converge vers $l$
		\begin{rqs}
			S'il existe une sous-suite de $u$ qui diverge, alors $u$ diverge. S'il existent 2 sous-suites de $u$ qui convergent des
			r\'eels distincts, alors $u$ diverge
		\end{rqs}
		\item Soit $u \in \mathbb{R}^\mathbb{N}, l \in \mathbb{R}$, $v_n = u_{2n}$, $w_n = u_{2n+1}$ \\
			Si $v$ et $w$ convergent vers une m\^eme limite $l \in \mathbb{R}$, alors $u$ converge vers $l$.
			(De m\^eme, pour toutes sous-suites qui convergent vers un m\^eme r\'eel et qui d\'ecrivent $\mathbb{N}$)
		\item Toute suite convergente  est born\'ee
	\end{enumerate}
\end{prop}

\subsection{Suites tendant vers $\infty$}
\begin{defi}
	Soit $u \in \mathbb{R}$. On dit que $u$ tend vers $+\infty$ (resp. $-\infty$) si:
	\[\forall M \in \mathbb{R}, \exists n_0 \in \mathbb{N}, \forall n \geq n_0, u_n \geq M \text{ (resp.} \leq)\]
\end{defi}
\begin{prop} $ $
	\begin{enumerate}
		\item Soient $u, v \in \mathbb{R}^\mathbb{N}$. On suppose $u_n \leq v_n$ pour $n$ assez grand: \\
			si $u$ tend vers $+\infty$, $v$ aussi. \\
			si $v$ tend vers $-\infty$, $u$ aussi.
		\item On suppose $u_n \rightarrow +\infty$ et $v$ minor\'ee: \\
			Alors $u+v$ tend vers $+\infty$
	\end{enumerate}
\end{prop}

\section{Ordre et limites}
\begin{prop} $ $
	\begin{enumerate}
		\item Conservation des in\'egalit\'ees larges par passage \`a la limite: \\
			$u, v$ 2 suites r\'eelles convergentes, $u_n \leq v_n$. Alors $\lim u \leq \lim v$
		\item Th\'eor\`eme des gendarmes: \\
			$u, v, w \in \mathbb{R}^\mathbb{N}, u_n \leq v_n \leq w_n$. Si $u$ et $v$ sont convergentes de m\^eme limite $l \in \mathbb{R}$
			alors $v$ converge vers $l$
		\item Suites monotones:
			Soit $u$ une suite r\'eelle croissante (analogue avec $u$ d\'ecroissante):
			\begin{enumerate}
				\item si $u$ est major\'ee alors $u$ converge
				\item si $u$ n'est pas major\'ee alors $u \rightarrow +\infty$
			\end{enumerate}
			%Soit $u$ une suite r\'eelle d\'ecroissante:
			%\begin{enumerate}
				%\item si $u$ est minor\'ee alors $u$ converge
				%\item si $u$ n'est pas minor\'ee alors $u \rightarrow -\infty$
			%\end{enumerate}
		\item Suites adjacentes:
			\begin{itemize}
				\item [-] D\'efinition: deux suites r\'eelles sont adjacentes si elles sont de monotonie contraire et si $u-v$ converge vers 0.
				\item [-] Deux suites adjacentes convergent vers un m\^eme r\'eel
			\end{itemize}
	\end{enumerate}
\end{prop}
\begin{theo} Segments embo\^it\'es \\
	Soit $(I_n)_{n \in \mathbb{N}}$ une suite de segments (il existe, pour
	$n \in \mathbb{N}, a_n, b_n \in \mathbb{R}$ tel que $I_n = [a_n, b_n]$ et $a_n \leq b_n$). On suppose $I_{n+1} \subset I_n$. Alors,
	\[\bigcup_{n \in \mathbb{N}} I_n \text{ est un segment non vide.}\]
\end{theo}
\begin{coro}
	$\mathbb{R}$ n'est pas d\'enombrable
\end{coro}

\section{Bolzano-Weierstrass (et adh\'erence)}
\begin{defi} Adh\'erence \\
	$u \in \mathbb{R}^\mathbb{N}, x \in \mathbb{R}$.  On dit que $x$ est valeur d'adh\'erence de $u$ s'il existe une sous-suite de $u$
	qui converge vers $x$
\end{defi}
\begin{prop} $ $
	\begin{itemize}
		\item [-] LASSE: soit $u \in \mathbb{R}^\mathbb{N}, x \in \mathbb{R}$
		\begin{enumerate}
			\item $x$ est valeur d'adh\'erence de $u$
			\item $\forall \epsilon > 0, \forall N \in \mathbb{N}, \exists n \geq N, |u_n - x| \leq \epsilon$
		\end{enumerate}
		\item [-] Soit $u$ une suite convergente de limite $l \in \mathbb{R}$. Alors $l$ est l'unique valeur d'adh\'erence de $u$
	\end{itemize}
\end{prop}

\begin{theo}
	Toute suite r\'elle born\'ee admet au moins une valeur d'adh\'erence
\end{theo}

\section{Suites particuli\`eres}
\begin{enumerate}
	\item Suites arithm\'etiques
	\item Suites g\'eom\'etriques
	\item Suites arithm\'etico-g\'eom\'etriques
	\item Suites v\'erifiant une relation de r\'ecurrence lin\'eaire d'ordre 2: \\
		$a \in \mathbb{R}^*, (b,c) \in \mathbb{R}^2\backslash\{0,0\}$. Soit $E$ l'ensemble des suites r\'eelles $u$ telles que:
		\[\forall n \in \mathbb{N}, au_{n+2} + bu_{n+1} + cu_n = 0\]
		Soit $P = aX^2 + bX + X$:
		\begin{enumerate}
			\item $P$ admet 2 racines r\'eelles distinctes $q_1, q_2$ alors:
				\[E = \{n \in \mathbb{N} \mapsto \alpha q_1^n + \beta q_2^n / \alpha, \beta \in \mathbb{R}\}\]
			\item $P$ a une racine double $q_0$ alors:
				\[E = \{n \in \mathbb{N} \mapsto (\alpha n + \beta) q_0^n / \alpha, \beta \in \mathbb{R}\}\]
			\item $P$ admet 2 racines complexes conjugu\'ees: $z_1 = re^{i\theta}, z_2 = re^{-i\theta}$, alors:
				\[E = \{n \in \mathbb{N} \mapsto r^n(\alpha \cos(n\theta) + \beta \sin(n\theta) / \alpha, \beta \in \mathbb{R}\}\]
		\end{enumerate}
\end{enumerate}

\end{document}
