\documentclass[fleqn]{article}
\usepackage{amssymb}
\usepackage{amsmath}
\usepackage{amsthm}

\title{Coefficients binomiaux}
\date{}

\theoremstyle{definition} \newtheorem*{defi}{D\'efinition}
\theoremstyle{definition} \newtheorem*{theo}{Th\'eor\`eme}
\theoremstyle{definition} \newtheorem*{prop}{Propri\'et\'e}

\begin{document}
\maketitle

\section{Coefficient binomial}
\begin{defi}
	Pour \(n \in \mathbb{N}\ et\ k \in [\![0,n]\!]\): \\
	\[ \binom{n}{k} = \frac{n!}{k!(n-k)!} \]
\end{defi}
\begin{itemize}
	\item Par convention, si $k>n, \binom{n}{k} = 0$
	\item Sym\'etrie: pour tout \(n \in \mathbb{N},\) \\
		$\binom{n}{0} = 1 = \binom{n}{n}$ \\
		$\binom{n}{1} = n = \binom{n}{n-1}$ \\
		$\binom{n}{k} = \binom{n}{n-k}$
\end{itemize}
\begin{prop} pour \(n \in \mathbb{N}\ et\ k \in [\![0,n]\!]\):
	\[\binom{n}{k+1} = \frac{n-k}{k+1} \times \binom{n}{k} \]
	\[\binom{n+1}{k+1} = \frac{n+1}{k+1} \times \binom{n}{k} \]
\end{prop}
\begin{theo} Pascal (\(n \in \mathbb{N}, k \in \mathbb{N}\)): \\
	\[\binom{n+1}{k+1} = \binom{n}{k} + \binom{n}{k+1}\]
\end{theo}
\begin{theo}
	Pour tout \(k,n \in \mathbb{N},\ \binom{n}{k} \in \mathbb{N}\)
\end{theo}
\begin{theo} Bin\^{o}me de Newton (\(a,b \in \mathbb{C}, n \in \mathbb{N}\)):
	\[(a+b)^n = \sum_{k=0}^{n} \binom{n}{k}a^kb^{n-k}\]
\end{theo}
\begin{prop} \(n \in \mathbb{N}^*, k \in [\![1,n]\!]\): \\
	\[k\binom{n}{k} = n\binom{n-1}{k-1}\] \\
\end{prop}

\subsection{Applications}
\begin{itemize}
	\item Soit \(p\) un nombre premier, \(k \in [\![1,p-1]\!]\) alors \(p\) divise \(\binom{p}{k}\)
	\item Propri\'{e}t\'{e}: \(a,b \in \mathbb{Z}\):
		\[(a+b)^p \equiv a^p + b^p \pmod{p}\]
		De fa\c{c}on plus g\'en\'erale, pour \(n \in \mathbb{N}^*, a_1, \hdots, a_n \in \mathbb{Z}\):
		\[(a_1+\hdots+a_n)^p \equiv a_1^p+\hdots+a_n^p \pmod{p}\]
	\item Petit th\'eor\`eme de Fermat (si \(p\) ne divise pas \(n\)):
		\[n^{p-1} \equiv 1\pmod{p} \]
\end{itemize}

\section{Sommes g\'eom\'etriques}
\begin{itemize}
	\item Principe de t\'elescopage (\(n \in \mathbb{N}\) et \(a_0, \hdots, a_{n+1} \in \mathbb{C}\)):
		\[\sum_{k=0}^{n} (a_{k+1} - a_k) = a_{n+1} - a_0\]
	\item Sommes g\'eom\'etriques (\(z \in \mathbb{C}, n \in \mathbb{N}\)):
		\[(z-1)\sum_{k=0}^{n}z^k = z^{n+1} -1\]
		\[\text{Si } z \neq 1, \sum_{k=0}^{n}z^k=\frac{z^{n+1}-1}{z-1}\]
	\item Propri\'{e}t\'{e} (\(a,b \in \mathbb{C}, n \in \mathbb{N}^*\)):
		\[a^n-b^n = (a-b)\sum_{k=0}^{n-1}a^kb^{n-k-1}\]
	\item Calcul de \(S_{n,p} = \sum_{k=1}^{n}k^p\)
		\begin{itemize}
			\item \(p = 0, S_{n,0} = \sum_{k=1}^{n}k^0 = \sum_{k=1}^{n}1 = n\)
			\item \(p = 1, S_{n,1} = \sum_{k=1}^{n}k^1 = \frac{n(n+1)}{2} \)
			\item \(p = 2, S_{n,2} = \sum_{k=1}^{n}k^2 = \frac{n(n+1)(2n+1)}{6} \)
		\end{itemize}
\end{itemize}

\section{Remarques sur les sommes}
\begin{prop} Changement d'indice\\
	Soient I et J deux ensembles finis d'indices et $\varphi$ une bijection de I sur J. Alors
	\[\underset{i \in I}{\sum} u_{\varphi(i)}\ = \underset{j \in J}{\sum} u_j\]
\end{prop}

\begin{prop} Sommes doubles
	\[\underset{i,j\in \Omega}{\sum} a_{ij} = \underset{i \in E}{\sum}\left(\underset{j \in F}{\sum}a_{ij}\right)\]
\end{prop}

\begin{prop} Sommes triangulaires
	\begin{align*}
	\underset{1 \leq i \leq j \leq n}{\sum} a_{ij} &= \sum_{i=1}^{n}\left(\sum_{j = i}^n a_{ij}\right) \\
	&= \sum_{j=1}^{n}\left(\sum_{i = 1}^j a_{ij}\right)
	\end{align*}
\end{prop}

\end{document}
