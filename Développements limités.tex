%Préamble {{{1
\documentclass[fleqn]{article}

\usepackage{amssymb}
\usepackage{amsmath}
\usepackage{amsthm}
\usepackage{verbatim}

\theoremstyle{definition} \newtheorem*{defi}{D\'efinition}
\theoremstyle{definition} \newtheorem*{theo}{Th\'eor\`eme}
\theoremstyle{definition} \newtheorem*{prop}{Propri\'et\'e}
\theoremstyle{definition} \newtheorem*{coro}{Corollaire}
\theoremstyle{remark} \newtheorem*{rqs}{Remarques}

\title{D\'eveloppements limit\'es}
\date{}

\begin{document}
\maketitle

%Définitions {{{1
\section{D\'efinitions}
\begin{defi}
	Soit $I$ un intervalle de $\mathbb{R}$, $f:I \rightarrow \mathbb{R},\ x_0 \in I,\ n \in \mathbb{N}$ \\
	On dira que f admet un DL \`a l'ordre n au voisinage de $x_0$ s'il existe \mbox{$a_0, \hdots, a_n \in \mathbb{R}$} tel que: \\
	\[f(x) - (a_0 + a_1(x-x_0) + \hdots + a_n(x-x_0)^n) \underset{x \rightarrow x_0}{=} o((x-x_0)^n)\]
\end{defi}

\begin{prop} unicit\'e des coefficients \\
	Si $f$ admet un $DL_n(x_0)$, il existe une unique liste $(a_0, \hdots, a_n) \in \mathbb{R}^{n+1}$ tel qu'au voisinage de $x_0$,
	\[f(x) = \sum_{i=0}^{n} a_i(x-x_0)^i  + o((x-x_0)^n)\]
\end{prop}

\begin{rqs}
	$\ $
	\begin{enumerate}
		\item Multiplication \'el\'ementaire: \\
		$f: I \rightarrow \mathbb{R}$, on suppose f admet un $DL_n(0)$, soit $\lambda \in \mathbb{R}, \alpha \in \mathbb{Z}^{*}$ \\
		$\lambda x^{\alpha} f(x) = \lambda a_0 x^{\alpha} + \hdots + \lambda a_n x^{\alpha+n} + o(x^{\alpha + n})$
		\item Composition \'el\'ementaire: \\
		$f: I \rightarrow \mathbb{R}$, on suppose f admet un $DL_n(0)$, soit $\lambda \in \mathbb{R}, \alpha \in \mathbb{N}^*$ \\
		$f(\lambda x^{\alpha}) = a_0 + a_1 \lambda x^{\alpha} + \hdots + \lambda ^n a_n x^{\alpha n} + o(x^{\alpha n})$
	\end{enumerate}
\end{rqs}

\begin{theo} Int\'egration des DL \\
	$f: I \rightarrow \mathbb{R}$, f d\'erivable. Supposons que $f'$ admet un $DL_n(0)$: au V(0)\\
	$f'(x) = a_0 + \hdots + a_n x^{n} + o(x^n)$, alors:
	\[f(x) = f(0) + a_0 x + a_1 \frac{x^2}{2} + \hdots + a_n \frac{x^{n+1}}{n+1} + o(x^{n+1})\]
\end{theo}

\begin{theo} D\'erivation des DL \\
	On obtient la partie r\'eguli\`ere du $DL_n(0)$ de $f'$ en d\'erivant la partie r\'eguli\`ere du $DL_{n+1}(0)$ de $f$
\end{theo}

%Opérations sur les DL {{{1
\section{Op\'erations sur les DL}
\begin{theo} Combinaison lin\'eaire\\
	$f,g: I \rightarrow \mathbb{R}$ admettent des $DL_n(0)$ de parties r\'eguli\`eres respectives $P$ et $Q$. \\
	Alors, $\lambda \in \mathbb{R},\ \lambda f + g$ admet un $DL_n(0)$ dont la partie r\'eguli\`ere est $\lambda P + Q$
\end{theo}

\begin{theo} Produit \\
	$f,g: I \rightarrow \mathbb{R}$ admettent des $DL_n(0)$ de parties r\'eguli\`eres respectives $P$ et $Q$. \\
	Alors $fg$ admet un $DL_n(0)$ dont la partie r\'eguli\`ere est le polyn\^ome $PQ$ tronqu\'e au degr\'e n
\end{theo}

\begin{theo} Composition \\
	$g: J \rightarrow \mathbb{R}$ admet un $DL_n(0)$ de partie r\'eguli\`ere $Q$ \\
	$f: I \rightarrow J$ v\'erifie que $f(0) = 0$ et admet un $DL_n(0)$ de partie r\'eguli\`ere $P$ \\
	Alors, $g \circ f$ admet un $DL_n(0)$ dont la partie r\'eguli\`ere est le polyn\^ome $Q \circ P$ tronqu\'e au degr\'e n
\end{theo}

\begin{theo} Appliquation d'un inverse \\
	$f: I \rightarrow \mathbb{R}$ admet un $DL_n(0)$ et $f(0) \neq 0$ \\
	Alors $\frac{1}{f}$ admet aussi un $DL_n(0)$ obtenu en composant les DL de $u \mapsto \frac{1}{1+u}$ et $x \mapsto g(x)$
	\[\frac{1}{f(x)} = \frac{1}{a_0} \frac{1}{1 + g(x)}\]
\end{theo}

%Utilisations des DL {{{1
\section{Utilisation des DL}
\begin{enumerate}
	\item Recherche d'\'equivalents et de limites \\
		$f: I \rightarrow \mathbb{R}, x_0 \in I$, f admet un $DL_n(x_0)$ \\
		$f(x) =  a_0 + a_1(x-x_0) + \hdots + a_n(x-x_0)^n + o((x-x_0)^n)$ \\
		On suppose qu'au moins un des $a_i$ n'est pas nul. Soit $p$ le minimum de l'ensemble des $a_i$ non nuls. \\
		Alors $f(x) \underset{x_0}{\sim} a_p(x-x_0)^p$
	\item Recherche d'asymptotes \\
		Pour trouver une aymptote on essaie de faire un developpement asymptotique de $x \mapsto \frac{f(x)}{x}$ au $V(+\infty)$, ce qui revient
		\`a faire un $DL(0)$ de $\varphi \mapsto h f(\frac{1}{h})$. \\
		Apr\`es on effectue une composition \'el\'ementaire par l'inverse de $\varphi$ puis on multiplie par $x$.
\end{enumerate}

%DL Classiques {{{1
\section{DL classiques}
\begin{itemize}
	\item [-] $\sin x = x - \frac{x^3}{6} + \frac{x^5}{120} + \hdots + (-1)^n \frac{x^{2n+1}}{(2n+1)!} + o(x^{2n+1})$
	\item [-] $\cos x = 1 - \frac{x^2}{2} + \frac{x^4}{24} + \hdots + (-1)^n \frac{x^{2n}}{(2n)!} + o(x^{2n})$
	\item [-] $\sinh x = x + \frac{x^3}{6} + \frac{x^5}{120} + \hdots + \frac{x^{2n+1}}{(2n+1)!} + o(x^{2n+1})$
	\item [-] $\cosh x = 1 + \frac{x^2}{2} + \frac{x^4}{24} + \hdots + \frac{x^{2n}}{(2n)!} + o(x^{2n})$
	\item [-] $e^x = 1 + x + \frac{x^2}{2} + \frac{x^3}{6} + \hdots + \frac{x^n}{n!} + o(x^n)$
	\item [-] $\ln(1+x) = x - \frac{x^2}{2} + \frac{x^3}{3} + \hdots + (-1)^{n-1} \frac{x^n}{n} + o(x^n)$
	\item [-] $(1+x)^{\alpha \in \mathbb{R}} = 1 + \alpha x + \frac{\alpha(\alpha -1)}{2} x^2 + \hdots + \frac{\alpha(\alpha-1)\hdots
		(\alpha-n+1)}{n!} x^n + o(x^n)$
	\item [-] $\frac{1}{1-x} = 1 + x + \hdots + x^n + o(x^n)$
	\item [-] $\frac{1}{1+x} = 1 - x + x^2 + \hdots + (-1)^n x^n + o(x^n)$
	\item [-] $\tan x = x + \frac{x^3}{3} + \frac{2x^5}{15} + o(x^5)$
	\item [-] $\arcsin x = x + \frac{x^3}{6} + \frac{3x^5}{40} + o(x^5)$
	\item [-] $\arctan x = x - \frac{x^3}{3} + \frac{x^5}{5} + o(x^5)$
\end{itemize}

%Remarques {{{1
\section{Remarques}
\begin{enumerate}
	\item On peut se ramener \`a un DL en $0$ en additionant le point demand\'e. \\
	Si on demande $DL_n$ en 1 de $x \mapsto \cos(x)$, on cherche alors $DL_n(0)$ de $x \mapsto \cos(x+1)$
\end{enumerate}

\end{document}
