\documentclass[fleqn]{article}
\usepackage{amssymb}
\usepackage{amsmath}
\usepackage{amsthm}
\usepackage{verbatim}

\title{D\'eveloppements limit\'es}
\date{}

\theoremstyle{definition} \newtheorem*{defi}{D\'efinition}
\theoremstyle{definition} \newtheorem*{theo}{Th\'eor\`eme}
\theoremstyle{definition} \newtheorem*{prop}{Propri\'et\'e}
\theoremstyle{definition} \newtheorem*{coro}{Corollaire}
\theoremstyle{remark} \newtheorem*{rqs}{Remarques}

\begin{document}
\maketitle

\section{D\'efinitions}
\begin{defi}
	Soit $I$ un intervalle de $\mathbb{R}$, $f:\rightarrow \mathbb{R},\ x_0 \in I,\ n \in \mathbb{N}$ \\
	On dira que f admet un DL \`a l'ordre n au voisinage de $x_0$ s'il existe $a_0, \hdots, a_n \in \mathbb{R}$ tel que: \\
	\[f(x) - (a_0 + a_1(x-x_0) + \hdots + a_n(x-x_0)^n) = o((x-x_0)^n)\]
\end{defi}

\begin{prop} unicit\'e des coefficients \\
	Si $f$ a un $DL_n(x_0)$, il existe une unique liste $(a_0, \hdots, a_n) \in \mathbb{R}^{n+1}$ tel que au voisinage de $x_0$,
	\[f(x) = \sum_{i=0}^{n} a_i(x-x_0)^i  + o((x-x_0)^n)\]
\end{prop}

\begin{rqs}
	$\ $
	\begin{enumerate}
		\item Multiplication \'el\'ementaire: \\
		$f: I \rightarrow \mathbb{R}$, on suppose f admet un $DL_n(0)$, soit $\lambda \in \mathbb{R}, \alpha \in \mathbb{Z}^{*}$ \\
		$\lambda x^{\alpha} f(x) = \lambda a_0 x^{\alpha} + \hdots + \lambda a_n x^{\alpha+n} + o(x^{\alpha + n})$
		\item Composition \'el\'ementaire: \\
		$f: I \rightarrow \mathbb{R}$, on suppose f admet un $DL_n(0)$, soit $\lambda \in \mathbb{R}, \alpha \in \mathbb{N}^*$ \\
		$f(\lambda x^{\alpha}) = a_0 + a_1 \lambda x^{\alpha} + \hdots + \lambda ^n a_n x^{\alpha n} + o(x^{\alpha n})$
	\end{enumerate}
\end{rqs}

\begin{theo} Int\'egration des DL \\
	$f: I \rightarrow \mathbb{R}$, f d\'erivable. \\
	Supposons que $f'$ admet un $DL_n(0)$: au voisinage de $x_0$, \\
	$f'(x) = a_0 + \hdots + a_n x^{n} + o(x^n)$ \\
	Alors, au voisinage de $0$: 
	\[f(x) = f(0) + a_0 x + a_1 \frac{x^2}{2} + \hdots + a_n \frac{x^{n+1}}{n+1} + o(x^{n+1})\]
\end{theo}


\end{document}
