\documentclass[fleqn]{article}
\usepackage{amssymb}
\usepackage{amsmath}
\usepackage{amsthm}
\usepackage{verbatim}
\usepackage{booktabs}

\title{Topologie dans $\mathbb{R}$}
\date{}

\theoremstyle{definition} \newtheorem*{defi}{D\'efinition}
\theoremstyle{definition} \newtheorem*{theo}{Th\'eor\`eme}
\theoremstyle{definition} \newtheorem*{adh}{Caract\'erisation s\'equentielle de l'adh\'erence}
\theoremstyle{definition} \newtheorem*{prop}{Propri\'et\'e}
\theoremstyle{definition} \newtheorem*{fermitude}{Caract\'erisation s\'equentielle de la fermitude}
\newcommand{\ra}[1]{\renewcommand{\arraystretch}{#1}}
\ra{1.3}

\begin{document}
\maketitle

\section{Voisinages}
\begin{defi} Voisinage \\
	Soit $x$ un r\'eel et $A \subset \mathbb{R}$. $A$ est voisinage de $x$ s'il existe $\epsilon > 0$ tel que:
	\[[x-\epsilon, x+\epsilon] \subset A\]
\end{defi}
Pour $x \in \mathbb{R}$ on note $V_\mathbb{R}(x)$ l'ensemble des voisinages de $x$ dans $\mathbb{R}$

\begin{prop} $ $
	\begin{itemize}
		\item [-] $A \in V_\mathbb{R}(x)$ et $A \subset B \Rightarrow B \in V_\mathbb{R}(x)$
		\item [-] Une r\'eunion quelconque de voisinages de $x$ est encore un voisinage de $x$
		\item [-] Une intersection finie de voisinages de $x$ l'est aussi
	\end{itemize}
\end{prop}

\section{Ouverts et ferm\'es}
\begin{defi}
	Soit $A \subset \mathbb{R}$,
	\begin{enumerate}
		\item On dit que $A$ est un ouvert si:
			\[\forall x, x \in A \Rightarrow A \in V_\mathbb{R}(x)\]
		\item On dit que $A$ est un ferm\'e lorsque $A^c$ est un ouvert
	\end{enumerate}
\end{defi}

\begin{prop} $ $
	\begin{itemize}
		\item [-] Une r\'eunion quelconque d'ouverts est un ouvert
		\item [-] Une intersection finie d'ouverts est un ouvert
		\item [-] Une r\'eunion finie de ferm\'es est un ferm\'e
		\item [-] Une intersection quelconque de ferm\'es est un ferm\'e
	\end{itemize}
\end{prop}

\section{Adh\'erence, int\'erieur}
\begin{defi}
	Soit $A \subset \mathbb{R}$
	\begin{enumerate}
		\item $x \in \mathbb{R}$ est int\'erieur \`a $A$ si $A$ est voisinage de $x$
		\item $x \in \mathbb{R}$ est un point adh\'erent \`a $A$ si pour tout voisinage $V$ de $x$, $V \cap A \neq \emptyset$
	\end{enumerate}
\end{defi}

\begin{prop} $ $
	\begin{itemize}
		\item [-] Adh\`erence, int\'erieur d'intervalles: \\
		\begin{tabular}{@{}ll@{}}
				$\quad Int([a,b])$ & $\quad \quad Adh(]a,b[)$ \\
				$\quad = Int(]a,b])$ & $\quad \quad = Adh(]a,b])$ \\
				$\quad = Int([a,b[)$ & $\quad \quad = Adh([a,b[)$ \\
				$\quad = Int(]a,b[) = ]a,b[$ & $\quad \quad = Adh([a,b]) = [a,b]$ \\
		\end{tabular}
		\item [-] Soit $A$ une partie major\'ee de $\mathbb{R}$, alors $\sup A$ est adh\'erent \`a $A$ (analogue pour $\inf\ A$)
		\item [-] Soit $A \subset \mathbb{R}$ \\
			$A$ est ferm\'e $\Leftrightarrow A = Adh(A)$ \\
			$A$ est ouvert $\Leftrightarrow A = Int(A)$
	\end{itemize}
\end{prop}
\begin{prop} Caract\'erisations s\'equentielles
	\begin{enumerate}
		\item Adh\'erence: soit $A \subset \mathbb{R} (A \neq \emptyset), x \in \mathbb{R}$
			\[x \in Adh(A) \Leftrightarrow \exists\ (a_n)_{n \in \mathbb{N}} \text{ de points de A qui converge vers } x\]
		\item Fermitude: soit $A \subset \mathbb{R}$
			\[A \text{ ferm\'ee} \Leftrightarrow \text{Pour toute suite cv. d'\'el\'e de } A\ (a_n), \text{on a } lim_{n \to \infty} a_n \in A\]
	\end{enumerate}
\end{prop}
\begin{theo}
	Boel-Lebesque \\ Soit $A \subset \mathbb{R} (A \neq \emptyset)$, LASSE:
	\begin{enumerate}
		\item $A$ est ferm\'ee born\'ee (compacte)
		\item Pour toute suite $(a_n)$ d'\'el\'ements de $a$, il existe une sous-suite de $(a_n)$ qui converge vers un \'el\'ement de $A$
	\end{enumerate}
\end{theo}

\end{document}
