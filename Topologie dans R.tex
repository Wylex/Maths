\documentclass[fleqn]{article}
\usepackage{amssymb}
\usepackage{amsmath}

\title{Topologie dans $\mathbb{R}$}
\date{}

\begin{document}
\maketitle

\section{Voisinages}
D\'efinition: Soit $x$ un r\'eel et $A \subset \mathbb{R}$. On dit que $A$ est voisinage de $x$ s'il existe $\epsilon > 0$ tel que:
\[[x-\epsilon, x+\epsilon] \subset A\]
Pour $x \in \mathbb{R}$ on note $V_\mathbb{R}(x)$ l'ensemble des voisinages de $x$ dans $\mathbb{R}$
\begin{itemize}
	\item $A \in V_\mathbb{R}(x)$ et $A \subset B \Rightarrow B \in V_\mathbb{R}(x)$
	\item Une r\'eunion quelconque de voisinages de $x$ est encore un voisinage de $x$
	\item Une intersection finie de voisinages de $x$ l'est aussi
\end{itemize}

\section{Ouverts et ferm\'es}
D\'efinition: Soit $A \subset \mathbb{R}$
\begin{itemize}
	\item On dit que $A$ est un ouvert si:
		\[\forall x, x \in A \Rightarrow A \in V_\mathbb{R}(x)\]
	\item On dit que $A$ est un ferm\'e lorsque $A^c$ est un ouvert
\end{itemize}
\begin{enumerate}
	\item Une r\'eunion quelconque d'ouverts est un ouvert
	\item Une intersection finie d'ouverts est un ouvert
	\item Une r\'eunion finie de ferm\'es est un ferm\'e
	\item Une intersection quelconque de ferm\'es est un ferm\'e
\end{enumerate}

\section{Adh\'erence, int\'erieur}
D\'efinition: Soit $A \subset \mathbb{R}$
\begin{itemize}
	\item $x \in \mathbb{R}$ est int\'erieur \`a $A$ si $A$ est voisinage de $x$
	\item $x \in \mathbb{R}$ est un point adh\'erent \`a $A$ si pour tout voisinage $V$ de $x$, $V \cap A \neq \emptyset$
\end{itemize}
\begin{itemize}
	\item Adh, int d'intervalles
		\begin{align*}
			Int([a,b]) &= Int(]a,b]) \\
					   &= Int([a,b[) \\
					   &= Int(]a,b[) = ]a,b[
		\end{align*}
		\begin{align*}
			Adh(]a,b[) &= Adh(]a,b]) \\
					   &= Adh([a,b[) \\
					   &= Adh([a,b]) = [a,b]
		\end{align*}
	\item Soit $A$ une partie major\'ee de $\mathbb{R}$, alors sup $A$ est adh\'erent \`a $A$
	\item Caract\'erisation s\'equentielle de l'ah\'erence: \\
		Soit $A \subset \mathbb{R} (A \neq \emptyset), x \in \mathbb{R}$
		\[x \in Adh(A) \Leftrightarrow \text{il existe une suite} (a_n)_{n \in \mathbb{N}} \text{ de points de A qui converge vers } x\]
	\item Caract\'erisation s\'equentielle de la fermitude: \\ 
		Soit $A \subset \mathbb{R}$
		\[A \text{ferm\'ee} \Leftrightarrow \text{Pour toute suite cv. d'\'el\'e de } A (a_n), \text{on a } lim_{n \to \infty} a_n \in A\]
	\item Th\'or\`eme: Boel-Lebesque: \\
		Soit $A \subset \mathbb{R} (A \neq \emptyset)$, LASSE:
		\begin{enumerate}
			\item $A$ est ferm\'ee born\'ee (compacte)
			\item Pour toute suite $(a_n)$ d'\'el\'e. de $a$, il existe une sous-suite de $(a_n)$ qui cv. vers un \'el\'e. de $A$
		\end{enumerate}
\end{itemize}

\end{document}
