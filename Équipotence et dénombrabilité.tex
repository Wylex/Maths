%Préamble {{{1
\documentclass[fleqn]{article}

\usepackage{amssymb}
\usepackage{amsmath}
\usepackage{amsthm}
\usepackage{verbatim}
\usepackage{booktabs}
\usepackage{mathrsfs}
\usepackage{stmaryrd}

\theoremstyle{definition} \newtheorem*{defi}{D\'efinition}
\theoremstyle{definition} \newtheorem*{theo}{Th\'eor\`eme}
\theoremstyle{definition} \newtheorem*{coro}{Corollaire}
\theoremstyle{definition} \newtheorem*{nota}{Notation}
\theoremstyle{definition} \newtheorem*{vocab}{Vocabulaire}
\theoremstyle{remark} \newtheorem*{rqs}{Remarques}
\theoremstyle{definition} \newtheorem*{prop}{Propri\'et\'e}
\newcommand{\ra}[1]{\renewcommand{\arraystretch}{#1}}
\newcommand*{\bfrac}[2]{\genfrac{}{}{0pt}{}{#1}{#2}}
\ra{1.3}

\title{\'Equipotence et d\'enombrabilit\'e}
\date{}

\begin{document}
\maketitle

%Équipotence {{{1
\section{\'Equipotence}
\begin{defi}
	Deux ensembles sont \'equipotents s'ils sont en bijection. C'est une relation d'\'equivalence
\end{defi}

\begin{theo} $ $
	\begin{enumerate}
		\item Deux ensebles finis sont \'equipotents ssi ils ont le m\^eme cardinal
		\item Cantor-Bernstein (HP): S'il existe une injection $\alpha: A \rightarrow B$ et $\beta: B \rightarrow A$ alors $A$ et $B$ sont
			\'equipotents
	\end{enumerate}
\end{theo}

%Ensembles dénombrables {{{1
\section{Ensembles d\'enombrables}
\begin{defi}
	Un ensemble est d\'enombrable s'il est \'equipotent \`a $\mathbb{N}$
\end{defi}

\begin{prop} $ $
	\begin{enumerate}
		\item Un ensemble \'equipotent \`a un ensemble d\'enombrable est d\'enombrable
		\item Toute partie de $\mathbb{N}$ est soit fini, soit d\'enombrable. Ainsi $\mathbb{N}$ est le "plus petit" ensemble infini
		\item Caract\'erisation: Soit $A$ en ensemble, LASSE:
			\begin{enumerate}
				\item $A$ est fini ou d\'enombrable
				\item Il existe $\varphi: A \rightarrow \mathbb{N}$ injective
				\item $A$ est vide ou il existe $\psi: \mathbb{N} \rightarrow A$ surjective
			\end{enumerate}
	\end{enumerate}
\end{prop}

\subsection{Constructions d'ensembles d\'enombrables}
\begin{enumerate}
	\item $\mathbb{N} \times \mathbb{N}$ est d\'enombrable $\Rightarrow$ un produit cart\'esien de $p$ ensembles finis ou d\'enombrables est
		fini ou d\'enombrable
	\item Une r\'eunion fini ou d\'enombrable d'ensembles finis ou d\'enombrables est fini ou d\'enombrable
\end{enumerate}

%Exemples {{{1
\section{Exemples}
\begin{theo}
	$\mathbb{N}, \mathbb{N}^*, \mathbb{Z}, \mathbb{Q}, \mathbb{N}^p, \mathbb{Z}^p, \mathbb{Q}^p, \mathbb{N}^a\times \mathbb{Z}^b\times
	\mathbb{Q}^c$ o\`u $a+b+c>1$ sont d\'enombrables
\end{theo}

\begin{theo} $ $
	\begin{enumerate}
		\item $\mathcal{P}(\mathbb{N})$  est fini non d\'enombrable
		\item $\mathbb{R}$ n'est pas d\'enombrable
	\end{enumerate}
\end{theo}

%Complément {{{1
\section{Compl\'ement}
\begin{defi}
	$\alpha \in \mathbb{C}$ est alg\'ebrique s'il existe $P \in \mathbb{Q}[X] \backslash \{0\}$ tel que $P(\alpha) = 0$ \\
	$\alpha$ est trancendant s'il n'est pas alg\'ebrique
\end{defi}

\begin{prop} $ $
	\begin{enumerate}
		\item L'ensemble des nombres alg\'ebriques est d\'enombrable
		\item Il existe (bcp) de r\'eels transcendants
	\end{enumerate}
\end{prop}

\end{document}
