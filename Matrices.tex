%Préamble {{{1
\documentclass[fleqn]{article}

\usepackage{amssymb}
\usepackage{amsmath}
\usepackage{amsthm}
\usepackage{verbatim}
\usepackage{booktabs}
\usepackage{mathrsfs}

\theoremstyle{definition} \newtheorem*{defi}{D\'efinition}
\theoremstyle{definition} \newtheorem*{theo}{Th\'eor\`eme}
\theoremstyle{definition} \newtheorem*{coro}{Corollaire}
\theoremstyle{definition} \newtheorem*{nota}{Notation}
\theoremstyle{definition} \newtheorem*{vocab}{Vocabulaire}
\theoremstyle{remark} \newtheorem*{rqs}{Remarques}
\theoremstyle{definition} \newtheorem*{prop}{Propri\'et\'e}
\newcommand{\ra}[1]{\renewcommand{\arraystretch}{#1}}
\newcommand*{\bfrac}[2]{\genfrac{}{}{0pt}{}{#1}{#2}}
\ra{1.3}

\title{Matrices}
\date{}

\begin{document}
\maketitle

%Faits de base {{{1
\section{Faits de base}
\begin{defi} Soit $K$ un corps, $n,p \in \mathbb{N}^*$\\
	Une matrice \`a $n$ lignes et \`a $p$ colonnes \`a coefficients dans $K$ est une application de $X = [\![0,n]\!]\times[\![0,p]\!]$ dans
	$K$. On note $M_{n,p}(K)$ l'ensemble de ces matrices.

	\begin{rqs} $M_{n,p}(K)$ est muni d'une structure de Kev et $\dim M_{n,p} = np$
	\end{rqs}
\end{defi}

\begin{vocab} $ $
	\begin{enumerate}
		\item On note $t_M \in M_{p,n}(K)$ la transpos\'ee de $M \in M_{n,p}(K)$ d\'efinie par: $\forall (i,j) \in
		[\![0,n]\!]\times[\![0,p]\!]\ t_M[i,j] = M[j,i]$
		\item $M \in M_n(K)$ est sym\'etrique lorsque $t_M = M$
		\item $M \in M_n(K)$ est antysym\'etrique lorsque $t_M = -M$
	\end{enumerate}
\end{vocab}

%Matrice d'une A.L relativement à un couple de bases {{{1
\section{Matrice d'une A.L relativement à un couple de bases}
$E\ (\dim = p),\ F\ (\dim = n)$ deux Kev.\\ $B = (e_1, \hdots, e_p),\ C = (u_1, \hdots, u_n)$ bases de $E$ et de $F$ respectivement
\begin{defi} Pour $f \in \mathscr{L}(E,F)$\\
	On appelle matrice de $f$ relativement au couple $(B,C)$ et on note $Mat_{B,C}(f)$ la matrice $M \in M_{n,p}(K)$ d\'efinie par:\\
	Pour $1 \leq i \leq n$ et $1 \leq j \leq p$, $M[i,j]$ i-\`eme coordon\'ee du vecteur $f(e_j)$ dans $C$: \\
	\[f(e_j) = \sum_{i=1}^n M_{ij} u_i\ (1\leq j \leq p)\]

\begin{rqs} $\psi_{B,C}$ est un isomorphisme de Kev:
	\begin{align*}
		\psi_{B,C}: \mathscr{L}(E,F) &\rightarrow M_{n,p}(K)\\
		f & \mapsto Mat_{B,C}(f)
	\end{align*}
\end{rqs}
\end{defi}

%"Produit" matriciel {{{1
\section{"Produit" matriciel}
\begin{defi} Soient $p,q,r \in \mathbb{N}^*,\ A \in M_{p,q}(K),\ B \in M_{q,r}(K)$ \\
	On d\'efinit le produit $AB$, \'el\'ement de $M_{p,r}(K)$ par: \\
	$\left. \begin{array}{l}
		\forall i \in [\![0,p]\!] \\
		\forall j \in [\![0,r]\!]
	\end{array}\right. (AB)[i,j] = \sum_{k=1}^q A[i,k]B[k,j]$

\end{defi}

\begin{prop} $ $
	\begin{enumerate}
		\item $\underset{B}{E} \overset{f}{\rightarrow} \underset{C}{F} \overset{g}{\rightarrow} \underset{D}{G}$ \\
			Alors, $Mat_{B,D}(g\circ f) = Mat_{C,D}(g) Mat_{B,C}(f)$
		\item
			\begin{enumerate}
				\item $A_1, A_2 \in M_{p,q}(K),\ B \in M_{q,r}(K),\ \alpha \in K$ Alors:\\ $(\alpha A_1 + A_2)B = \alpha_1 A_1 B + A_2 B$
				\item $B_1, B_2 \in M_{q,r}(K),\ A \in M_{p,q}(K),\ \alpha \in K$ Alors:\\ $A(\alpha B_1 + B_2) = \alpha A B_1 + A B_2$
			\end{enumerate}
		\item $A \in M_{p,q}(K),\ B \in M_{q,r}(K),\ C \in M_{r,s}(K)$ Alors $(AB)C = A(BC)$
		\item $A \in M_{p,q}(K),\ B \in M_{q,r}(K),\ \alpha \in K$ Alors $\alpha (AB) = (\alpha A)B = A (\alpha B)$
		\item $t(AB) = t_B t_A$
	\end{enumerate}
\end{prop}

\end{document}
