%Préamble {{{1
\documentclass[fleqn]{article}

\usepackage{amssymb}
\usepackage{amsmath}
\usepackage{amsthm}
\usepackage{verbatim}
\usepackage{booktabs}
\usepackage{mathrsfs}
\usepackage{stmaryrd}

\theoremstyle{definition} \newtheorem*{defi}{D\'efinition}
\theoremstyle{definition} \newtheorem*{theo}{Th\'eor\`eme}
\theoremstyle{definition} \newtheorem*{coro}{Corollaire}
\theoremstyle{definition} \newtheorem*{nota}{Notation}
\theoremstyle{definition} \newtheorem*{vocab}{Vocabulaire}
\theoremstyle{remark} \newtheorem*{rqs}{Remarques}
\theoremstyle{definition} \newtheorem*{prop}{Propri\'et\'e}
\newcommand{\ra}[1]{\renewcommand{\arraystretch}{#1}}
\newcommand*{\bfrac}[2]{\genfrac{}{}{0pt}{}{#1}{#2}}
\ra{1.3}

\title{Matrices}
\date{}

\begin{document}
\maketitle

%Faits de base {{{1
\section{Faits de base}
\begin{defi} Soit $K$ un corps, $n,p \in \mathbb{N}^*$\\
	Une matrice \`a $n$ lignes et \`a $p$ colonnes \`a coefficients dans $K$ est une application de $X = \llbracket 0,n \rrbracket \times
	\llbracket 0,p \rrbracket $ dans $K$. On note $M_{n,p}(K)$ l'ensemble de ces matrices.

	\begin{rqs} $M_{n,p}(K)$ est muni d'une structure de Kev et $\dim M_{n,p} = np$
	\end{rqs}
\end{defi}

\begin{vocab} $ $
	\begin{enumerate}
		\item On note $t_M \in M_{p,n}(K)$ la transpos\'ee de $M \in M_{n,p}(K)$ d\'efinie par: $\forall (i,j) \in
			\llbracket 0,n \rrbracket \times \llbracket 0,p \rrbracket ,\ t_M[i,j] = M[j,i]$
		\item $M \in M_n(K)$ est sym\'etrique lorsque $t_M = M$
		\item $M \in M_n(K)$ est antysym\'etrique lorsque $t_M = -M$
	\end{enumerate}
\end{vocab}

\subsection{Matrice d'une A.L relativement à un couple de bases}
$E\ (\dim = p),\ F\ (\dim = n)$ deux Kev.\\ $\mathscr{B} = (e_1, \hdots, e_p),\ \mathscr{C} = (u_1, \hdots, u_n)$ bases de $E$ et
de $F$ respectivement
\begin{defi} Pour $f \in \mathscr{L}(E,F)$\\
	On appelle matrice de $f$ relativement au couple $(\mathscr{B},\mathscr{C})$ et on note $Mat_{\mathscr{B},\mathscr{C}}(f)$ la matrice
	$M \in M_{n,p}(K)$ d\'efinie par:\\ Pour $1 \leq i \leq n$ et $1 \leq j \leq p$, $M[i,j]$ i-\`eme coordon\'ee du vecteur $f(e_j)$ dans
	$\mathscr{C}$: \\
	\[f(e_j) = \sum_{i=1}^n M_{ij} u_i\ (1\leq j \leq p)\]

\begin{theo} $\psi_{\mathscr{B},\mathscr{C}}$ est un isomorphisme de Kev:
	\begin{align*}
		\psi_{\mathscr{B},\mathscr{C}}: \mathscr{L}(E,F) &\rightarrow M_{n,p}(K)\\
		f & \mapsto Mat_{\mathscr{B},\mathscr{C}}(f)
	\end{align*}
\end{theo}
\end{defi}

\subsection{"Produit" matriciel}
\begin{defi} Soient $p,q,r \in \mathbb{N}^*,\ A \in M_{p,q}(K),\ B \in M_{q,r}(K)$ \\
	On d\'efinit le produit $AB \in M_{p,r}(K)$ par: \\
	$\left. \begin{array}{l}
		\forall i \in  \llbracket 0,p \rrbracket  \\
		\forall j \in  \llbracket 0,r \rrbracket
	\end{array}\right. (AB)[i,j] = \sum_{k=1}^q A[i,k]B[k,j]$
\end{defi}

\begin{prop} $ $
	\begin{enumerate}
		\item [-] Soient $\underset{\mathscr{B}}{E} \overset{f}{\rightarrow} \underset{\mathscr{C}}{F} \overset{g}{\rightarrow}
			\underset{\mathscr{D}}{G}$: \\
			Alors, $Mat_{\mathscr{B},\mathscr{D}}(g\circ f) = Mat_{\mathscr{C},\mathscr{D}}(g) Mat_{\mathscr{B},\mathscr{C}}(f)$
		\item [-]
			\begin{enumerate}
				\item $A_1, A_2 \in M_{p,q}(K),\ B \in M_{q,r}(K),\ \alpha \in K$ Alors:\\ $(\alpha A_1 + A_2)B = \alpha A_1 B + A_2 B$
				\item $A \in M_{p,q}(K),\ B_1, B_2 \in M_{q,r}(K),\ \alpha \in K$ Alors:\\ $A(\alpha B_1 + B_2) = \alpha A B_1 + A B_2$
			\end{enumerate}
		\item [-] $A \in M_{p,q}(K),\ B \in M_{q,r}(K),\ C \in M_{r,s}(K)$: Alors $(AB)C = A(BC)$
		\item [-] $A \in M_{p,q}(K),\ B \in M_{q,r}(K),\ \alpha \in K$: Alors $\alpha (AB) = (\alpha A)B = A (\alpha B)$
		\item [-] $t(AB) = t_B t_A$
		\item [-] Si $x \in E$ et si $X \in M_{q,1}(K)$ est la colonne des composantes de $x$ dans $\mathscr{B}$ alors $MX$ est la colonne des
			composantes de $f(x)$ dans $\mathscr{C}$
	\end{enumerate}
\end{prop}

\begin{prop} Produits particuliers
	\begin{enumerate}
		\item Par une matrice \'el\'ementaire:
			\begin{enumerate}
				\item $E_{ij} A$: pour $k \neq i,\ l_k(E_{ij}A) = 0$ et $l_i (E_{ij}A) = l_j(A)$
				\item $A E_{ij}$: pour $k \neq j,\ c_k(A E_{ij}) = 0$ et $c_j(A E_{ij}) = c_i(A)$
			\end{enumerate}
		\item Par une matrice diagonale:
			\begin{enumerate}
				\item $l_i(DA) = \alpha_i l_i(A)$
				\item $c_j(AD) = \alpha_j c_j(A)$
			\end{enumerate}
	\end{enumerate}
\end{prop}

%La K-algèbre Mn(K) {{{1
\section{La K-alg\`ebre $M_n(K)$}
$(M_n(K), +, \times, .)$ est une $K$ alg\`ebre de neutre l'identit\'e $I_n$

\subsection{Groupe des matrices inversibles}
On note $GL_n(K)$ le groupe des inversibles de l'anneau $(M_n(K), +, \times)$

\begin{prop} $ $
	\begin{enumerate}
		\item $A \in GL_n(K)$ on a $t_A \in GL_n(K)$ et $(t_A)^{-1} = t_{A^{-1}}$
		\item Soit $E$ un Kev de dimension $n$, $\mathscr{B}$ une base de $E$, $f \in \mathscr{L}(E)$:\\
			$f \in GL(E) \Leftrightarrow Mat_{\mathscr{B}}(f) \in GL_n(K)$ et $Mat_{\mathscr{B}}(f^{-1}) = Mat_{\mathscr{B}}(f)^{-1}$
		\item Soit $M \in M_n(K)$, LASSE:
			\begin{enumerate}
				\item $M \in GL_n(K)$
				\item $(c_1, \hdots, c_n)$ est libre
				\item $(l_1, \hdots, l_n)$ est libre
			\end{enumerate}
		\item Soit $A \in M_n(K)$, s'il existe $B \in M_n(K)$ tel que $AB = I_n$ (ou $BA = I_n$) alors $A \in GL_n(K)$ et $B = A^{-1}$
	\end{enumerate}
\end{prop}

\subsection{Sous alg\`ebres remarquables de $M_n(K)$}
\begin{defi} Sous alg\`ebre \\
	Soit $A$ une K-alg\`ebre. $B \subset A$ est une sous alg\`ebre si:
	\begin{enumerate}
		\item $B$ est un sev de $(A, +, .)$
		\item $B$ est un sous anneau de $(A,+,\times)$
	\end{enumerate}
\end{defi}

\begin{prop} $ $
	\begin{enumerate}
		\item $D_n(K)$ est une sous alg\`ebre de $M_n(K)$\\
			Soit $A \in D_n(K),\ A \in GL_n(K) \Leftrightarrow \forall i \in  \llbracket 1,n \rrbracket , \alpha_i \neq 0$\\
			Si c'est le cas, $Diag(\alpha_1, \hdots, \alpha_n)^{-1} = Diag(1/\alpha_1, \hdots, 1/\alpha_n)$
			\begin{rqs} $A,B \in D_n(K) \Rightarrow AB = BA$ \end{rqs}
		\item $TS_n(K)$ est une sous alg\`ebre de $M_n(K)$ \\
			Soit $A \in TS_n(K)$, $A \in GL_n(K) \Leftrightarrow \forall i \in  \llbracket 1,n \rrbracket ,\ A[i,i] \neq 0$
	\end{enumerate}
\end{prop}

%Formules de changement de base {{{1
\section{Formules de changement de base}
Soit $E$ un Kev de dimension finie $n \in \mathbb{N}^*$. $\mathscr{B} = (e_1, \hdots, e_n),\ \mathscr{B}' = (e_1', \hdots, e_n')$ \\
On appellera matrice de passage de $\mathscr{B}$ vers $\mathscr{B}'$, la matrice $M \in M_n(K)$ d\'efinie par: $\forall (i,j) \in  \llbracket 1,n \rrbracket ^2,
M_{ij}$ est la i\`eme coordonn\'e de $e_j'$ dans $\mathscr{B}$
\[e'_j = \sum_{i=1}^n M_{ij}e_i\]

\begin{rqs}
	La matrice de passage de $\mathscr{B}$ vers $\mathscr{B}'$ est $Mat_{\mathscr{B}',\mathscr{B}}(Id_E)$
\end{rqs}

\begin{prop} $ $
	\begin{enumerate}
		\item Soient $\mathscr{B},\ \mathscr{B}'$ deux bases du Kev $E$. Soit $P$ la matrice de passage
		de $\mathscr{B}$ vers $\mathscr{B}'$ et $Q$ de $\mathscr{B}'$ vers $\mathscr{B}$. Alors $P$ et $Q$ sont inversibles et inverses l'une de
		l'autre.
		\item Soient $\mathscr{B},\ \mathscr{B}'$ deux bases du Kev $E$, $x_1, \hdots, x_p \in E$. Soient $M = Mat_\mathscr{B}(x_1, \hdots, x_p),
		M' = Mat_{\mathscr{B}'}(x_1, \hdots, x_p),\ P$ la matrice de passage de $\mathscr{B}$ vers $\mathscr{B}'$: $M = PM'$
		\item Soient $E,F$ deux Kev, $\mathscr{B}, \mathscr{B}'$ bases de $E$, $\mathscr{C}, \mathscr{C}'$ bases de $F$, $f \in \mathscr{L}(E,F)$.
		$P$ la matrice de passage de $\mathscr{B}$ vers $\mathscr{B}'$ (resp $Q$ avec $\mathscr{C}$, $\mathscr{C}'$), $M =
		Mat_{\mathscr{C}, \mathscr{B}}(f)$ et $M' = Mat_{\mathscr{B}', \mathscr{C}'}(f)$: $M' = Q^{-1}MP$
	\end{enumerate}
\end{prop}

\begin{defi} Similitude et \'equivalence
	\begin{enumerate}
		\item Soient $A,B \in M_n(K)$. On dit que $B$ est semblable \`a $A$ s'il existe $P \in GL_n(K)$ tel que $B=P^{-1}AP$
		\item Soient $A,B \in M_{n,p}(K)$On dit que $A$ et $B$ sont \'equivalentes s'il existe $P \in GL_p(K),\ Q \in GL_n(K)$ tels que $B = QAP$
	\end{enumerate}
\end{defi}

%Rang d'une matrice {{{1
\section{Rang d'une matrice}
\begin{defi} Rang \\
	Le rang d'une matrice $M \in M_{n,p}(K)$ est le rang de l'AL de $K^p$ dans $K^n$ canoniquement associ\'e \`a $M$
\begin{rqs}
	$rg\ M = rg\ (c_1(M), \hdots, c_p(M))$
\end{rqs}
\end{defi}

\begin{prop} $ $
	\begin{enumerate}
		\item Soit $E$ un Kev de dimension finie, $n \in \mathbb{N}^*, \mathscr{B} = (e_1, \hdots, e_n)$ une base de $E$, $p \in \mathbb{N}^*,
			x_1, \hdots, x_p \in E$ alors $rg\ (x_1, \hdots, x_p) = rg\ Mat_{\mathscr{B}}(x_1, \hdots, x_p)$
		\item Soit $E,F$ deux Kev de dimension $p,n \in \mathbb{N}^*$ et $\mathscr{B},\mathscr{C}$ deux bases, $f \in \mathscr{L}(E,F)$. Alors
			$rg\ f = rg\ Mat_{\mathscr{B},\mathscr{C}}(f)$
		\item  Soit $A \in M_{n,p}(K), B \in M_{p,q}(K)$, alors:
			\begin{enumerate}
				\item $rg\ (AB) \leq \min (rg\ A, rg\ B)$
				\item si $p=n$ et $A \in GL_p(K)$, $rg\ (AB) = rg\ B$
				\item Si $p=q$ et $B \in GL_q(K)$, $rg\ (AB) = rg\ A$
			\end{enumerate}
	\end{enumerate}
\end{prop}

\begin{theo}
	Soit $A \in M_{n,p}(K)$ et $r = rg\ A\ (r \leq \min (n,p))$. Alors $A \sim J_{n,p,r}$ \\
	Corollaire: $A,B \in M_{n,p}(K)$, alors $A \sim B \Leftrightarrow rg\ A = rg\ B$
\end{theo}

\begin{prop} Autre caract\'erisation du rang
	\begin{itemize}
		\item [-] S'il existe une sous matrice carr\'ee de $M$ inversible de taille $r$, $rg\ M \geq r$
		\item [-] Si de plus toute sous matrice carr\'ee de taille $r+1$ n'est pas inversible alors $rg\ M = r$
	\end{itemize}
\end{prop}

%Trace d'une matrice carrée {{{1
\section{Trace d'une matrice carr\'ee}
\begin{defi} Trace \\
	Pour $A = a_{ij} \in M_n(K)$, la trace de $A$ est le scalaire $Tr(A) = \sum_{i=1}^n a_{ii}$
\end{defi}

\begin{prop} $ $
	\begin{enumerate}
		\item $Tr$ est lin\'eaire
		\item Pour $A \in M_n(K)$, $Tr(t_A) = Tr(A)$
		\item $\forall A,B \in M_n(K), Tr(AB) = Tr(BA)$ \\
			Ainsi, si $A$ et $B$ sont semblables, $Tr(A) = Tr(B)$
	\end{enumerate}
\end{prop}

\begin{theo}
	Soit $E$ un Kev de dimension finie, $n \in \mathbb{N}^*, f \in \mathscr{L}(E)$. Si $\mathscr{B}$ et $\mathscr{B}'$ sont deux bases de $E$,
	alors $Tr\ Mat_{\mathscr{B}}(f) = Tr\ Mat_{\mathscr{B}'}(f)$ \\
	Ainsi, la trace de $f$ est alors la trace de $Mat_{\mathscr{B}}(f)$ o\`u $\mathscr{B}$ est une base qlq de $E$
\end{theo}

%Remaques {{{1
\section{Remarques}

\begin{enumerate}
	\item Lorsqu'on demande de trouver des matrices particuli\`eres, on peut essayer en multipliant par la base canonique pour r\'eduire leur
		nombre et proc\'eder par analyse synth\`ese.
	\item Il peut \^etre int\'eressant de transformer un probl\`eme sur les matrices en un probl\`eme sur les fonctions gr\^ace aux fonctions
		canoniquement associ\'ees.
	\item $M = \frac{1}{2}(M +\ ^t M) + \frac{1}{2}(M -\ ^t M)$
\end{enumerate}

\end{document}
