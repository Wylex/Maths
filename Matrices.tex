%Préamble {{{1
\documentclass[fleqn]{article}

\usepackage{amssymb}
\usepackage{amsmath}
\usepackage{amsthm}
\usepackage{verbatim}
\usepackage{booktabs}
\usepackage{mathrsfs}

\theoremstyle{definition} \newtheorem*{defi}{D\'efinition}
\theoremstyle{definition} \newtheorem*{theo}{Th\'eor\`eme}
\theoremstyle{definition} \newtheorem*{coro}{Corollaire}
\theoremstyle{definition} \newtheorem*{nota}{Notation}
\theoremstyle{definition} \newtheorem*{vocab}{Vocabulaire}
\theoremstyle{remark} \newtheorem*{rqs}{Remarques}
\theoremstyle{definition} \newtheorem*{prop}{Propri\'et\'e}
\newcommand{\ra}[1]{\renewcommand{\arraystretch}{#1}}
\newcommand*{\bfrac}[2]{\genfrac{}{}{0pt}{}{#1}{#2}}
\ra{1.3}

\title{Matrices}
\date{}

\begin{document}
\maketitle

%Faits de base {{{1
\section{Faits de base}
\begin{defi} Soit $K$ un corps, $n,p \in \mathbb{N}^*$\\
	Une matrice \`a $n$ lignes et \`a $p$ colonnes \`a coefficients dans $K$ est une application de $X = [\![0,n]\!]\times[\![0,p]\!]$ dans
	$K$. On note $M_{n,p}(K)$ l'ensemble de ces matrices.

	\begin{rqs} $M_{n,p}(K)$ est muni d'une structure de Kev et $\dim M_{n,p} = np$
	\end{rqs}
\end{defi}

\begin{vocab} $ $
	\begin{enumerate}
		\item On note $t_M \in M_{p,n}(K)$ la transpos\'ee de $M \in M_{n,p}(K)$ d\'efinie par: $\forall (i,j) \in
		[\![0,n]\!]\times[\![0,p]\!],\ t_M[i,j] = M[j,i]$
		\item $M \in M_n(K)$ est sym\'etrique lorsque $t_M = M$
		\item $M \in M_n(K)$ est antysym\'etrique lorsque $t_M = -M$
	\end{enumerate}
\end{vocab}

%Matrice d'une A.L relativement à un couple de bases {{{1
\subsection{Matrice d'une A.L relativement à un couple de bases}
$E\ (\dim = p),\ F\ (\dim = n)$ deux Kev.\\ $\mathscr{B} = (e_1, \hdots, e_p),\ \mathscr{C} = (u_1, \hdots, u_n)$ bases de $E$ et
de $F$ respectivement
\begin{defi} Pour $f \in \mathscr{L}(E,F)$\\
	On appelle matrice de $f$ relativement au couple $(\mathscr{B},\mathscr{C})$ et on note $Mat_{\mathscr{B},\mathscr{C}}(f)$ la matrice
	$M \in M_{n,p}(K)$ d\'efinie par:\\ Pour $1 \leq i \leq n$ et $1 \leq j \leq p$, $M[i,j]$ i-\`eme coordon\'ee du vecteur $f(e_j)$ dans $C$: \\
	\[f(e_j) = \sum_{i=1}^n M_{ij} u_i\ (1\leq j \leq p)\]

\begin{rqs} $\psi_{\mathscr{B},\mathscr{C}}$ est un isomorphisme de Kev:
	\begin{align*}
		\psi_{\mathscr{B},\mathscr{C}}: \mathscr{L}(E,F) &\rightarrow M_{n,p}(K)\\
		f & \mapsto Mat_{\mathscr{B},\mathscr{C}}(f)
	\end{align*}
\end{rqs}
\end{defi}

%"Produit" matriciel {{{1
\subsection{"Produit" matriciel}
\begin{defi} Soient $p,q,r \in \mathbb{N}^*,\ A \in M_{p,q}(K),\ B \in M_{q,r}(K)$ \\
	On d\'efinit le produit $AB$, \'el\'ement de $M_{p,r}(K)$ par: \\
	$\left. \begin{array}{l}
		\forall i \in [\![0,p]\!] \\
		\forall j \in [\![0,r]\!]
	\end{array}\right. (AB)[i,j] = \sum_{k=1}^q A[i,k]B[k,j]$

\end{defi}

\begin{prop} $ $
	\begin{enumerate}
		\item [-] Soient $\underset{\mathscr{B}}{E} \overset{f}{\rightarrow} \underset{\mathscr{C}}{F} \overset{g}{\rightarrow}
			\underset{\mathscr{D}}{G}$: \\
			Alors, $Mat_{\mathscr{B},\mathscr{D}}(g\circ f) = Mat_{\mathscr{C},\mathscr{D}}(g) Mat_{\mathscr{B},\mathscr{C}}(f)$
		\item [-]
			\begin{enumerate}
				\item $A_1, A_2 \in M_{p,q}(K),\ B \in M_{q,r}(K),\ \alpha \in K$ Alors:\\ $(\alpha A_1 + A_2)B = \alpha A_1 B + A_2 B$
				\item $B_1, B_2 \in M_{q,r}(K),\ A \in M_{p,q}(K),\ \alpha \in K$ Alors:\\ $A(\alpha B_1 + B_2) = \alpha A B_1 + A B_2$
			\end{enumerate}
		\item [-] $A \in M_{p,q}(K),\ B \in M_{q,r}(K),\ C \in M_{r,s}(K)$: Alors $(AB)C = A(BC)$
		\item [-] $A \in M_{p,q}(K),\ B \in M_{q,r}(K),\ \alpha \in K$: Alors $\alpha (AB) = (\alpha A)B = A (\alpha B)$
		\item [-] $t(AB) = t_B t_A$
	\end{enumerate}
\end{prop}

\begin{prop} Produits particuliers
	\begin{enumerate}
		\item Par une matrice \'el\'ementaire:
			\begin{enumerate}
				\item $E_{ij} A$: pour $k \neq i,\ l_k(E_{ij}A) = 0$ et $l_i (E_{ij}A) = l_j(A)$
				\item $A E_{ij}$: pour $k \neq j,\ c_k(A E_{ij}) = 0$ et $c_j(A E_{ij}) = c_i(A)$
			\end{enumerate}
		\item Par une matrice diagonale:
			\begin{enumerate}
				\item $l_i(DA) = \alpha_i l_i(A)$
				\item $c_j(AD) = \alpha_j c_j(A)$
			\end{enumerate}
	\end{enumerate}
\end{prop}

%La K-algèbre Mn(K) {{{1
\section{La K-alg\`ebre $M_n(K)$}
$(M_n(K), +, \times, .)$ est une $K$ alg\`ebre de neutre l'identit\'e

\subsection{Matrices inversibles}
On note $GL_n(K)$ le groupe des inversibles de l'anneau $(M_n(K), +, \times)$

\begin{prop} $ $
	\begin{enumerate}
		\item $A \in GL_n(K)$ on a $t_A \in GL_n(K)$ et $(t_A)^{-1} = t_{A^{-1}}$
		\item Soit $E$ un Kev de dimension $n$, $\mathscr{B}$ une base de $E$, $F \in \mathscr{L}(E)$:\\
			$f \in GL(E) \Leftrightarrow Mat_{\mathscr{B}}(f^{-1}) \in GL_n(K)$ et $Mat_{\mathscr{B}}(f^{-1}) = Mat_{\mathscr{B}}(f)^{-1}$
		\item Soit $M \in M_n(K)$, LASSE:
			\begin{enumerate}
				\item $M \in GL_n(K)$
				\item $(c_1, \hdots, c_n)$ est libre
				\item $(l_1, \hdots, l_n)$ est libre
			\end{enumerate}
		\item Soit $A \in M_n(K)$, s'il existe $B \in M_n(K)$ tel que $AB = I_n$ (ou $BA = I_n$) alors $A \in GL_n(K)$ et $B = A^{-1}$
	\end{enumerate}
\end{prop}

\subsection{Sous alg\`ebres remarquables de $M_n(K)$}
\begin{defi} Sous alg\`ebre \\
	Soit $A$ une K-alg\`ebre. $B \subset A$ est une sous alg\`ebre si:
	\begin{enumerate}
		\item $B$ est un sev de $(A, +, .)$
		\item $B$ est un sous anneau de $(A,+,\times)$
	\end{enumerate}
\end{defi}

\begin{prop} $ $
	\begin{enumerate}
		\item $D_n(K)$ est une sous alg\`ebre de $M_n(K)$\\
			Soit $A \in D_n(K),\ A \in GL_n(K) \Leftrightarrow \forall i \in [\![1,n]\!], \alpha_i \neq 0$\\
			Si c'est le cas, $Diag(\alpha_1, \hdots, \alpha_n)^{-1} = Diag(1/\alpha_1, \hdots, 1/\alpha_n)$
			\begin{rqs} $A,B \in D_n(K) \Rightarrow AB = BA$ \end{rqs}
		\item $TS_n(K)$ est une sous alg\`ebre de $M_n(K)$ \\
			Soit $A \in TS_n(K)$, $A \in GL_n(K) \Rightarrow \forall i \in [\![1,n]\!],\ A[i,i] \neq 0$
	\end{enumerate}
\end{prop}

\end{document}
