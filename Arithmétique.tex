%Préamble {{{1
\documentclass[fleqn]{article}

\usepackage{amssymb}
\usepackage{amsmath}
\usepackage{amsthm}
\usepackage{verbatim}
\usepackage{booktabs}
\usepackage{mathrsfs}

\theoremstyle{definition} \newtheorem*{defi}{D\'efinition}
\theoremstyle{definition} \newtheorem*{theo}{Th\'eor\`eme}
\theoremstyle{definition} \newtheorem*{coro}{Corollaire}
\theoremstyle{remark} \newtheorem*{rqs}{Remarques}
\theoremstyle{definition} \newtheorem*{prop}{Propri\'et\'e}
\newcommand{\ra}[1]{\renewcommand{\arraystretch}{#1}}
\newcommand*{\bfrac}[2]{\genfrac{}{}{0pt}{}{#1}{#2}}
\ra{1.3}

\title{Arithm\'etique dans $\mathbb{Z}$}
\date{}

\begin{document}
\maketitle

%Divisibilité dans Z {{{1
\section{Divisibilit\'e dans $\mathbb{Z}$}
On dit que $a$ divise $b$ s'il existe $c \in \mathbb{Z}$ tel que $b=ac$
\begin{prop} $\forall a \in \mathbb{Z}$
	\begin{enumerate}
		\item [-] $\bfrac{+}{-} 1 | a$ et $\bfrac{+}{-} a | a$
		\item [-] $a | b$ dans $\mathbb{Z} \Leftrightarrow |a|$ divise $|b|$ dans $\mathbb{N}$
		\item [-] $a | b$ et $b | a \Leftrightarrow b \in \{\bfrac{+}{-} a\}$
		\item [-] $a|b$ et $b|c \Rightarrow a | c$
		\item [-] $a|b$ et $a|c \Rightarrow a| bu + cv$
		\item [-] $ab |ac \Rightarrow b|c$
	\end{enumerate}
\end{prop}

\begin{theo} Division euclidienne dans $\mathbb{Z}$ \\
Soit $a \in \mathbb{Z},\ b \in \mathbb{Z}^*$. Alors il existe un unique $(q,r)$ tel que:
\begin{enumerate}
	\item $a = bq + r$
	\item $0 \leq r \leq |b|$
\end{enumerate}
\end{theo}

\begin{theo} Petit th\'eor\`eme de Fermat
	\begin{enumerate}
		\item $p \in \mathcal{P}$ alors $\forall m \in \mathbb{N},\ m^p \equiv m \pmod{p}$
		\item Si $m$ n'est pas divisible par $p$, $m^{p-1} \equiv 1 \pmod{p}$
	\end{enumerate}
\end{theo}

%PGCD {{{1
\section{PGCD}
\begin{defi}
	Soit $n \in \mathbb{N}^*,\ a_1, \hdots, a_n \in \mathbb{Z}$. Il existe un unique $ d \in \mathbb{N}$ tel que:
	\begin{enumerate}
		\item $d | a_i\ (1 \leq i \leq n)$
		\item pour $k \in \mathbb{Z},\ k | a_i\ (1 \leq i \leq n) \Rightarrow k | d$
	\end{enumerate}
\end{defi}

\pagebreak

\begin{prop} $ $
	\begin{enumerate}
		\item [-] B\'ezout V1:\\
			$\exists u_1, \hdots, u_n \in \mathbb{Z},\ d = \sum_{i = 1}^{n} a_iu_i$
		\item [-] $\land_{i = 1}^{n}a_i = \land_{i = 1}^n |a_i|$
		\item [-] $d = 0 \Leftrightarrow a_i = 0\ (1 \leq i \leq n)$
		\item [-] $a | b \Leftrightarrow a \land b = a$
		\item [-] $\forall a \in \mathbb{N}, a \land 0 = a$
		\item [-] Soit $p$ un nombre premier et $a \in \mathbb{Z}$. Alors $p|a$ ou $p \land a = 1$
		\item [-] Soit $k \in \mathbb{N}$. Alors $\land_{i = 1}^n (ka_i) = k(\land_{i = 1}^n a_i)$
	\end{enumerate}
\end{prop}

\subsection{Calcul de PGCD}
\begin{enumerate}
	\item Valuation p-adique \\
		$a, b \in \mathbb{N}^*$ et $d = a \land b$. On a alors $\forall p \in \mathcal{P},\ V_p(d) = \min(V_p(a), V_p(b))$
		\[d = \prod_{p \in \mathcal{P}} p^{\min(V_p(a), V_p(b))}\]
	\item Algorithme d'Euclide
\end{enumerate}

%PPCM {{{1
\section{PPCM}
\begin{defi} $a_1, \hdots, a_n \in \mathbb{Z}$. Il existe un unique $m \in \mathbb{N}$ tel que:
	\begin{enumerate}
		\item $a_i |m\ (1 \leq i \leq n)$
		\item $\forall k \in \mathbb{Z},\ a_i | k\ (1 \leq i \leq n) \Rightarrow m | k$
	\end{enumerate}
\end{defi}

\begin{prop} $ $
	\begin{enumerate}
		\item [-] $V_p(m) = \max (V_p(a_i))$
		\item [-] $\lor a_i = \lor |a_i|$
		\item [-] S'il existe $i$ telle que $a_i = 0$, $\lor a_i = 0$
		\item [-] $(\lor a_i) | a_1 \hdots a_n$
		\item [-] Si $\forall i \neq j,\ a_i \land a_j = 1,\ a_1 \hdots a_n | m \Rightarrow m = M$
	\end{enumerate}
\end{prop}

\begin{prop} Lien PGCD, PPCM \\
	$a, b \in \mathbb{N}^*\ d = a \land b,\ m = a \lor b$ \\
	Alors $md = ab$
\end{prop}

%Entiers premiers entre eux {{{1
\section{Entiers premiers entre eux}
\begin{defi}
	$n \in \mathbb{N}^*\ a_1, \hdots, a_n \in \mathbb{Z}$ \\
	On dit que $a_1, \hdots, a_n$ sont premiers entre eux si $\land_{i=1}^n a_i = 1$

	\begin{rqs}
		$a_1, \hdots, a_n$ premiers entre eux $\Leftrightarrow$ 1 est le seul entier qui divise tous les $a_i \Leftrightarrow$ il n'existe aucun
		nombre premier qui divise tous les $a_i$
	\end{rqs}
\end{defi}

\begin{theo} B\'ezout V2 \\
	$a_1, \hdots, a_n \in \mathbb{Z}$ premiers entre eux $\Leftrightarrow$ il existe $u_1, \hdots, u_n \in \mathbb{Z}$ tels que
	$\sum a_i u_i = 1$
\end{theo}

\begin{theo} Gauss \\
	$a,b,c \in \mathbb{Z}$ \\
	$\left. \begin{array}{l}
		a \land b = 1 \\
		a | bc
	\end{array}\right\} \Rightarrow a | c$

	\begin{rqs}
		Soit $p$ un nombre premier, $a,b \in \mathbb{Z}$ \\
		$p | ab \Rightarrow p|a$ ou $p|b$
	\end{rqs}
\end{theo}

\begin{prop} Variantes Gauss
	\begin{enumerate}
		\item
			$\left. \begin{array}{l}
				a \land b = 1 \\
				a | c \\
				b | c
			\end{array}\right\} \Rightarrow ab | c$
		\item
			$\left. \begin{array}{l}
				a \land c = 1 \\
				b \land c = 1
			\end{array}\right\} \Leftrightarrow ab \land c = 1$
			\begin{rqs}
				$a\land b = 1 \Rightarrow a^{\alpha} \land b^{\beta} = 1$
			\end{rqs}
	\end{enumerate}
\end{prop}

%Équation diophantienne {{{1
\section{\'Equation diophantienne}

%Remarques {{{1
\section{Remarques}
\begin{enumerate}
	\item Regarder si un nombre est pair, impair (ou autres diviseurs)
	\item D\'ecomposer en nombre premiers
	\item Prendre $p \in \mathcal{P},\ p | \land$
\end{enumerate}

\end{document}
