%Préamble {{{1
\documentclass[fleqn]{article}

\usepackage{amssymb}
\usepackage{amsmath}
\usepackage{amsthm}
\usepackage{verbatim}
\usepackage{booktabs}
\usepackage{mathrsfs}

\theoremstyle{definition} \newtheorem*{defi}{D\'efinition}
\theoremstyle{definition} \newtheorem*{theo}{Th\'eor\`eme}
\theoremstyle{definition} \newtheorem*{coro}{Corollaire}
\theoremstyle{definition} \newtheorem*{nota}{Notation}
\theoremstyle{remark} \newtheorem*{rqs}{Remarques}
\theoremstyle{definition} \newtheorem*{prop}{Propri\'et\'e}
\newcommand{\ra}[1]{\renewcommand{\arraystretch}{#1}}
\newcommand*{\bfrac}[2]{\genfrac{}{}{0pt}{}{#1}{#2}}
\ra{1.3}

\title{Polyn\^omes}
\date{}

\begin{document}
\maketitle

%Construction {{{1
\section{Construction}
Dans la suite, $K$ est un corps et $(K[X], + , \times)$ un anneau commutatif, int\'egre
\subsection{Degr\'e}
Pour $P =  \sum_{m \in \mathbb{N}} a_m X^m \in K[X]\backslash \{0\}$ \\
$\deg P = \max \{m \in \mathbb{N},\ a_m \neq 0\}$

\begin{itemize}
	\item [-] $\deg (P+Q) \leq \max (\deg P, \deg Q)$
	\item [-] $\deg (PQ) = \deg P + \deg Q$
\end{itemize}

\begin{rqs}
	$PQ = \sum_k C_k X^k$ avec $C_k = \sum_{l=0}^k a_l b_{k-l}$
\end{rqs}

\subsection{Inversibles}
$P$ est inversible si et seulement si $P = \lambda \in K^*$

\subsection{Divilibilit\'e}
$A, B \in K[X]$. On dit que $A$ divise $B$ s'il existe $C \in K[X]$ tel que $B = AC$

\begin{prop} $ $
	\begin{itemize}
		\item [-] $A | 0$
		\item [-] $\forall \lambda \in K^*, \lambda | A \land \lambda A | A$
		\item [-] $A | B$ et $B|C \Rightarrow A | C$
		\item [-]
			$\left. \begin{array}{l}
				A | B \\
				A | C
			\end{array}\right. \Rightarrow \forall U, V \in K[X],\ A |  BU + VC$
		\item [-] $A,B \in K[X],\ A|B$ et $B|A \Leftrightarrow \exists \lambda \in K^*,\ B = \lambda A$
		\item [-] $B$ est associ\'e \`a $A$ si $B$ s'\'ecrit $B = \lambda A,\ \lambda \in K^*$
		\item [-] $A | B \Rightarrow \deg A \leq \deg B$
		\item [-] $A|B$ et $\deg A = \deg B \Leftrightarrow B$ associ\'e \`a $A$
	\end{itemize}
\end{prop}

\begin{defi} Polyn\^omes irr\'eductibles \\
	$P \in K[X]$, on dit que $P$ est irr\'eductible si:
	\begin{enumerate}
		\item $P$ n'est pas constant ($\deg P \geq 1$)
		\item les seuls diviseurs de $P$ sont les \'el\'ements de $K^*$ et les associ\'es de $P$
	\end{enumerate}
\end{defi}

\begin{prop} Soit $P \in K[X]$:
	\begin{itemize}
		\item [-] Si $\deg P = 1$. Alors $P$ est irr\'eductible dans $K[X]$
		\item [-] Si $P$ est non constant ($\deg P \geq 1$):
			\begin{align*} P \text{ pas irr\'eductible } &\Leftrightarrow \text{ il existe un diviseur } S \text{ de } P \text{ tel que }
			 1 \leq \deg S < \deg P \\
			&\Leftrightarrow P \text{ s'\'ecrit } P = ST \text{ avec } \deg S \geq 1,\ \deg T \geq 1 \end{align*}
		\item [-] Tout $P \in K[X]$ non constant est produit d'irr\'eductibles
	\end{itemize}
\end{prop}

\subsection{Division euclidienne}
\begin{theo} Soit $A \in K[X],\ B \in K[X]\backslash \{0\}$
	Alors il existe un unique couple $(Q,R) \in K[X]^2$, appel\'e division euclidienne de $A$ par $B$ tel que:
	\begin{enumerate}
		\item $A = BQ + R$
		\item $\deg R < \deg B$
	\end{enumerate}
\end{theo}

\begin{prop} Invariance de la divison euclidienne par extension de corps \\
	Soit $K$ un sous corps de $L$ \\
	$B | A$ dans $K[X] \Leftrightarrow B | A$ dans $L[X]$
\end{prop}

%Racines {{{1
\section{Racines}
\subsection{Racines simples}
\begin{defi} Soit $P = \sum_k a_k X^k \in K[X]$ \\
	On appelle fonction polynomiale associ\'e \`a $P$ et on note $\tilde{P}$ l'application: \begin{align*} \tilde{P}: K &\rightarrow K\\
	t &\mapsto \sum_k a_kt^k \end{align*}
\end{defi}

\begin{prop} $ $
	\begin{itemize}
		\item [-] $\tilde{P+Q}(t) = \tilde{P}(t) + \tilde{Q}(t)$
		\item [-] $\tilde{PQ}(t) = \tilde{P}(t) \tilde{Q}(t)$
		\item [-] $\sim$ est un morphisme d'anneaux
	\end{itemize}
\end{prop}

\begin{defi} Racines \\
	On dit que $x$ est racine de $P$ si $\tilde{P}(x) = 0_K$
	\begin{rqs} $ $
		\begin{itemize}
			\item [-] Si $P$ est constant non nul, $P$ n'a pas de racines
			\item [-] Si $P$ est de degr\'e 1, $P$ a une unique racine
			\item [-] Si $P$ est de degr\'e 2, $P$ n'admet de racines que si $\Delta$ est un carr\'e dans $K$
		\end{itemize}
	\end{rqs}
\end{defi}

\begin{theo} $ $
	\begin{enumerate}
		\item $x$ racine de $P \Leftrightarrow X - x$ divise $P$ dans $K[X]$
		\item $x_i (1 \leq i \leq r)$ racine de $P \Leftrightarrow \prod (X - x_i)$ divise $P$ dans $K[X]$
		\item Soit $P \in K[X]$ de degr\'e $n$. Alors $P$ admet au plus $n$ racines distinctes
		\item Soit $n$ tel que $\deg P \leq n$. Si $P$ admet $n+1$ racines distinctes dans $K$ alors $P = 0$ \\
			Si $P$ admet une infinit\'e de racines dans $K$ alors $P = 0$
		\item $P,Q \in K[X]$. Soit $n$ tel que $\deg P \leq n$ et $\deg Q \leq n$ \\
			Si $\tilde{P}$ et $\tilde{Q}$ co\"incident en $n+1$ points distincts, alors $P=Q$
		\item On suppose $K$ infini, alors $P \in K[X] \mapsto \tilde{P} \in \mathcal{F}(K,K)$ est injective
		\item Soit $P \in K[X]$ de degr\'e $n \geq 1$. Supposons que $P$ admette $n$ racines distinctes $x_1, \hdots, x_n$ alors: \\
			$P = c\prod_{i=1}^n(X-x_i)$ avec $c = $ coeffdom($P$)
	\end{enumerate}
\end{theo}

\begin{theo} Polyn\^omes interpolateurs de Lagrange \\
	Soit $n\in \mathbb{N}$, $x_0, \hdots, x_n$ distincts dans $K$, $y_0, \hdots, y_n$ quelconques. Alors il existe un unique $P \in K[X]$
	de degr\'e $\leq n$, tel que $\tilde{P}(x_i) = y_i$
\end{theo}

\subsection{Racines multiples}

\begin{prop} $ $
	\begin{enumerate}
		\item Soit $P \in K[X]$ non constante, $r \in \mathbb{R}^*,\ x_1, \hdots, x_r$ distincts dans $K$ et $\alpha_1, \hdots, \alpha_r \in
		\mathbb{N}^*$\\
		Si pour tout $i$, $(X-x_i)^{\alpha_i} | P$ alors $\prod_{i=1}^r (X - x_i)^{\alpha_i} | P$
		\item Soit $P\in K[X]$ de degr\'e $n$. Alors $P$ admet au plus $n$ racines (chaque racine \'etant compt\'ee autant de fois que son ordre
		de multiplicit\'e)
	\end{enumerate}
\end{prop}

%Polynômes scindés, relations entre coeff et racines {{{1
\section{Polyn\^omes scind\'es, relations entre coeff et racines}
\begin{defi} Polyn\^omes scind\'es \\
	$P \in K[X]$ non constant est scind\'e sur $K$ si $P$ est produit de polyn\^omes de \mbox{degr\'e 1.}
	ie si $P$ s'\`ecrit $P = c(X-x_1)\hdots(X-x_n)$
\end{defi}

\begin{prop}
	Le coefficient de $X^{n-k}$ dans $P = (X+x_1) \hdots (X + x_n)$ est $\sum_{I \in P_k(n)} \prod_{j \in I} x_j$
\end{prop}

%Arithmétique {{{1
\section{Arithm\'etique dans $K[X]$}
\begin{defi} PGCD\\
	Soient $A,B \in K[X]$. Il existe un et un seul $D \in K[X]$, nul ou unitaire tel que:
	\begin{enumerate}
		\item $D | A$ et $D | B$
		\item $\forall P \in K[X],\ P|A$ et $P|B \Rightarrow P|D$
	\end{enumerate}
\end{defi}

\begin{prop} $ $
	\begin{enumerate}
		\item B\'ezout
		\item Gauss et variantes
	\end{enumerate}
\end{prop}

%Dérivation {{{1
\section{D\'erivation}
\begin{prop} $ $
	\begin{enumerate}
		\item $\deg D^k(P) = \deg P - k$
		\item $D(\alpha P + Q) = \alpha D(P) + D(Q)$ \\
			$D(PQ) = P'Q + PQ'$
		\item $D^k(PQ) = \sum_{i = 0}^k \binom{k}{i}P^{(i)}Q^{(k-i)}$
	\end{enumerate}
\end{prop}
\begin{theo} Taylor
	\[a \in K,\ P = \sum_k \frac{\tilde{P}^{(k)}(a)}{k!} (X - a)^k\]
\end{theo}

%Composition {{{1
\section{Composition}
\begin{prop} $ $
	\begin{enumerate}
		\item $\deg P(Q) = \deg P \times \deg Q$
		\item $(\alpha P + Q) \circ S = \alpha P \circ S + Q \circ S$ \\
				$(PQ) \circ S = (P \circ S)(Q \circ S)$
		\item $(P\circ Q) \circ S = P \circ (Q \circ S)$
		\item $(P \circ Q)' = Q' (P' \circ Q)$
	\end{enumerate}
\end{prop}

\end{document}
