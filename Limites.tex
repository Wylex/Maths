\documentclass[fleqn]{article}
\usepackage{amssymb}
\usepackage{amsmath}
\usepackage{amsthm}
\usepackage{verbatim}

\title{Limites}
\date{}

\theoremstyle{definition} \newtheorem*{defi}{D\'efinition}
\theoremstyle{definition} \newtheorem*{theo}{Th\'eor\`eme}
\theoremstyle{definition} \newtheorem*{adh}{Caract\'erisation s\'equentielle de l'adh\'erence}
\theoremstyle{remark} \newtheorem*{rqs}{Remarques}

\begin{document}
\maketitle

\section{Topologie dans $\mathbb{R}$}
\begin{defi}
	$\overline{\mathbb{R}} = \mathbb{R} \cup \{ +\infty, -\infty\}$
\end{defi}
\begin{itemize}
	\item Intervalles de $\overline{\mathbb{R}}$: ce sont les parties de $\overline{\mathbb{R}}$ du type ($a,b \in \overline{\mathbb{R}}$ et
		$a \leq b$): \\
		$[a,b]$ ou $[a,b[$ ou $]a,b[$
	\item Voisinage: Soit $x \in \overline{\mathbb{R}}, A \subset \overline{\mathbb{R}}$ \\
		On dit que $A$ est voisinage de $x$ dans $\overline{\mathbb{R}}$ si:
		\begin{itemize}
			\item[-] si $x \in \mathbb{R}, \exists \epsilon > 0$ tel que $[x - \epsilon, x + \epsilon] \subset A$
			\item[-] si $x = +\infty, \exists M \in \mathbb{R}$ tel que $[M, +\infty] \subset A$
			\item[-] si $x = - \infty, \exists M \in \mathbb{R}$ tel que $[-\infty, M] \subset A$
		\end{itemize}
	\item $\overline{\mathbb{R}}$ est s\'epar\'ee: \\
		Pour $x, y \in \overline{\mathbb{R}}$ avec $x \neq y$, il existe $U \in V_{\overline{\mathbb{R}}}(x), V \in V_{\overline{\mathbb{R}}}$
		tel que: $U \cap V \neq \emptyset$
	\item Propri\'et\'e: Soit $(u_n)$ une suite r\'eelle et $l \in \overline{\mathbb{R}}$. On a: \\
		$u_n \rightarrow l \Leftrightarrow \forall\ V \in V_{\overline{\mathbb{R}}}(l), \exists n_0 \in \mathbb{N}, \forall n \geq n_0, u_n
		\in V$
	\item Point adh\'erent: \\
		Soit $A \subset \overline{\mathbb{R}}, x \in \overline{\mathbb{R}}$. On dit que $x$ est adh\'erent \`a $A$ dans
		$\overline{\mathbb{R}}$ si: \\
		$\forall\ V \in V_{\overline{\mathbb{R}}}, A \cap V \neq \emptyset$
\end{itemize}
\begin{rqs}
	Soit $D \subset \mathbb{R}, x \in \overline{\mathbb{R}}$
	\begin{enumerate}
		\item si $x \in \mathbb{R}$ alors: \\
			$x$ adh\'erent \`a $D$ dans $\overline{\mathbb{R}} \Leftrightarrow x$ adh\'erent \`a $D$ dans $\mathbb{R}$
		\item si $x = +\infty,$ \\
			$+\infty$ adh\'erent \`a $D$ dans $\overline{\mathbb{R}} \Leftrightarrow D$ non major\'ee dans $\mathbb{R}$
		\item si $x = -\infty,$ \\
			$-\infty$ adh\'erent \`a $D$ dans $\overline{\mathbb{R}} \Leftrightarrow D$ non minor\'ee dans $\mathbb{R}$
	\end{enumerate}
\end{rqs}
\begin{adh}
	Soit $A \subset \mathbb{R}, x \in \overline{\mathbb{R}}$,
	\[x \in Adh_{\overline{\mathbb{R}}}(A) \Leftrightarrow \text{ il existe une suite } (a_n) \text{ de points de } A \text{ telle que }
	a_n \rightarrow x\]
\end{adh}

\section{Notion de limite}

\end{document}
