%Préamble {{{1
\documentclass[fleqn]{article}

\usepackage{amssymb}
\usepackage{amsmath}
\usepackage{amsthm}
\usepackage{verbatim}

\theoremstyle{definition} \newtheorem*{defi}{D\'efinition}
\theoremstyle{definition} \newtheorem*{theo}{Th\'eor\`eme}
\theoremstyle{definition} \newtheorem*{prop}{Propri\'et\'e}
\theoremstyle{remark} \newtheorem*{rqs}{Remarques}

\title{Limites}
\date{}

\begin{document}
\maketitle

%Topologie dans R {{{1
\section{Topologie dans $\overline{\mathbb{R}}$}
\begin{defi}
	$\overline{\mathbb{R}} = \mathbb{R} \cup \{ +\infty, -\infty\}$
\end{defi}

\begin{prop} $ $
	\begin{itemize}
		\item [-] Intervalles de $\overline{\mathbb{R}}$ \\
			Ce sont les parties de $\overline{\mathbb{R}}$ du type ($a,b \in \overline{\mathbb{R}}$ et $a \leq b$): $[a,b]$ ou $[a,b[$ ou $]a,b[$
		\item [-] Voisinages en $\overline{\mathbb{R}}$ \\
			Soit $x \in \overline{\mathbb{R}}, A \subset \overline{\mathbb{R}}$. On dit que $A$ est voisinage de $x$ dans
			$\overline{\mathbb{R}}$ si:
				\begin{itemize}
					\item[-] si $x \in \mathbb{R},\ \exists \epsilon > 0$ tel que $[x - \epsilon, x + \epsilon] \subset A$
					\item[-] si $x = +\infty,\ \exists M \in \mathbb{R}$ tel que $[M, +\infty] \subset A$
					\item[-] si $x = - \infty,\ \exists M \in \mathbb{R}$ tel que $[-\infty, M] \subset A$
				\end{itemize}
		\item [-]  $\overline{\mathbb{R}}$ est s\'epar\'ee \\
			Pour $x, y \in \overline{\mathbb{R}}$ avec $x \neq y$, il existe $U \in V_{\overline{\mathbb{R}}%
			}(x), V \in V_{\overline{\mathbb{R}}%
			}$ tel que: $U \cap V \neq \emptyset$
		\item [-] Limite d'une suite \\
			Soit $(u_n)$ une suite r\'eelle et $l \in \overline{\mathbb{R}}$. On a: \\
			$u_n \rightarrow l \Leftrightarrow \forall\ V \in V_{\overline{\mathbb{R}}%
			}(l),\ \exists n_0 \in \mathbb{N}, \forall n \geq n_0, u_n \in V$
	\end{itemize}
\end{prop}

\begin{defi} Point adh\'erent \\
		Soit $A \subset \overline{\mathbb{R}}, x \in \overline{\mathbb{R}}$. On dit que $x$ est adh\'erent \`a $A$ dans
		$\overline{\mathbb{R}}$ si: \\
		$\forall\ V \in V_{\overline{\mathbb{R}}%
		}, A \cap V \neq \emptyset$
\end{defi}

\begin{rqs}
	Soit $D \subset \mathbb{R}, x \in \overline{\mathbb{R}}$
	\begin{enumerate}
		\item si $x \in \mathbb{R}$ alors: \\
			$x$ adh\'erent \`a $D$ dans $\overline{\mathbb{R}} \Leftrightarrow x$ adh\'erent \`a $D$ dans $\mathbb{R}$
		\item si $x = +\infty,$ \\
			$x$ adh\'erent \`a $D$ dans $\overline{\mathbb{R}} \Leftrightarrow D$ non major\'ee dans $\mathbb{R}$
		\item si $x = -\infty,$ \\
			$x$ adh\'erent \`a $D$ dans $\overline{\mathbb{R}} \Leftrightarrow D$ non minor\'ee dans $\mathbb{R}$
	\end{enumerate}
\end{rqs}
\begin{prop} Caract\'erisation s\'equentielle de l'adh\'erence \\
	Soit $A \subset \mathbb{R}, x \in \overline{\mathbb{R}}$,
	\[x \in Adh_{\overline{\mathbb{R}}%
	}(A) \Leftrightarrow \text{ il existe une suite } (a_n) \text{ de points de } A \text{ telle que } a_n \rightarrow x\]
\end{prop}

%Notion de limite {{{1
\section{Notion de limite}
\begin{defi}
	Soit $D \subset \mathbb{R}, a \in \overline{\mathbb{R}}$ adh\'erent \`a $D, l \in \overline{\mathbb{R}}, f: D \rightarrow \mathbb{R}$.
	\begin{enumerate}
		\item $\forall\ V \in V_{\overline{\mathbb{R}}%
		}(l), \exists U \in V_{\overline{\mathbb{R}}%
		}(a), f(U \cap D) \subset V$ [topologique]
		\item Pour toute $(x_n)$ de points de $D$ telle que $x_n \rightarrow a$ on a $f(x_n) \rightarrow l $ [s\'equentielle]
	\end{enumerate}
\end{defi}

\begin{prop} Lien entre limite et continuit\'e: \\
	$D \subset \mathbb{R}, x_0 \in D, f:D \rightarrow \mathbb{R}$
	\begin{enumerate}
		\item La seule limite possible de $f$ en $x_0$ est $f(x_0)$
		\item $f$ admet une limite en $x_0 \Leftrightarrow f(x) \rightarrow f(x_0) \Leftrightarrow f$ est continue en $x_0$
	\end{enumerate}
\end{prop}

\begin{prop}
	Supposons $lim_{x \to a} f(x) \rightarrow l$ avec $l \neq -\infty$, alors si $m \in \mathbb{R}, m < l$, on a $f(x) \geq m$
	sur voisinage de $a$
\end{prop}

\begin{prop} Prolongement par continuit\'e
	\begin{itemize}
		\item [-] $D \subset \mathbb{R}, x_0 \in D, f:D \backslash \{x_0\} \rightarrow \mathbb{R}$ continue
		\item [-] On suppose $x_0$ adh\'erent \`a $D \backslash \{x_0\}$, on suppose que $f$ admet une limite finie $l \in \mathbb{R}$
			en $x_0$.
		\item [-] Si on pose $\tilde{f}$ telle que
			\begin{align*}
				\tilde{f}: D &\rightarrow \mathbb{R} \\
				x = x_0 &\mapsto l \\
				x \neq x_0 &\mapsto f(x)
			\end{align*}
		Alors $\tilde{f}$ est continue sur $D$. On dit qu'on a prolong\'e f par continuit\'e en $x_0$
	\end{itemize}
\end{prop}

\begin{prop} Limites \`a gauche et \`a droite
	\begin{itemize}
		\item [-] $D \subset \mathbb{R}, x_0 \in \mathbb{R}, f:D \rightarrow \mathbb{R}$ \\
			$D^+ = D \cap ]x_0, +\infty[$ \\
			$D^- = D \cap ]-\infty, x_0[$
		\item [-] On suppose que $x_0$ est adh\'erent \`a $D^+$ et $D^-$. On note: \\
			$f^+ = f_{|D^+}$ \\
			$f^- = f_{|D^-}$
		\item [-] Si $f^+$ admet une limite en $x_0$, on dit que f admet une limite \`a droite en $x_0$. De m\^eme pour la gauche
		\begin{enumerate}
			\item Supposons que f admet en $x_0$ une limite $l \in \overline{\mathbb{R}}$. $f$ admet alors $l$ comme limite \`a droite
				et \`a gauche de $x_0$.
				\begin{rqs}
					Si $f$ n'a pas de limite \`a droite (resp. gauche) en $x_0, f$ n'a pas de limite en $x_0$. \\
					Si $f$ admet en $x_0$ des limites \`a droite et \`a gauche diff\'erentes, $f$ n'a pas de limite en $x_0$
				\end{rqs}
			\item Supposons que $f$ admet en $x_0$ des limites \`a droite et \`a gauches \'egales \`a $l$. \\
				Si $x_0 \notin D, lim_{x \to x_0} f(x) \rightarrow l$ \\
				Si $x_0 \in D$, la seule limite possible pour $f$ en $x_0$ est $f(x_0)$ donc si $l \neq f(x_0), f$ n'admet pas de limite
				en $x_0$
		\end{enumerate}
	\end{itemize}
\end{prop}

%Théorèmes généraux {{{1
\section{Th\'eor\`emes g\'en\'eraux}
Th\'eor\`emes g\'en\'eraux classiques valables pour les limites.
\begin{prop} $ $
	\begin{enumerate}
		\item Conservation des in\'egalit\'es larges par passage \`a la limite
		\item Comparaison
		\item Encadrement
	\end{enumerate}
\end{prop}

%Monotonie et limites {{{1
\section{Monotonie et limites}
\begin{theo} $ $ \\
	Soient $a \in \mathbb{R},\ b \in \overline{\mathbb{R}},\ f:[a,b[ \rightarrow \mathbb{R}$ croissante alors:
	\begin{enumerate}
		\item si $f$ est major\'ee, $f$ admet en $b$ une limite finie: $f(x) \underset{b}{\rightarrow} l = \sup f$
		\item si $f$ n'est pas major\'ee alors $f(x) \underset{b}{\rightarrow} +\infty$
	\end{enumerate}
	\begin{rqs}
		Propri\'et\'e avec ses correspondants corollaires
	\end{rqs}
\end{theo}

\begin{theo} $ $ \\
	Soit I un intervalle non trivial de $\mathbb{R},\ f:I \rightarrow \mathbb{R}$, croissante (par exemple). \\
	Soit $x_0 \in Int(I)$ alors $f$ admet en $x_0$ des limites telles que:
	\[\lim_{x_0^- f} \leq f(x_0) \leq \lim_{x_0^+}\]
\end{theo}

\begin{theo} $ $ \\
	Soit I un intervalle non trivial de $\mathbb{R},\ f:I\rightarrow \mathbb{R}$ monotone. Alors:
	\[f \text{ est continue} \Leftrightarrow f(I) \text{ est un intervalle}\]
\end{theo}

\begin{theo} Bijection \\
	I intervalle non trivial de $\mathbb{R},\ f:I\rightarrow \mathbb{R}$ continue, strictement monotone,$J = f(I)$
	\begin{enumerate}
		\item Alors J est un intervalle non trivial de $\mathbb{R}$
		\item $\tilde{f}$ d\'efinie par:
			\begin{align*}
				\tilde{f}: I &\rightarrow J \\
				t &\mapsto f(t)
			\end{align*}
			est bijective. $\tilde{f}^{-1}$ est strictement monotone de m\^eme monotonie que $f$ et continue
	\end{enumerate}
	\begin{rqs}
		On peut lorsque la fonction est strictement monotone et continue pr\'evoir la forme de l'intervalle des images
	\end{rqs}
\end{theo}

\end{document}
