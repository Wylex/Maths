%Préamble {{{1
\documentclass[fleqn]{article}

\usepackage{amssymb}
\usepackage{amsmath}
\usepackage{amsthm}
\usepackage{verbatim}
\usepackage{booktabs}
\usepackage{mathrsfs}

\theoremstyle{definition} \newtheorem*{defi}{D\'efinition}
\theoremstyle{definition} \newtheorem*{theo}{Th\'eor\`eme}
\theoremstyle{definition} \newtheorem*{coro}{Corollaire}
\theoremstyle{definition} \newtheorem*{nota}{Notation}
\theoremstyle{definition} \newtheorem*{vocab}{Vocabulaire}
\theoremstyle{remark} \newtheorem*{rqs}{Remarques}
\theoremstyle{definition} \newtheorem*{prop}{Propri\'et\'e}
\newcommand{\ra}[1]{\renewcommand{\arraystretch}{#1}}
\newcommand*{\bfrac}[2]{\genfrac{}{}{0pt}{}{#1}{#2}}
\ra{1.3}

\title{D\'eterminants}
\date{}

\begin{document}
\maketitle

%Construction du déterminant {{{1
\section{Construction du d\'eterminant}
\begin{defi}
	Soit $n \in \mathbb{N}^*,\ E$ et $F$ deux Kev, $f: E^n \rightarrow F$
	\begin{enumerate}
		\item On dit que $f$ est n-lin\'eaire si $f$ est lin\'eaire par rapport \`a chacune de ses variables:
		$\forall (u_1, \hdots, u_n) \in E^n,\ \forall j \in [\![1,n]\!]$, l'application $\varphi_j: x \in E \mapsto f(u_1, \hdots, u_{j-1},
		x, u_{j+1}, \hdots, u_n)$ est lin\'eaire
		\item On dit que $f$ est altern\'ee si pour tout $(u_1, \hdots, u_n) \in E^n$ ayant deux \'el\'ements \'egaux, $f(u_1, \hdots, u_n) = 0$
	\end{enumerate}
\end{defi}

\begin{prop} $ $
	\begin{itemize}
		\item [-] $\mathscr{L}^n(E,F)$ et $\mathscr{A}^n(E,F)$ sont des sev de $\mathscr{F}(E^n,F)$
		\item [-] Soit $f \in \mathscr{L}^n(E,F)$, soit $(x_1, \hdots, x_n) \in E^n$\\Si l'un des $x_i$ est nul,
			\mbox{$f(x_1, \hdots, x_n) = 0_F$}
		\item [-] Soit $f \in \mathscr{L}^n(E,F),\ (e_1, \hdots, e_p) \in E^n$ et $x_1, \hdots, x_n \in Vect(e_1, \hdots, e_p)$\\On \'ecrit
			$x_j = \sum_{i=1}^p \alpha_{ij} e_i$ alors:
			\begin{align*}
				f(x_1, \hdots, x_n) &= \sum_{i_1 = 1}^p \hdots \sum_{i_n = 1}^p \alpha_{i_1 1} \hdots, \alpha_{i_n n}
					f(e_{i_1}, \hdots, e_{i_n})\\
									&= \sum_{\phi \in \mathscr{F}([\![1,n]\!], [\![1,p]\!])} \alpha_{\phi(1) 1} \hdots \alpha_{\phi(n) n}
					f(e_{\phi(1)}, \hdots, e_{\phi(n)})
			\end{align*}
		\item [-] Soit $f \in \mathscr{A}^n(E,F),\ x_1, \hdots, x_n \in E$. Soit $j \in [\![1,n]\!], y \in Vect(x_i)_{i \in [\![1,n]\!]
			\backslash \{j\}}$ alors $f(x_1, \hdots, x_j + y, \hdots, x_n) = f(x_1, \hdots, x_j, \hdots, x_n)$
		\item [-] $f \in \mathscr{A}^n(E,F),\ (x_1, \hdots, x_n) \in E^n$: $(x_1, \hdots, x_n)$ li\'ee $\Rightarrow f(x_1, \hdots, x_n) = 0$
		\item [-] $f \in \mathscr{A}^n(E,F)$ alors $f$ est antysim\'etrique: \\
			$\forall (x_1, \hdots, x_n) \in E^n,\ \forall\ 1 \leq i < j \leq n,\\
			f(x_1, \hdots, x_i, \hdots, x_j, \hdots, x_n) = -f(x_1, \hdots, x_j, \hdots, x_i, \hdots, x_n)$
		\item [-] $f \in \mathscr{A}^n(E,F)$ alors $\forall \sigma \in S_n:\\ \forall (x_1, \hdots, x_n) \in E^n,\ f(x_{\sigma(1)}, \hdots,
			x_{\sigma(n)}) = \epsilon(\sigma)f(x_1, \hdots, x_n)$
	\end{itemize}
\end{prop}

\begin{theo} D\'eterminant\\
	Soit $E$ un Kev de dimension fini $n \in \mathbb{N}^*$, soit $B = (e_1, \hdots, e_n)$ une base de $E$
	\begin{enumerate}
		\item Il existe une unique forme n-lin\'eaire altern\'ee $\varphi$ sur $E^n$ telle que \mbox{$\varphi(e_1, \hdots, e_n) = 1$}. $\varphi$
			s'appelle le d\'eterminant dans la base $B$ et se note $\det_B$
		\item Pour $(x_1, \hdots, x_n) \in E^n$ on a $\det_B(x_1, \hdots, x_n) = \sum_{\sigma \in S_n} \epsilon(\sigma) \prod_{k=1}^n
			a_{\sigma(k) k}$\\ o\`u $(a_{ij}) = Mat_B(x_1, \hdots, x_n)$
		\item $\mathscr{A}^n(E,F) = Vect(\det_B)$
		\item $(x_1, \hdots, x_n)$ est une base de $E \Leftrightarrow \det_B (x_1, \hdots, x_n) \neq 0$
	\end{enumerate}
\end{theo}

\end{document}
