%Préamble {{{1
\documentclass[fleqn]{article}

\usepackage{amssymb}
\usepackage{amsmath}
\usepackage{amsthm}
\usepackage{stmaryrd}
\usepackage{verbatim}
\usepackage{booktabs}
\usepackage{mathrsfs}

\theoremstyle{definition} \newtheorem*{defi}{D\'efinition}
\theoremstyle{definition} \newtheorem*{theo}{Th\'eor\`eme}
\theoremstyle{definition} \newtheorem*{coro}{Corollaire}
\theoremstyle{remark} \newtheorem*{rqs}{Remarques}
\theoremstyle{definition} \newtheorem*{prop}{Propri\'et\'e}
\newcommand{\ra}[1]{\renewcommand{\arraystretch}{#1}}
\ra{1.3}

\title{Probabilit\'es}
\date{}

\begin{document}
\maketitle

%Axiomes des probabilités {{{1
\section{Axiomes des probabilit\'es}
Soit $\Omega$ un ensemble fini.
\begin{defi}
	Une probabilit\'e sur $\Omega$ est une application $P: \mathcal{P}(\Omega) \rightarrow [0,1]$ tq:
	\begin{enumerate}
		\item $P(\Omega) = 1$
		\item $\forall A,B \in \mathcal{P}(\Omega),\ A \cap B = \emptyset \Rightarrow P(A \cup B) = P(A) + P(B)$
	\end{enumerate}
On dit ainsi que $(\Omega,P)$ est un espace probabilis\'e. Les \'el\'ements de $\mathcal{P}(\Omega)$ s'appellent les \'ev\'enements.
	\begin{enumerate}
		\item Pour $\omega \in \Omega,\ \{\omega\}$ est un \'ev\'enement \'el\'ementaire
		\item $\emptyset$ est l'\'ev\'enement impossible, $\Omega$ est l'\'ev\'enement certain
		\item $A \in \mathcal{P}(\Omega),\ \bar{A}$ est l'\'ev\'enement contraire
		\item $A$ est quasi-certain si $P(A) = 1$, quasi-impossible si $P(A) = 0$
		\item Deux \'ev\'enements sont incompatibles si $A \cap B = \emptyset$
	\end{enumerate}
\end{defi}

\begin{prop} Soit $(\Omega, P)$ un espace probabilis\'e
	\begin{itemize}
		\item [-] $P(\emptyset) = 0$
		\item [-] Si $A_1, \hdots, A_n$ sont incompatibles deux \`a deux, $P(\bigcup A_i) = \sum P(A_i)$
		\item [-] $P(\bar{A}) = 1 - P(A)$
		\item [-] Si $\{B_1, \hdots, B_n\}$ est un syst\`eme complet d'\'ev\'enements (si $\bigcup B_i = \Omega$ et $B_i \cap B_j = \emptyset$)
			alors pour tout $A \in \mathcal{P}(\Omega)$, $P(A) = \sum P(A \cap B_i)$
		\item [-] $A,B \in \mathcal{P}(\Omega),\ A \subset B \Rightarrow P(A) \leq P(B)$
		\item [-] Soient $A_1, \hdots, A_n$ des ev on a $P(\bigcup A_i) \leq \sum P(A_i)$
		\item [-] Identit\'e de Poincar\'e:\\
			Soient $A_1, \hdots, A_n \in \mathcal{P}(\Omega)$ alors $P(\bigcup A_i) = \sum_{k=1}^n (-1)^{k-1} \sum_{I \in \mathcal{P}_k(n)}
			P(\bigcap_{j \in I} A_j)$
	\end{itemize}
\end{prop}

\begin{theo}
	Soit $\Omega$ un ensemble fini. Se donner une probabilit\'e sur $\Omega$ revient \`a se donner une famille $(p_\omega)_{\omega \in \Omega}$
	de r\'eels positifs de somme 1
\end{theo}

%Probabilités conditionnelles - indépendance {{{1
\section{Probabilit\'es conditionnelles - Ind\'ependance}
\begin{defi} Soit $(\Omega, P)$ un espace probabilis\'e
	\begin{enumerate}
		\item Pour $A,B \in \mathcal{P}(\Omega)$ on dit que $A$ et $B$ sont ind\'ependants lorsque \\$P(A \cap B) = P(A)P(B)$\\
			Plus g\'en\'eralement, $A_1, \hdots, A_n \in \mathcal{P}(\Omega)$ sont ind\'ependants si: \[\forall I \subset \{1,
			\hdots, n\},\ |I| \geq 2 \Rightarrow P(\bigcap_{i \in I} A_i) = \prod_{i \in I} P(A_i)\]
		\item Soit $B \in \mathcal{P}(\Omega),\ P(B) > 0$, pour $A \in \mathcal{P}(\Omega)$, la probabilit\'e conditionnelle de $A$ sachant $B$
			est: \[P_B(A) = P(A|B) = \frac{P(A\cap B)}{P(B)}\]
	\end{enumerate}
\end{defi}

\begin{prop} $ $
	\begin{itemize}
		\item [-] Si $A_1, \hdots, A_n$ sont ind\'ependants alors $\forall I \subset [\![1,n]\!],\ (A_i)_{i \in I}$ sont ind\'ependants
		\item [-] Soient $A,B$ ind\'ependants alors $(\bar{A},B), (A, \bar{B}),(\bar{A}, \bar{B})$ sont ind\'ependants
		\item [-] Soient $A_1, \hdots, A_n$ indep, $B_i \in \{A_i, \bar{A_i}\}$ alors $B_1, \hdots, B_n$ sont indep
	\end{itemize}
\end{prop}

\begin{theo} $ $
	\begin{enumerate}
		\item Formule des probabilit\'es totales\\
			$(\Omega, P)$ un syst\`eme, $\{E_1, \hdots, E_n\}$ un syst\`eme complet tel que $P(E_i) > 0$ alors $\forall A \in
			\mathcal{P}(\Omega):$ \[P(A) = \sum P_{E_i}(A)P(E_i)\]
		\item Formule de Bayes\\
			Soit $\{E_1, \hdots, E_n\}$ un SCE tel que $P(E_i) > 0$, $A \in \mathcal{P}(\Omega)$ tel que $P(A) > 0$ alors pour $(1 \leq i
			\leq n)$, \[P_A(E_i) = \frac{P_{E_i}(A) P(E_i)}{\sum P_{E_j}(A) P(E_j)}\]
		\item Formule des probabilit\'es compos\'ees\\
			$A_1, \hdots, A_n$ des ev. tels que $P(A_1 \cap \hdots \cap A_{n-1}) \neq 0$, alors:
			\[P(A_1 \cap \hdots, A_n) = P(A_1) \times P_{A_1}(A_2) \times P_{A_1 \cap A_2}(A_3) \times \hdots \times
			P_{A_1 \cap \hdots \cap A_{n-1}}(A_n)\]
	\end{enumerate}
\end{theo}

%Succésion d'expériences {{{1
\section{Succ\'esion d'exp\'eriences al\'eatoires}
Soient $(\Omega_1, P_1),\ (\Omega_2, P_2)$ deux exp\'eriences al\'eatoires\\ On pose pour $\omega = (\omega_1, \omega_2) \in \Omega_1 \times
\Omega_2$: \[p_\omega = P_1(\{\omega_1\})P_2(\{\omega_2\})\]

\begin{enumerate}
	\item $\forall \omega \in \Omega_1 \times \Omega_2$, on a $p_\omega \geq 0$
	\item $\sum_{(\omega_1, \omega_2) \in \Omega_1 \times \Omega_2} p_\omega = 1$
\end{enumerate}
La famille $(p_\omega)_{(\omega_1, \omega_2) \in \Omega_1 \times \Omega_2}$ d\'efinie donc bien une probabilit\'e sur $\Omega_1 \times \Omega_2$\\
On a alors pour $A \subset \Omega_1,\ B \subset \Omega_2,\ P(A \times B) = \sum_{\omega_1 \in A} P_1(\{\omega_1\}) \sum_{\omega_2 \in B}
P_2(\{\omega_2\})$

%Variables aléatoires {{{1
\section{Variables al\'eatoires}
\subsection{G\'en\'eralit\'es}
\begin{defi} Variable al\'eatoire
	\begin{itemize}
		\item [-] Soit $(\Omega, P)$ un espace probabilis\'e fini. Une variable al\'eatoire sur $\Omega$ est une application $X$ d\'efinie sur
			$\Omega$
		\item [-] Soit $(\Omega, P)$ un espace probabilis\'e, $X: \Omega \rightarrow E$ une va\\
			Si $B \subset E,\ X^{-1}(B) = \{\omega \in \Omega ,\ X(\omega) \in B\}$ se note $\{X \in B\}$\\
			On pose $Q(B) = P(\{X \in B\})$. Ceci d\'efinit une probabilit\'e sur $(E, \mathcal{P}(E))$. $Q$ s'appelle la loi de $X$.
	\end{itemize}
\end{defi}

\begin{defi} Lois
	\begin{enumerate}
		\item Loi de Bernouilli\\
			Une va $Y$ d\'efinie sur un espace probabilis\'e $(\Omega, P)$ suit une loi de Bernouilli de param\'etre $p \in [0,1]$ lorsque
			$\{0,1\} \subset Y(\Omega)$ (en g\'en\'eral $=$) et $P(Y = 0) = 1-p,\ P(Y = 1) = p$ \\
			On \'ecrit $Y \hookrightarrow B(p)$
		\item Loi binomiale \\
			Une va $Y$ suit la loi binomiale de param\`etre $n$ et $p$ lorsque $Y(\Omega) = [\![0,n]\!]$ et $\forall k \in [\![0,n]\!],\
			P(X=k) = \binom{n}{k} p^k (1-p)^{n-k}$\\
			On \'ecrit $Y \hookrightarrow B(n,p)$
		\item Loi uniforme \\
			$X: \Omega \rightarrow E$ suit une loi uniforme sur $E$ si $\forall x \in E,\ P(X = x) = \frac{1}{|E|}$
	\end{enumerate}
\end{defi}

\begin{prop} D\'eterminer la loi de $X$ revient \`a:
	\begin{enumerate}
		\item D\'eterminer $X(\Omega)$
		\item Pour chaque $x \in X(\Omega)$ trouver $P(X=x)$
	\end{enumerate}
\end{prop}

\subsection{Variables al\'eatoires ind\'ependantes}
\begin{defi} Soit $(\Omega, P)$ un espace probabilis\'e
	\begin{enumerate}
		\item $X: \Omega \rightarrow E,\ Y: \Omega \rightarrow F$ des va. On dit que $X$ et $Y$ sont ind\'ependantes si:\\ $\forall A \subset E,\
		B \subset F$ les \'ev $\{X \in A\}$ et $\{Y \in B\}$ sont ind\'ependants \\(ie $P(\{X\in A\} \cap \{Y \in B\}) = P(\{X\in A\})
		P(\{Y \in B\})$
		\item On peut g\'en\'eraliser pour $n$ va (il faut que les ev $(\{X_i \in A_i\})_{1\leq i \leq n}$ soient indep)
	\end{enumerate}
	\begin{rqs}
			On \'ecrit $P(X \in A,\ X \in B)$ au lieu de $P(\{X \in A\} \cap \{Y \in B\})$
	\end{rqs}
\end{defi}

\begin{theo}
	Soient $E_1, \hdots, E_n$ des ensembles, $X_i: \Omega \rightarrow E_i$ alors:\\
	$(X_1, \hdots, X_n)$ sont indep $\Leftrightarrow \forall (A_1, \hdots, A_n) \in \mathcal{P}(E_1) \times \hdots \times \mathcal{P}(E_n),\\
	P(X_1 \in A_1, \hdots, X_n \in A_n) = \prod_1^n P(X_i \in A_i)$
\end{theo}

\begin{prop} $ $
	\begin{enumerate}
		\item $X$ et $Y$ indep $\Leftrightarrow \forall x,y \in X(\Omega),Y(\Omega),\\ P(X=x, Y=y) = P(X=x)\times P(Y=y)$
		\item $X_1, \hdots, X_n$ indep $\Leftrightarrow \forall (x_1, \hdots, x_n) \in X_1(\Omega), \hdots, X_n(\Omega),\\ P(X_1=x_1, \hdots,
			X_n = x_n) = \prod P(X_i = x_i)$
		\item Soient $X_i: \Omega \rightarrow E_i$ indep, soit $k \in [\![0,n-1]\!],\\ Z=(X_1, \hdots, X_k): \omega \in \Omega \mapsto
			(X_1(\omega), \hdots, X_k(\omega))$\\
			$T= (X_{k+1}, \hdots, X_n): \omega \in \Omega \mapsto (X_{k+1}(\omega), \hdots, X_n(\omega))$\\
			Alors $Z$ et $T$ sont ind\'ependantes
		\item $X_i: \Omega \rightarrow E_i$ et $f_i : E_i \rightarrow F_i$. On suppose $(X_i)$ indep $Y_i = f_i(X_i)$ alors $(Y_i)$ sont indep
		\item $X_i: \Omega \rightarrow E_i$ indep. \\Soit $k \in \llbracket 1,n-1 \rrbracket\  f: E_1 \times \hdots \times E_k \rightarrow F $ et
			$g: E_{k+1} \times \hdots, \times E_n \rightarrow G$ \\
			Alors, $f(X_1, \hdots, X_k)$ et $g(X_{k+1}, \hdots, X_n)$ sont indep
	\end{enumerate}
\end{prop}

\subsection{Esp\'erance et Variance}
\subsubsection{Esp\'erance}

\begin{defi}
	Soit $X$ une va r\'eelle (\`a valeurs dans $\mathbb{R}$) sur l'espace probabilis\'e fini $(\Omega, P)$. On appelle esp\'erance de $X$
	le r\'eel $E(X) = \sum_{x \in X(\Omega)} xP(X=x)$
\end{defi}

\begin{prop} Esp\'erance
	\begin{itemize}
		\item [-] $E(X) = \sum_{\omega \in \Omega} X(\omega) P(\{\omega\})$
		\item [-] Si $X$ est une va certaine ($\exists a \in \mathbb{R}, P(X=a) = 1$) alors $E(X) = a$
		\item [-] Si $X$ est positive alors $E(X) \geq 0$
		\item [-] $E(\alpha X + Y) = \alpha E(X) + E(Y)$
		\item [-] $|E(X)| \leq E(|X|)$
		\item [-] Formule de transfert: $\Omega \overset{X}{\rightarrow} E \overset{f}{\rightarrow}\mathbb{R}$ \\
			$E(f(X)) = \sum_{x \in X(\Omega)} f(x) P(X=x)$
		\item [-] In\'egalit\'e de Markov:\\
			$X:\Omega \rightarrow \mathbb{R}_+$ une va positive alors: $\forall \epsilon > 0, P(\{X \geq \epsilon\}) \leq \frac{E(X)}{\epsilon}$
	\end{itemize}
\end{prop}

\subsubsection{Variance}
\begin{defi}
	On pose $V(X) = E((X - E(X))^2)$ la variance et $\sigma(X) = \sqrt{V(X)}$ l'\'ecart type de $X$
\end{defi}

\begin{prop} $ $
	\begin{enumerate}
		\item [-] $V(X) = E(X^2) - (E(X))^2$ (On d\'eduit que la variance ne d\'epend que de la loi de $X$)
		\item [-] $E(|X-E(X)|) \leq \sqrt{V(X)}$
		\item [-] $V(X+Y) = V(X) + V(Y) + 2E((X-E(X))(Y-E(Y)))$ \\
			$E((X-E(X))(Y-E(Y)))$ s'appelle la covariance de $X$ et $Y$
		\item [-] $(X,Y) \mapsto cov(X,Y)$ est une forme bilin\'eaire, sym\'etrique et positive
		\item [-] $V(X) = cov(X,X)$
	\end{enumerate}
\end{prop}

\begin{theo}
	Soient $X,Y$ deux va indep r\'eelles alors $E(XY) = E(X)E(Y)$ (dans ce cas $cov(X,Y) = 0$) \\
	Plus g\'en\'eralement, si $X_1, \hdots, X_n$ sont indep, $E(X_1 \hdots X_n) = E(X_1) \hdots E(X_n)$
\end{theo}

\begin{prop}
	Si $X_1, \hdots, X_n$ sont des va indep alors $V(\sum X_i) = \sum V(X_i)$
\end{prop}

\begin{prop}
	$X:\Omega \rightarrow \mathbb{R},\ \alpha, \beta \in \mathbb{R}$, alors $V(\alpha X + \beta) = \alpha^2 V(X)$
\end{prop}

\begin{defi} Soit $X$ une va r\'eelle:
	\begin{enumerate}
		\item $X$ est dite centr\'ee si $E(X) = 0$
		\item $X$ est dite r\'ediute si $V(X) = 1$
	\end{enumerate}
\end{defi}

\begin{theo} In\'egalit\'e de Bienaym\'e - Tch\'ebychev\\
	Soit $X: \Omega \rightarrow \mathbb{R},\ \epsilon > 0$ \\
	Alors $P(|X-E(X)|\geq \epsilon) \leq \frac{V(X)}{\epsilon^2}$
\end{theo}

\end{document}
