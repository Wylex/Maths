%Préamble {{{1
\documentclass[fleqn]{article}

\usepackage{amssymb}
\usepackage{amsmath}
\usepackage{amsthm}
\usepackage{verbatim}
\usepackage{booktabs}
\usepackage{mathrsfs}

\theoremstyle{definition} \newtheorem*{defi}{D\'efinition}
\theoremstyle{definition} \newtheorem*{theo}{Th\'eor\`eme}
\theoremstyle{definition} \newtheorem*{coro}{Corollaire}
\theoremstyle{definition} \newtheorem*{nota}{Notation}
\theoremstyle{remark} \newtheorem*{rqs}{Remarques}
\theoremstyle{definition} \newtheorem*{prop}{Propri\'et\'e}
\newcommand{\ra}[1]{\renewcommand{\arraystretch}{#1}}
\newcommand*{\bfrac}[2]{\genfrac{}{}{0pt}{}{#1}{#2}}
\ra{1.3}

\title{Anneaux et corps}
\date{}

\begin{document}
\maketitle

%Définitions {{{1
\section{D\'efinitions}
\begin{defi}
	Un anneau est un triplet $(A, +, \times)$ o\`u $A$ est un ensemble non vide, $+$ et $\times$ des LCI sur $A$ telles que:
	\begin{enumerate}
		\item $(A, +)$ est un groupe commutatif
		\item $(A, \times)$ est un mono\"ide
		\item $\times$ est distributive par rapport \`a $+$
	\end{enumerate}
L'anneau est dit commutatif lorsque $\times$ est commutatif
\end{defi}

\begin{nota} $ $
	\begin{itemize}
		\item [-] Pour $x \in A$ on notera $-x$ l'oppos\'e de $x$.\\
			Le neutre de $+$ est "le z\'ero" de $A$, not\'e $0_A$
		\item [-] Le neutre de $\times$ est l'\'el\'ement unit\'e, not\'e $1_A$\\
			On note $\mathcal{U}(A)$ le groupe des inversibles de $(A, \times)$. On note l'inverse $x^{-1}$
	\end{itemize}
\end{nota}

%Règles de calcul {{{1
\section{R\`egles de calcul}
\subsection{Concernant le $+$}
Soit $(A, +, \times)$ un anneau
\begin{enumerate}
	\item $\forall x \in A,\ 0_A \times x = x \times 0_A = 0_A$
	\item $\forall x \in A, \forall n \in \mathbb{Z}: nx = (n1_A) \times x = x \times (n1_A)$
	\item $\forall x,y \in A, \forall n \in \mathbb{Z}:\ x \times (ny) = (nx) \times y = n(x \times y)$
	\item $(-x)(-y) = x \times y$
\end{enumerate}

\subsection{Concernant le $\times$}
Soient $x, y \in A$ tels que $x \times y = y \times x$
\begin{enumerate}
	\item $\forall n \in \mathbb{N},\ (x + y)^n = \sum_{k=0}^n \binom{k}{n}x^k \times y^{n-k}$
	\item $x^n - y^n = (x-y)\sum_{k=0}^{n-1}x^ky^{n-1-k}$
\end{enumerate}

%Anneaux intégres-corps {{{1
\section{Anneaux int\'egres-corps}
\begin{defi}
	Un anneau $(A, + , \times)$ est int\'egre si:
	\[\forall x,y \in A, x\times y = 0_A \Rightarrow (x = 0_A \lor y = 0_A)\]
	\begin{rqs} Dans programme officiel, $(A, \times)$ doit \^etre de plus commutatif \end{rqs}
\end{defi}

\begin{prop} $ $
	Si $(A,+,\times)$ est int\'egre et si $a \in A\backslash \{0\}$:
		\begin{align*}
			\forall x,y \in A,\ & a \times x = a \times y \Rightarrow x = y \\
								& x \times a = y \times a \Rightarrow x = y
		\end{align*}
\end{prop}

\begin{defi} Corps \\
	Un corps est un anneau $(A,+,\times)$ commutatif (non nul) dans lequel tout \'el\'elent non nul est inversible
	($\mathcal{U}(A) = A\backslash\{0_A\}$)
\end{defi}

\begin{prop}
	Un corps est un anneau int\`egre
\end{prop}

%Sous-anneaux, morphismes d'anneaux {{{1
\section{Sous-anneaux, morphismes d'anneaux}
\begin{defi} Sous-anneau \\
	$(A,+,\times)$ un anneau, $B \subset A$. $B$ est un sous-anneau de $A$ si:
	\begin{enumerate}
		\item $B$ est un sous-groupe de $(A,+)$
		\item $B$ est stable par produit
		\item $1_A \in B$
	\end{enumerate}

	\begin{rqs}
		Si $A$ est un anneau int\'egre, tout sous-anneau de $A$ aussi
	\end{rqs}
\end{defi}

\begin{defi} Sous-corps \\
	Soit $(K,+,\times)$ un corps. On dira que $B \subset K$ est un sous corps de $K$ si:
	\begin{enumerate}
		\item $B$ est un sous anneau de K
		\item $\forall x \in B\backslash\{0\},\ x^{-1} \in B$
	\end{enumerate}
\end{defi}

\begin{defi} Morphismes d'anneaux \\
	$A, B$ deux anneaux, $f:A \rightarrow B$ est un morsphisme d'anneaux si:
	\begin{enumerate}
		\item $\forall x,y \in A, f(x+y) = f(x) + f(y)$
		\item $\forall x,y \in A, f(xy) = f(x)f(y)$
		\item $f(1_A) = 1_B$
	\end{enumerate}
\end{defi}

\end{document}
