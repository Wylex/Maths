\documentclass[fleqn]{article}
\usepackage{amssymb}
\usepackage{amsmath}
\usepackage{amsthm}
\usepackage{verbatim}
\usepackage{booktabs}

\title{\'Equivalents}
\date{}

\theoremstyle{definition} \newtheorem*{defi}{D\'efinition}
\theoremstyle{definition} \newtheorem*{theo}{Th\'eor\`eme}
\theoremstyle{remark} \newtheorem*{rqs}{Remarques}
\newcommand{\ra}[1]{\renewcommand{\arraystretch}{#1}}
\ra{1.3}

\begin{document}
\maketitle

\section{D\'efinitions}
Soit $D$ une partie de $\mathbb{R}, a \in \overline{\mathbb{R}}$ adh\'erent \`a $D$, $f,g: D \rightarrow \mathbb{R}$. Au voisinage de A:
\begin{enumerate}
	\item f n\'egligeable devant g: \\
		$\exists\ U \in V_{\overline{\mathbb{R}}}(a), \epsilon: U \cap D \rightarrow \mathbb{R}$ tel que
		$\epsilon(x)\underset{a}{\rightarrow}  0$ et $\forall t \in U \cap D, f(t)=\epsilon(t)g(t)$
	\item f domin\'ee par g: \\
		$\exists\ U \in V_{\overline{\mathbb{R}}}(a), \omega: U \cap D \rightarrow \mathbb{R}$ tel que $\omega$ born\'ee et
		$\forall t \in U \cap D, f(t)=\omega(t)g(t)$
	\item f \'equivalent \`a g: \\
		$\exists\ U \in V_{\overline{\mathbb{R}}}(a), \varphi: U \cap D \rightarrow \mathbb{R}$ tel que
		$\varphi(x)\underset{a}{\rightarrow} 1$ et $\forall t \in U \cap D, f(t)=\varphi(t)g(t)$
\end{enumerate}

\begin{rqs} $ $
	\begin{enumerate}
		\item Une fonction n'est \'equivalente, domin\'ee ou n\'egligeable par 0 que si elle est nulle sur le voisinage
		\item $f(x) = o(1)$ signifie que $f(x) \rightarrow 0$
		\item Soit $l \in \mathbb{R}^{*}, f(x) \sim l \Leftrightarrow f(x) \rightarrow l$
	\end{enumerate}
\end{rqs}

\begin{theo}
	Supposons que $f(x)\underset{a}{\sim} g(x)$\\
	Alors, f et g on m\^eme signe au voisinage de a et s'annulent simultan\'ement. De plus, si g admet une limite en a alors f aussi
\end{theo}

\section{\'Equivalents classiques}
\begin{tabular}{@{}ll@{}}
		$\sin x \underset{0}{\sim} x$ & $\arcsin x \underset{0}{\sim} x$ \\
		$\tan x \underset{0}{\sim} x$ & $\arctan x \underset{0}{\sim} x$ \\
		$1 - \cos x \underset{0}{\sim} \frac{x^2}{2}$ & $\ln(1+x) \underset{0}{\sim} x$ \\
		$\ln x \underset{1}{\sim} x-1$ & $e^x -1 \underset{0}{\sim} x$ \\
		$(a+x)^{\alpha} -1 \underset{0}{\sim} \alpha x$ & $\sqrt{1+x} - 1 \underset{0}{\sim} \frac{x}{2}$ \\
		$\cos x \underset{0}{\sim} 1$ & $\sqrt{1+x} \underset{0}{\sim} 1$ \\
\end{tabular}

\section{Faits de base}
\begin{itemize}
	\item $f_1(x) = o(g(x))$\\
		$f_2(x)	= o(g(x))$ \\
		Alors $\forall \alpha \in \mathbb{R}, \alpha f_1(x) + f_2(x) =  o(g(x))$
	\item $f(x) \sim g(x) \Leftrightarrow f(x) - g(x) = o(g(x))$
	\item $f(x) = o(g(x)) \Rightarrow f(x) + g(x) \sim g(x)$
	\item l'\'equivalence est r\'eflexive, sym\'etrique et transitive
	\item $f_1(x) \sim g_1(x)$\\
		$f_2(x) \sim g_2(x)$\\
		Alors, $f_1(x)f_2(x) \sim g_1(x)g_2(x)$ \\et si $f_2(x), g_2(x) \neq 0$ au voisinage de a, $\frac{f_1(x)}{f_2(x)} \sim
		\frac{g_1(x)}{g_2(x)}$
\end{itemize}

\begin{theo} Composition (\`a droite)\\
	$f,g: D \rightarrow \mathbb{R}, a \in Adh_{\overline{\mathbb{R}}}(D), \varphi: B \rightarrow D, b \in Adh_{\overline{\mathbb{R}}}(B)$ \\
	On suppose $\varphi(x)\underset{t\rightarrow b}{\rightarrow} a$ et $f(x)\underset{a}{\sim} g(x)$, alors
	\[f(\varphi(t))\underset{b}{\sim} g(\varphi(t))\]

\end{theo}

\begin{rqs}
	La composition \`a gauche marche si on \'el\'eve les fonction \`a une puissance $\alpha \in \mathbb{R}$ ou si on compose par $\ln$ avec une
	fonction g qui tend vers $0$ ou $+\infty$
\end{rqs}

\end{document}
