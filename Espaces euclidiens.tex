%Préamble {{{1
\documentclass[fleqn]{article}

\usepackage{amssymb}
\usepackage{amsmath}
\usepackage{amsthm}
\usepackage{verbatim}
\usepackage{booktabs}
\usepackage{mathrsfs}

\theoremstyle{definition} \newtheorem*{defi}{D\'efinition}
\theoremstyle{definition} \newtheorem*{theo}{Th\'eor\`eme}
\theoremstyle{definition} \newtheorem*{coro}{Corollaire}
\theoremstyle{remark} \newtheorem*{rqs}{Remarques}
\theoremstyle{definition} \newtheorem*{prop}{Propri\'et\'e}
\newcommand{\ra}[1]{\renewcommand{\arraystretch}{#1}}
\ra{1.3}

\title{Espaces euclidiens}
\date{}

\begin{document}
\maketitle

%Produit scalaire, norme euclidienne {{{1
\section{Produit scalaire, norme euclidienne}

\subsection{Produit scalaire}

\begin{defi} Soit $E$ un Rev, un produit scalaire sur $E$ est une application $\varphi : E \times E \rightarrow \mathbb{R}$ telle que:
	\begin{enumerate}
		\item $\varphi$ est bilin\'eaire
		\item $\varphi$ est sym\'etrique: $\varphi(x,y) = \varphi(y,x)$
		\item $\varphi$ est d\'efinie positive: $\forall x \in E \backslash \{0\},\ \varphi(x,x) > 0$
	\end{enumerate}
\end{defi}

\begin{rqs} Soit $\varphi$ bilin\'eaire, alors $\varphi$ est d\'efinie positive $\Leftrightarrow$ \begin{enumerate}
	\item $\forall x \in E,\ \varphi(x,x) \geq 0$
	\item $\varphi(x,x) = 0 \Rightarrow x = 0$
	\end{enumerate}
\end{rqs}


\subsection{Norme euclidienne}

\begin{defi} Soit $E$ un Rev. Une norme sur $E$ est une application $N:E \rightarrow \mathbb{R}_+$ telle que:
	\begin{enumerate}
		\item Pour $x \in E, N(x) = 0 \Leftrightarrow x = 0$
		\item Pour $x \in E, \alpha \in \mathbb{R}, N(\alpha x) = |\alpha| N(x)$
		\item Pour $x,y \in E,\ N(x+y) \leq N(x) + N(y)$
	\end{enumerate}
\end{defi}

\begin{theo}  Norme euclidienne \\
	Soit $E$ un Rev et $\langle\ \rangle$ un ps sur $E$, pour $x \in E$, posons \mbox{$\|x\| = \sqrt{\langle x,x \rangle}$}, alors $x \in E
	\mapsto \|x\|$ est une norme sur $E$ appel\'ee norme euclidienne associ\'e au produit scalaire $\langle\ \rangle$
\end{theo}

\begin{prop} Relations entre ps et $\| \ \|$
	\begin{enumerate}
		\item [-] In\'egalit\'e de Cauchy-Schwarz: $\forall x,y \in E,\ |\langle x,y \rangle| \leq \|x\| \|y\|$
		\item [-] $\|x+y\| = \|x\| + \|y\| \Leftrightarrow (x \lor y = 0) \lor (\exists \lambda \in \mathbb{R}_+,\ y = \lambda x)$
		\item [-] $\|\sum_{i=1}^n x_i \| ^2 = \sum_{i=1}^n \|x_i\|^2 + 2\sum_{1 \leq i < j \leq n} \langle x_i,x_j \rangle$
		\item [-] En particulier, (identit\'e du parall\'elogramme):\\
			$\|x+y\|^2 + \|x-y\|^2 = 2(\|x\|^2 + \|y\|^2)$
		\item [-] $\|x+y\|^2 - \|x-y\|^2= 4\langle x,y \rangle$
	\end{enumerate}
\end{prop}

%Familles orthogonales, processus d'orthonormalisation {{{1
\section{Familles orthogonales, processus d'orthonormalisation}

\begin{defi} Vecteurs orthogonaux \\
	Soit $\langle\ \rangle$ un ps sur le Rev $E$, $(x,y) \in E^2$. Ainsi $x$ est orthogonal \`a $y$ si $\langle x,y \rangle = 0$
	\begin{rqs} Si $x_i \perp x_j,\ i \neq j$ alors $\|\sum_{i=1}^n x_i \|^2 = \sum_{i=1}^n \|x_i\|^2$ \end{rqs}
\end{defi}

\begin{defi} Famille orthogonale \\
	Soit $E$ un Rev muni d'un ps $\langle\ \rangle$ et $(x_i)_{i\in I}$ une famille de vecteurs de $E$. On dit que la famille est orthogonale si
	$x_i \perp x_j$ pour $i\neq j$. Elle est orthonorm\'ee si de plus $\|x_i\| = 1$
\end{defi}

\begin{theo} Orthonormalisation de Gram-Schmidt \\
	Soit $E$ un Rev muni d'un ps $\langle\ \rangle$ et $(x_1, \hdots, x_n)$ une famille libre de vecteurs de $E$. Alors il existe une unique
	famille orthonormale $(e_1, \hdots, e_n)$ de vecteurs de $E$ telle que:
	\begin{enumerate}
		\item $Vect(e_1, \hdots, e_i) = Vect(x_1, \hdots, x_i)\ (1 \leq i \leq n)$
		\item $\langle e_i,x_i \rangle > 0\ (1 \leq i \leq n)$
	\end{enumerate}
\end{theo}

\begin{prop} $ $
	\begin{enumerate}
		\item [-] Soit $(x_i)_{i\in I}$ une famille orthogonale de vecteurs non nuls de $E$, alors la famille est libre. En particulier, toute
			famille orthonorm\'ee de vecteurs de $E$ est libre
		\item [-] Soit $E$ un Rev de dimension finie  $\langle\ \rangle$ un ps sur $E$, alors $E$ admet une base orthonorm\'ee
			\begin{rqs} Travailler ser des bases orthonorm\'ees permet de simplifier les calculs sur les ps \end{rqs}
	\end{enumerate}
\end{prop}

\begin{defi} Angle g\'eom\'etrique \\
	Soient $x,y \in E\backslash \{0\}$. Par d\'efinition, l'angle g\'eom\'etrique de $x$ et $y$ est:
	\[\theta = \arccos \frac{\langle x,y \rangle}{\|x\| \|y\|} \in [0, \pi]\]
	Ainsi, $\langle x,y \rangle = \|x\| \|y\| \cos \theta$
\end{defi}

%Orthogonal d'une partie {{{1
\section{Orthogonal d'une partie}
\begin{defi} Ensembles orthogonaux
	\begin{enumerate}
		\item Soit $E$ un Rev muni d'un ps $\langle\ \rangle$ et $A$ une partie non vide de $E$. L'orthogonal de $A$ est l'ensemble $A^{\perp} =
			\{x \in E,\ \forall a \in A\ \langle x,a \rangle = 0\}$
		\item Soient $A,B \subset E$. On dit que $A$ et $B$ sont orthogonaux ($A \perp B$) si $\forall a \in A,\ \forall b \in B,\ a \perp b
			\ (\Leftrightarrow A \subset B^\perp$ ou $B \subset A^\perp$)
	\end{enumerate}
\end{defi}

\begin{prop} $ $
	\begin{enumerate}
		\item [-] $A^{\perp}$ est un sev de $E$
		\item [-] $A \subset B \Rightarrow B^\perp \subset A^\perp$
		\item [-] $A \subset A^{\perp \perp}$
		\item [-] $A^\perp = Vect(A)^\perp$
	\end{enumerate}
\end{prop}

\begin{theo} Soit $F$ un sev de dimension finie de $E$, alors $E = F \oplus F^\perp$
	\begin{rqs}
		Soit $F$ sev de $E$. Si $E = F \oplus F^\perp$ alors $F^{\perp \perp} = F$. Ainsi, Si $F$ est un sev de dimension finie de $E$ alors
		$F = F^{\perp \perp}$
	\end{rqs}
\end{theo}

\subsection{Projecteurs orthogonaux, sym\'etries orthogonales, distance \`a un sev}

Soit $F$ un sev de $E$ tel que $E = F \oplus F^\perp$. Soit $x \in E$, $x = y + z$ ($y \in F ,\ z \in F^\perp$)
\begin{defi} Projecteur
	\begin{itemize}
		\item [-] On appelle projecteur orthogonal sur $F$, le projecteur sur $F$ parall\`element \`a $F^{\perp}$:  $P_F(x) = y$
		\item [-] On dit \'egalement que $p$ est projecteur orthogonal si $Ker\ p \perp Im\  p\ (\Leftrightarrow Ker\ p
			\subset Im\ p^\perp$ ou $ Im\ p \subset Ker\ p^\perp$.)
	\end{itemize}
\end{defi}
\begin{defi} Sym\'etrie
	\begin{enumerate}
		\item [-] On appelle sym\'etrie orthogonale par rapport \`a $F$, la sym\'etrie par rapport \`a $F$ parall\'element \`a $F^\perp$: $s_F(x)
			= y - z$
		\item [-] On dit que $s$ est sym\'etrie orthogonale si $Ker\ (s-Id_E) \perp Ker\ (s + Id_E)$
	\end{enumerate}
\end{defi}
\begin{defi} Distance \\
	On appelle distance de $x$ \`a $F$ $d(x,F) = \inf \{ \|x-y'\| ,\ y' \in F\}$ \\
	Ainsi $d(x,F) = \|z\|$ o\`u $z = x - P_F(x)$
\end{defi}

\begin{prop} Norme projections, sym\'etries
	\begin{enumerate}
		\item [-] $p$ est un projecteur orthogonal $\Leftrightarrow \forall x \in E, \|p(x)\| \leq \|x\|$
		\item [-] $s$ est une sym\'etrie orthogonale $\Leftrightarrow \forall x \in E, \|s(x)\| = \|x\|$
	\end{enumerate}
\end{prop}


%Espaces euclidiens {{{1
\section{Espaces euclidiens}

\begin{defi}
	Un espace euclidien est un couple $(E, \langle\  \rangle)$ o\`u $E$ est un Rev de dimension finie et $\langle\  \rangle$ un ps sur $E$
\end{defi}

\begin{prop} Faits de base
	\begin{enumerate}
		\item [-] $E$ admet des bases orhonorm\'ees
		\item [-] Soit $F$ un sev de $E$, alors $E = F \oplus F^\perp$ ($\Rightarrow \dim F^\perp = \dim E - \dim F$ et $F=F^{\perp \perp}$)
		\item [-] Soient $F$ et $G$ deux sev de $E$ alors:
			\begin{enumerate}
				\item $(F+G)^\perp = F^\perp \cap G^\perp$
				\item $(F\cap G)^\perp = F^\perp + G^\perp$
			\end{enumerate}
	\end{enumerate}
\end{prop}

\begin{theo} BON incompl\`ete\\
	On peut compl\'eter toute famille orthonorm\'ee de vecteurs de $E$ en base orthonorm\'ee
	\begin{rqs}
		Dans le cas $|E| = 2$, si $x$ est vecteur unitaire de $E$, il existe exactement deux vecteurs de $E$ pour compl\'eter la BON
	\end{rqs}
\end{theo}

\subsection{Hyperplans}

\begin{theo} Formes lin\'eaires \\
	Si $\varphi$ est une forme lin\'eaire sur $E$ alors il existe un unique $a \in E$ tel que $\varphi = \langle a,. \rangle$ \\
	Ainsi, si $H$ est un hyperplan, il existe alors $a \in E\backslash \{0\}$ tel que: \\$H = Ker\ \langle a,. \rangle = \{x \in E ,\ \langle a,x
	\rangle = 0\} = \{a\}^\perp$
\end{theo}

\begin{prop}
	Pour $a,a' \in E\backslash \{0\}$, $\{a\}^\perp = \{a'\}^\perp \Leftrightarrow \exists \alpha \in \mathbb{R}^*, a' = \alpha a$
\end{prop}

\begin{prop}
	Soit $H$ un hyperplan. Pour $x \in E$, $d(x,H) = d(x, \{a\}^\perp) = \frac {|\langle a,x \rangle|}{\|a\|}$
\end{prop}

\begin{defi}
	On appelle r\'eflexion toute sym\'etrie orthogonale par rapport \`a un hyperplan.
\end{defi}

%Le groupe orthogonal {{{1
\section{Le groupe orthogonal}

\begin{defi}
	Soit $f \in \mathscr{L}(E)$. On dit que $f$ est une isom\'etrie (ou transformation orthogonale) si $f$ conserve la norme: $\forall x \in E,
	\|f(x)\| = \|x\|$
\end{defi}

\begin{prop} Ensemble $O(E)$ des isom\'etries de $E$
	\begin{enumerate}
		\item [-] $O(E) \subset GL(E)$
		\item [-] $O(E)$ est un sous groupe de $(GL(E),\circ)$
	\end{enumerate}
\end{prop}

\begin{prop} Isom\'etries
	\begin{enumerate}
		\item [-] $\forall x,y \in F,\ \langle f(x), f(y) \rangle = \langle x,y \rangle$
			\begin{enumerate}
				\item $\forall x,y \in E,\ \langle x,y \rangle = 0 \Leftrightarrow \langle f(x), f(y) \rangle = 0$
				\item $f$ conserve les angles g\'eom\'etriques
			\end{enumerate}
		\item [-] Soit $f \in O(E)$ et $F$ un sev de $E$. On suppose $F$ stable par $f$ alors, $F^\perp$ est stable par $f$
		\item [-] Soit $B = (e_1, \hdots, e_n)$ une BON de $E$, $f \in \mathscr{L}(E)$. On a alors:\\
			$f \in O(E) \Leftrightarrow (f(e_1), \hdots, f(e_n))$ est une BON de $F$
	\end{enumerate}
\end{prop}

\subsection{Matrices orthogonales}
\begin{defi} Soit $M \in M_n(\mathbb{R})$. On dit que $M$ est une matrice orthogonale ssi l'endomorphisme de $\mathbb{R}^n$ canoniquement
associ\'e \`a $M$ est une isom\'etrie de l'espace euclidien $(\mathbb{R}^n,$ (ps canonique)). On note $O_n(\mathbb{R})$ l'ensemble des matrices
orthogonales
\end{defi}

\begin{prop} $ $
	\begin{enumerate}
		\item $O_n(\mathbb{R})$ est un ss gpe de $(GL_n(\mathbb{R}), \times)$
		\item $M \in O_n(\mathbb{R}) \Leftrightarrow (C_1(M), \hdots, C_n(M))$ est une BON de $(\mathbb{R}^n, .)$
		\item Soit $M \in M_n(\mathbb{R})$, LASSE:
			\begin{enumerate}
				\item $M \in O_n(\mathbb{R})$
				\item $t_M \times M = I_n$
				\item $M \times t_M = I_n$
				\item $M$ est inversible et $M^{-1} = t_M$
			\end{enumerate}
		\item $M \in O_n(\mathbb{R}) \Rightarrow |\det| = 1$.  Il existe des matrices orthogonales de determinant 1 et -1. On note
			$SO_n(\mathbb{R})$ l'ensemble des matrices orthogonales de d\'eterminant 1.
		\item Soit $(E, \langle  \rangle)$ un espace euclidien, $B$ une BON de $E$, $f \in \mathscr{L}(E)$, alors $f \in O(E) \Leftrightarrow Mat_B(f)
			\in O_n(\mathbb{R})$
		\item $f \in O(E) \Rightarrow |\det f| = 1$
		\item Soit $P$ la matrice de passage d'une BON vers une autre BON alors $P \in O_n(\mathbb{R})$
	\end{enumerate}
\end{prop}

\subsection{Produit mixte}

\begin{prop}
	Soit $f \in \mathscr{L}(E)$ et $\beta$ une BOND. On a:
	\begin{itemize}
		\item [-] $f \in SO(E) \Leftrightarrow f(\beta)$ est une BOND
		\item [-] $f \in O(E) \backslash SO(E) \Leftrightarrow f(\beta)$ est une BONR
	\end{itemize}
\end{prop}

\begin{defi} Produit mixte \\
	Soit $(E, \langle\   \rangle)$ un espace euclidien orient\'e de dimension $n$. Pour $B,C$ deux BOND de $E$ on a $\det_B = \det_C$.
	Par d\'efinition, pour $(x_1, \hdots, x_n) \in E^n$, le produit mixte de $(x_1, \hdots, x_n)$ est $[x_1, \hdots, x_n] =
	\det_B (x_1, \hdots, x_n)$ Avec $B$ une BOND quelconque.
\end{defi}

\subsection{Produit vectoriel en dimension 3}

\begin{defi}
	Soit $E$ euclidien orient\'e de dimension trois. Pour tout $(x,y) \in E^2$, il existe un et un seul $w \in E$ appell\'e produit vectoriel de
	$x$ et $y$ tel que $\forall x \in E,\ [x,y,z] = \langle w,z \rangle$
\end{defi}

\begin{prop} $ $
	\begin{enumerate}
		\item [-] $x \wedge y \in \{x,y\}^\perp$
		\item [-] $(x,y) \in E^2 \mapsto x \wedge y$ est bilin\'eaire altern\'e (donc antysim\'etrique)
		\item [-] Soit $(x,y) \in E^2$:
			\begin{enumerate}
				\item Si $(x,y)$ est li\'ee alors $x \wedge y = 0_E$
				\item Si $(x,y)$ est libre alors $(x,y,x \wedge y)$ est une base directe de $E$
				\item $x \wedge y = 0 \Leftrightarrow (x,y)$ li\'ee
			\end{enumerate}
		\item [-] Soit $(e_1, e_2, e_3)$ une BOND de $E$, alors:
			\begin{enumerate}
				\item $e_1 \wedge e_2 = e_3$
				\item $e_2 \wedge e_3 = e_1$
				\item $e_3 \wedge e_1 = e_2$
			\end{enumerate}
		\item [-] Si $a$ et $b$ sont deux vecteurs unitaires orthogonaux, alors $a \wedge b$ est l'unique vecteur de $E$ tel que
			$(a,b,a \wedge b)$ est une BOND de $E$
		\item [-] $\| a \wedge b \| = \|a\| \|b\| \sin \theta$ (avec $\theta$ l'angle g\'eom\'etrique)
	\end{enumerate}
\end{prop}

\end{document}
