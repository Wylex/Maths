%Préamble {{{1
\documentclass[fleqn]{article}

\usepackage{amssymb}
\usepackage{amsmath}
\usepackage{amsthm}
\usepackage{verbatim}
\usepackage{booktabs}
\usepackage{mathrsfs}

\theoremstyle{definition} \newtheorem*{defi}{D\'efinition}
\theoremstyle{definition} \newtheorem*{theo}{Th\'eor\`eme}
\theoremstyle{definition} \newtheorem*{coro}{Corollaire}
\theoremstyle{remark} \newtheorem*{rqs}{Remarques}
\theoremstyle{definition} \newtheorem*{prop}{Propri\'et\'e}
\newcommand{\ra}[1]{\renewcommand{\arraystretch}{#1}}
\ra{1.3}

\title{Espaces euclidiens}
\date{}

\begin{document}
\maketitle

%Produit scalaire, norme euclidienne {{{1
\section{Produit scalaire, norme euclidienne}
\subsection{Produit scalaire}
\begin{defi} Soit $E$ un Rev, un produit scalaire sur $E$ est une application $\varphi : E \times E \rightarrow \mathbb{R}$ telle que:
\begin{enumerate}
	\item $\varphi$ est bilin\'eaire
	\item $\varphi$ est sym\'etrique: $\varphi(x,y) = \varphi(y,x)$
	\item $\varphi$ est d\'efinie positive: $\forall x \in E \backslash \{0\},\ \varphi(x,x) > 0$
\end{enumerate}
\end{defi}

\begin{rqs} Soit $\varphi$ bilin\'eaire, alors $\varphi$ est d\'efinie positive $\Leftrightarrow$ \begin{enumerate}
	\item $\forall x \in E,\ \varphi(x,x) \geq 0$
	\item $\varphi(x,x) = 0 \Rightarrow x = 0$
	\end{enumerate}
\end{rqs}

\subsection{Norme euclidienne}
\begin{defi} Soit $E$ un Rev. Une norme sur $E$ est une application $N:E \rightarrow \mathbb{R}_+$ telle que:
	\begin{enumerate}
		\item Pour $x \in E, N(x) = 0 \Leftrightarrow x = 0$
		\item Pour $x \in E, \alpha \in \mathbb{R}, N(\alpha x) = |\alpha| N(x)$
		\item Pour $x,y \in E,\ N(x+y) \leq N(x) + N(y)$
	\end{enumerate}
\end{defi}

\begin{theo} Soit $E$ un Rev et $<,>$ un ps sur $E$, pour $x \in E$, posons \mbox{$\|x\| = \sqrt{<x,x>}$}, alors $x \in E \mapsto \|x\|$ est
une norme sur $E$ appel\'ee norme euclidienne associ\'e au produit scalaire $<,>$
\end{theo}

\begin{rqs} In\'egalit\'e de Cauchy-Schwarz\\
	$\forall x,y \in E,\ |<x,y>| \leq \|x\| \|y\|$
\end{rqs}

\begin{prop} Relations entre ps et $\|.\|$
	\begin{enumerate}
		\item $\|\sum_{i=1}^n x_i \| ^2 = \sum_{i=1}^n \|x_i\|^2 + 2\sum_{1 \leq i < j \leq n} <x_i,x_j>$
		\item En particulier, (identit\'e du parall\'elogramme):\\
			$\|x+y\|^2 + \|x-y\|^2 = 2(\|x\|^2 + \|y\|^2)$
		\item $4<x,y> = \|x+y\|^2 - \|x-y\|^2$
	\end{enumerate}
\end{prop}

%Familles orthogonales, processus d'orthonormalisation {{{1
\section{Familles orthogonales, processus d'orthonormalisation}

\begin{defi} Vecteurs orthogonaux \\
	Soit $<,>$ un ps sur le Rev $E$, $x,y \in E$.
	On dit que $x$ est orthogonal \`a $y$ si $<x,y> = 0$

	\begin{rqs}
		Si $x_i \perp x_j,\ i \neq j$ alors $\|\sum_{i=1}^n x_i \|^2 = \sum_{i=1}^n \|x_i\|^2$
	\end{rqs}
\end{defi}

\begin{defi} Angle g\'eom\'etrique \\
Soient $x,y \in E\backslash \{0\}$. Par d\'efinition, l'angle g\'eom\'etrique de $x$ et $y$ est $\theta = \arccos
\frac{<x,y>}{\|x\| \|y\|} \in [0, \pi].$\\
Ainsi, $<x,y> = \|x\| \|y\| \cos \theta$
\end{defi}

\begin{defi} Famille orthogonale \\
Soit $E$ un Rev muni d'un ps $<>$, $(x_i)_{i\in I}$ une famille de vecteurs de $E$. On dit que la famille est orthogonale si
$x_i \perp x_j$ pour $i\neq j$. Elle est orthonorm\'ee si de plus $\|x_i\| = 1$
\end{defi}

\begin{theo} Soit $(x_i)_{i\in I}$ une famille orthogonale de vecteurs non nuls de $E$, alors la famille est libre. En particulier, toute famille
orthonorm\'ee de vecteurs de $E$ est libre
\end{theo}

\begin{theo} Orthonormalisation de Gram-Schmidt \\
Soit $E$ un Rev muni d'un ps $<>,\ (x_1, \hdots, x_n)$ une famille libre de vecteurs de $E$. Alors il existe une unique famille orthonormale
$(e_1, \hdots, e_n)$ de vecteurs de $E$ telle que:
\begin{enumerate}
	\item $Vect(e_1, \hdots, e_i) = Vect(x_1, \hdots, x_i)\ (1 \leq i \leq n)$
	\item $<e_i,x_i> >0\ (1 \leq i \leq n)$
\end{enumerate}
\end{theo}

\begin{theo} Soit $E$ un Rev de dimension finie  $<,>$ un ps sur $E$, alors $E$ admet une base orthonorm\'ee
	\begin{rqs}
	Travailler ser des bases orthonorm\'ees permet de simplifier les calculs sur les ps
	\end{rqs}
\end{theo}

%Orthogonal d'une partie {{{1
\section{Orthogonal d'une partie}
\begin{defi} Soit $E$ un Rev muni d'un ps $<,>$, soit $A$ une partie non vide de $E$. L'orthogonal de $A$ est l'ensemble $A^{\perp} =
	\{x \in E,\ \forall a \in A\ <x,a> = 0\}$
\end{defi}

\begin{prop} $ $
	\begin{enumerate}
		\item [-] $A^{\perp}$ est un sev de $E$
		\item [-] Pour $A,B \subset E,\ A \subset B \Rightarrow B^\perp \subset A^\perp$
		\item [-] $A \subset A^{\perp \perp}$
		\item [-] $A^\perp = Vect(A)^\perp$
	\end{enumerate}
\end{prop}

\begin{defi} Soient $A,B \subset E$. On dit que $A$ et $B$ sont orthogonaux ($A \perp B$) si $\forall a \in A,\ \forall b \in B,\ a \perp b
	\Leftrightarrow A \subset B^\perp$ ou $B \subset A^\perp$
\end{defi}

\begin{theo} Soit $F$ un sev de dimension finie de $E$, alors $E = F \oplus F^\perp$
	\begin{rqs}
		Soit $F$ sev de $E$. Si $E = F \oplus F^\perp$ alors $F^{\perp \perp} = F$. Ainsi, Si $F$ est un sev de dimension finie de $E$ alors
		$F = F^{\perp \perp}$
	\end{rqs}
\end{theo}

\subsection{Projecteurs orthogonaux, sym\'etries orthogonales, distance \`a un sev}
Soit $F$ un sev de $E$ tel que $E = F \oplus F^\perp$. Soit $x \in E$, $x = y + z$ ($y \in F ,\ z \in F^\perp$)
\begin{defi} Projecteur \\
	On appelle projecteur orthogonal sur $F$, le projecteur sur $F$ parall\`element \`a $F^{\perp}$ \\
	$P_F(x) = y$
\end{defi}
\begin{defi} Sym\'etrie \\
	On appelle sym\'etrie orthogonale par rapport \`a $F$, la sym\'etrie par rapport \`a $F$ parall\'element \`a $F^\perp$: $s_F(x) = y - z$
\end{defi}
\begin{defi} Distance \\
	On appelle distance de $x$ \`a $F$ $d(x,F) = \inf \{ \|x-y'\| ,\ y' \in F\}$ \\
	Alors $d(x,F) = \|z\|$ o\`u $z = x - P_F(x)$
\end{defi}


%Espaces euclidiens {{{1
\section{Espaces euclidiens}

\begin{defi}
	Un espace euclidien est un couple $(E, <>)$ o\`u $E$ est un Rev de dimension finie et $<>$ un ps sur $E$
\end{defi}

\begin{prop} Faits de base
	\begin{enumerate}
		\item [-] $E$ admet des bases orhonorm\'ees
		\item [-] Soit $F$ un sev de $E$, alors $E = F \oplus F^\perp$ ($\Rightarrow \dim F^\perp = \dim E - \dim F$ et $F=F^{\perp \perp}$)
		\item [-] Soient $F$ et $G$ deux sev de $E$ alors:
			\begin{enumerate}
				\item $(F+G)^\perp = F^\perp \cap G^\perp$
				\item $(F\cap G)^\perp = F^\perp + G^\perp$
			\end{enumerate}
	\end{enumerate}
\end{prop}

\begin{theo} BON incompl\`ete\\
	On peut compl\'eter toute famille orthonorm\'ee de vecteurs de $E$ en base orthonorm\'ee
	\begin{rqs}
		Dans le cas $|E| = 2$, si $x$ est vecteur unitaire de $E$, il existe exactement deux vecteurs de $E$ pour compl\'eter la BON
	\end{rqs}
\end{theo}

\subsection{Hyperplans}

\begin{theo}
	Si $\varphi$ est une forme lin\'eaire sur $E$ alors il existe un unique $a \in E$ tel que $\varphi = <a,.>$ \\
	Soit $H$ un hyperplan, il existe alors $a \in E\backslash \{0\}$ tel que $H = \ker <a,.> = \{x \in E , <a,x> = 0\} = \{a\}^\perp$
\end{theo}

\begin{prop}
	Pour $a,a' \in E\backslash \{0\}$, $\{a\}^\perp = \{a'\}^\perp \Leftrightarrow \exists \alpha \in \mathbb{R}^*, a' = \alpha a$
\end{prop}

\begin{prop}
	Soit $H$ un hyperplan. Pour $x \in E$, $d(x,H) = d(x, \{a\}^\perp) = \frac {|<a,x>|}{\|a\|}$
\end{prop}

\begin{defi}
	On appelle r\'eflexion toute sym\'etrie orthogonale par rapport \`a un hyperplan.
\end{defi}

\begin{defi} $ $
	\begin{enumerate}
		\item Soit $p$ un projecteur de $E$, on dit que $p$ est projecteur orthogonal si $Ker\ p \perp Im\  p \Leftrightarrow Ker\ p
			\subset Im\ p^\perp$ ou $ Im\ p \subset Ker\ p^\perp$ .
		\item Soit $s$ une sym\'etrie de $E$, on dit que $s$ est sym\'etrie orthogonale si $Ker\ (s-Id_E) \perp Ker\ (s + Id_E)$
	\end{enumerate}
\end{defi}

\begin{prop} $ $
	\begin{enumerate}
		\item $p$ est un projecteur orthogonal $\Leftrightarrow \forall x \in E, \|p(x)\| \leq \|x\|$
		\item $s$ est une sym\'etrie orthogonale $\Leftrightarrow \forall x \in E, \|s(x)\| = \|x\|$
	\end{enumerate}
\end{prop}

%Le groupe orthogonal {{{1
\section{Le groupe orthogonal}

\begin{defi}
	Soit $f \in \mathscr{L}(E)$. On dit que $f$ est une isom\'etrie (ou transformation orthogonale) si $f$ conserve la norme: $\forall x \in E,
	\|f(x)\| = \|x\|$
\end{defi}

\begin{prop} On note $O(E)$ l'ensemble des isom\'etries de $E$
	\begin{enumerate}
		\item $O(E) \subset GL(E)$
		\item $O(E)$ est un sous groupe de $(GL(E),\circ)$
	\end{enumerate}
\end{prop}

\end{document}
