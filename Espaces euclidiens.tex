%Préamble {{{1
\documentclass[fleqn]{article}

\usepackage{amssymb}
\usepackage{amsmath}
\usepackage{amsthm}
\usepackage{verbatim}
\usepackage{booktabs}
\usepackage{mathrsfs}

\theoremstyle{definition} \newtheorem*{defi}{D\'efinition}
\theoremstyle{definition} \newtheorem*{theo}{Th\'eor\`eme}
\theoremstyle{definition} \newtheorem*{coro}{Corollaire}
\theoremstyle{remark} \newtheorem*{rqs}{Remarques}
\theoremstyle{definition} \newtheorem*{prop}{Propri\'et\'e}
\newcommand{\ra}[1]{\renewcommand{\arraystretch}{#1}}
\ra{1.3}

\title{Espaces euclidiens}
\date{}

\begin{document}
\maketitle

%Produit scalaire, norme euclidienne {{{1
\section{Produit scalaire, norme euclidienne}
\subsection{Produit scalaire}
\begin{defi} Soit $E$ un Rev, un produit scalaire sur $E$ est une application $\varphi : E \times E \rightarrow \mathbb{R}$ telle que:
\begin{enumerate}
	\item $\varphi$ est bilin\'eaire
	\item $\varphi$ est sym\'etrique: $\varphi(x,y) = \varphi(y,x)$
	\item $\varphi$ est d\'efinie positive: $\forall x \in E \backslash \{0\},\ \varphi(x,x) > 0$
\end{enumerate}
\end{defi}

\begin{rqs} Soit $\varphi$ bilin\'eaire, alors $\varphi$ est d\'efinie positive $\Leftrightarrow$ \begin{enumerate}
	\item $\forall x \in E, \varphi(x,x) \geq 0$
	\item $\varphi(x,x) = 0 \Rightarrow x = 0$
	\end{enumerate}
\end{rqs}

\subsection{Norme euclidienne}
\begin{defi} Soit $E$ un Rev. Une norme sur $E$ est une application $N:E \rightarrow \mathbb{R}_+$ telle que:
	\begin{enumerate}
		\item Pour $x \in E, N(x) = 0 \Leftrightarrow x = 0$
		\item Pour $x \in E, \alpha \in \mathbb{R}, N(\alpha x) = |\alpha| N(x)$
		\item Pour $x,y \in E,\ N(x+y) \leq N(x) + N(y)$
	\end{enumerate}
\end{defi}

\begin{theo} Soit $E$ un Rev et $<,>$ un ps sur $E$, pour $x \in E$, posons $\|x\| = \sqrt{<x,x>}$. Alors $x \in E \mapsto \|x\|$ est
une norme sur $E$ appel\'ee norme euclidienne (associ\'e au ps $<,>$)
\end{theo}

\begin{rqs} In\'egalit\'e de Cauchy-Schwarz\\
	$\forall x,y \in E,\ |<x,y>| \leq \|x\| \|y\|$
\end{rqs}

\begin{prop} Quelques relations
	\begin{enumerate}
		\item $\|\sum_{i=1}^n x_i \| ^2 = \sum_{i=1}^n \|x_i\|^2 + 2\sum_{1 \leq i < j \leq n} <x_i,x_j>$
		\item En particulier, (identit\'e du parall\'elogramme):\\
			$\|x+y\|^2 + \|x-y\|^2 = 2(\|x\|^2 + \|y\|^2)$
		\item $4<x,y> = \|x+y\|^2 - \|x-y\|^2$
	\end{enumerate}
\end{prop}

\begin{defi} Vecteurs orthogonaux \\
	Soit $<,>$ un ps sur le Rev $E$, $x,y \in E$.
	On dit que $x$ est orthogonal \`a $y$ si $<x,y> = 0$

	\begin{rqs}
		Si $x_i \perp x_j,\ i \neq j$ alors $\|\sum_{i=1}^n x_i \|^2 = \sum_{i=1}^n \|x_i\|^2$
	\end{rqs}
\end{defi}

\begin{defi} Angle g\'eom\'etrique \\
Soient $x,y \in E\backslash \{0\}$. Par d\'efinition, l'angle g\'eom\'etrique de $x$ et $y$ est $\theta = \arccos
\frac{<x,y>}{\|x\| \|y\|} \in [0, \pi].$\\
Ainsi, $<x,y> = \|x\| \|y\| \cos \theta$
\end{defi}

\subsection{Familles orthogonales}
\begin{defi} Soit $E$ un Rev muni d'un ps $<>$, $(x_i)_{i\in I}$ une famille de vecteurs de $E$. On dit que la famille est orthogonale si
$x_i \perp x_j$ pour $i\neq j$. Elle est orthonorm\'ee si de plus $\|x_i\| = 1$
\end{defi}

\begin{theo} Soit $(x_i)_{i\in I}$ une famille orthogonale de vecteurs non nuls de $E$, alors la famille est libre. En particulier, toute famille
orthonorm\'ee de vecteurs de $E$ est libre
\end{theo}

\begin{theo} Orthonormalisation de Gram-Schmidt \\
Soit $E$ un Rev muni d'un ps $<>. (x_1, \hdots, x_n)$ une famille libre de vecteurs de $E$. Alors il existe une unique famille orthonormale
$(e_1, \hdots, e_n)$ de vecteurs de $E$ telle que:
\begin{enumerate}
	\item $Vect(e_1, \hdots, e_i) = Vect(x_1, \hdots, x_i)\ (1 \leq i \leq n)$
	\item $<e_i,x_i> >0\ (1 \leq i \leq n)$
\end{enumerate}

\end{theo}


\end{document}
