\documentclass[fleqn]{article}
\usepackage{amssymb}
\usepackage{amsmath}
\usepackage{amsthm}

\title{Bric \`a Brac}
\date{}

\theoremstyle{definition} \newtheorem*{defi}{D\'efinition}
\theoremstyle{definition} \newtheorem*{theo}{Th\'eor\`eme}
\theoremstyle{definition} \newtheorem*{coro}{Corollaire}
\theoremstyle{remark} \newtheorem*{rqs}{Remarques}
\theoremstyle{definition} \newtheorem*{prop}{Propri\'et\'e}
\newcommand{\ra}[1]{\renewcommand{\arraystretch}{#1}}
\ra{1.3}

\begin{document}
\maketitle

\section{Logique}
\begin{enumerate}
	\item N\'egation des connecteurs:
		\begin{enumerate}
			\item  $\overline{\overline{P}} \equiv  P$
			\item  $\overline{P \lor Q} \equiv \overline{P} \land \overline{Q}$
			\item  $\overline{P \land Q} \equiv \overline{P} \lor \overline{Q}$
			\item  $\overline{P \Rightarrow Q} \equiv P \land \overline{Q}$
		\end{enumerate}
	\item N\'egation des quantificateurs:
		\begin{enumerate}
			\item $\overline{\forall x \in E, P(x)} \equiv \exists x \in E,\ \overline{P(x)}$
			\item $\overline{\exists x \in E, P(x)} \equiv \forall x \in E,\ \overline{P(x)}$
		\end{enumerate}
	\item Contraposition: \\
		$P \Rightarrow Q \equiv \overline{Q} \Rightarrow \overline{P}$
\end{enumerate}

\section{Ensembles}
Soient $E$ et $F$ deux ensembles:
\begin{enumerate}
	\item Distributivit\'e: \\
		\(E \cap(F \cup G) = (E \cap F) \cup (E \cap G)\) \\
		\(E \cup (F \cap G) = (E \cup F) \cap (E \cup G)\)
	\item Compl\'ementaires: soient \(A, B \in \mathcal{P}(E)\): \\
		\((A \cap B)^c = A^c \cup B^c\) \\
		\((A \cup B)^c = A^c \cap B^c\)
\end{enumerate}

\section{Applications}
\begin{defi}
	\(E, F\) deux ensembles non vides, \( f \in \mathcal{F}(E, F)\):
	\begin{enumerate}
		\item \(f\) est injective si \(\forall x,x' \in E,\ x \neq x' \Rightarrow f(x) \neq f(x')\) \\
		ou de fa\c{c}on \'equivalente, \(\forall x,x' \in E,\ f(x) = f(x') \Rightarrow x = x'\)
		\item \(f\) est surjective si \(\forall y \in F, \exists x \in E, f(x) = y\)
		\item \(f\) est bijective si elle est \`a la fois injective et surjective
	\end{enumerate}
\end{defi}
\begin{prop} $ $
	\begin{enumerate}
		\item Composition:
			\begin{enumerate}
				\item \(f: E \rightarrow F,\ g: F \rightarrow G\) \\
					\(f\) et \(g\) injectives \(\Rightarrow\) \(g \circ f\) injective \\
					\(f\) et \(g\) surjectives \(\Rightarrow\) \(g \circ f\) surjectives \\
					\(f\) et \(g\) bijective \(\Rightarrow\) \(g \circ f\) bijective
				\item \(f: E \rightarrow F\) bijective, alors \(g = f^{-1}\) est l'unique application telle que: \\
					\(f \circ g = I_{dF}\) \\
					\(g \circ f = I_{dE}\) \\
					(S'il existe une telle $g$ alors $f$ est bijective)
				\item Associativit\'e de la composition: \(f: E \rightarrow F,\ g: F \rightarrow G\,\ h: G \rightarrow H\), \\
					\(h \circ (g \circ f) = (h \circ g) \circ f\)
			\end{enumerate}
		\item Image directe, image r\'eciproque:
			\begin{enumerate}
				\item \(A, A' \in \mathcal{P}(E)\) \\
					\(f(A \cup A') = f(A) \cup f(A')\)
				\item \(B, B' \in \mathcal{P}(F)\) \\
					\(f^{-1}(B \cup B') = f^{-1}(B) \cup f^{-1}(B')\) \\
					\(f^{-1}(B \cap B') = f^{-1}(B) \cap f^{-1}(B')\) \\
					\(f^{-1}(B^c) = (f^{-1}(B))^c\)
			\end{enumerate}
	\end{enumerate}
\end{prop}

\section{Relations Binaires}
Soit \(R\) une relation binaire ser l'ensemble \(E\):
\begin{enumerate}
	\item \(R\) est r\'eflexive si \(\forall x \in E, xRx\)
	\item \(R\) est sym\'etrique si \(\forall x,y \in E, xRy \Rightarrow yRx\)
	\item \(R\) est antisym\'etrique si \(\forall x,y \in E, (xRy \land yRx) \Rightarrow x=y\)
	\item \(R\) est transitive si \(\forall x,y,z \in E, (xRy \land yRz) \Rightarrow xRz\)
\end{enumerate}

\subsection{Relations d'\'equivalence}
\begin{defi}
	On dit que \(R\) est une relation d'\'equivalence si \(R\) est r\'eflexive, sym\'etrique et transitive
\end{defi}
\begin{prop} La classe d'\'equivalence
	\[Cl_R(x) = \{y \in E ,xRy\}\]
\end{prop}

\subsection{Relations d'ordre}
\begin{defi} $ $
	\begin{enumerate}
		\item \(R\) est une relation d'ordre si \(R\) est r\'eflexive, antisym\'etrique et transitive \\
			Dans ce cas, \((E,R)\) est un ordre ou un ensemble ordonn\'ee.
		\item L'ordre \((E,R)\) est totale si \(\forall x,y \in E,\ xRy \lor yRx\)
	\end{enumerate}
\end{defi}

\begin{prop} $ $
	\begin{enumerate}
		\item Maximums et minimums: \\
			Soit \((E, \preceq)\) un ordre. Soit \(A \subset E, A \neq \emptyset\). Un \'el\'ement \(a \in A\) est appel\'e:
			\begin{itemize}
				\item [-] maximum de \(A\) si \(\forall x \in A, x \preceq a\)
				\item [-] minimum de \(A\) si \(\forall x \in A, a \preceq x\)
			\end{itemize}
		\item Majorants et minorants:
			\begin{itemize}
				\item [-] A est major\'ee s'il existe \(x \in E, \forall a \in A, a \preceq x\)
				\item [-] A est minor\'ee s'il existe \(x \in E, \forall a \in A, x \preceq a\)
			\end{itemize}
		\item Bornes sup\'erieures et inf\'erieures:
			\begin{itemize}
				\item [-] Soit \(A \subset E\), A major\'ee. \\
					En cas d'existence, la borne sup\'erieure de A est le minimum de l'ensemble B des majorants de A. Not\'ee $\sup A$
				\item [-] Supposons A minor\'ee, en cas d'existence, la borne inf\'erieure de A est le maximum de l'ensemble B des minorants de A.
					Not\'ee $\inf A$ \\
			\end{itemize}
	\end{enumerate}
\end{prop}


\end{document}
