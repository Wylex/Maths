\documentclass[fleqn]{article}
\usepackage{amssymb}
\usepackage{amsmath}
\usepackage{amsthm}

\title{Bric \`a Brac}
\date{}

\theoremstyle{definition} \newtheorem*{defi}{D\'efinition}
\theoremstyle{plain} \newtheorem*{theo}{Th\'eor\`eme}

\begin{document}
\maketitle

\section{Un peu de logique}
\begin{itemize}
	\item N\'egation des connecteurs: \\
		NON(NON(\(P) \equiv  P\) \\
		NON(\(P \lor Q) \equiv \) (NON P) \(\land\) (NON Q) \\
		NON(\(P \land Q) \equiv \) (NON P) \(\lor\) (NON Q) \\
		NON(\(P \Rightarrow Q) \equiv P \land\) NON(\(Q\))
	\item N\'egation des quantificateurs: \\
		NON(\(\forall x \in E, P(x)) \equiv \exists x \in E\), NON(\(P(x)\)) \\
		NON(\(\exists x \in E, P(x)) \equiv \forall x \in E\), NON(\(P(x)\))
	\item Contraposition: \\
		\((P \Rightarrow Q) \equiv\) (NON \(Q)\Rightarrow\)(NON \(P\))
\end{itemize}

\section{Ensembles}
\begin{itemize}
	\item $(E \cap F) \subset E \subset (E \cup F)$ \\
		$(E \cap F) \subset F \subset (E \cup F)$ \\
		$E \cap \emptyset = \emptyset$ \\
		$E \cup \emptyset = E$
	\item Distributivit\'e: \\
		\(E \cap(F \cup G) = (E \cap F) \cup (E \cap G)\) \\
		\(E \cup (F \cap G) = (E \cup F) \cap (E \cup G)\)
	\item si \(A, B \in \mathcal{P}(E)\): \\
		\((A \cap B)^c = A^c \cup B^c\) \\
		\((A \cup B)^c = A^c \cap B^c\)
\end{itemize}

\section{Applications}
\begin{defi}
	\(E, F\) deux ensembles non vides, \( f \in \mathcal{F}(E, F)\):
	\begin{enumerate}
		\item \(f\) est injective si \(\forall x,x' \in E, (x \neq x') \Rightarrow (f(x) \neq f(x'))\) \\
		ou de fa\c{c}on \'equivalente, \(\forall x,x' \in E, (f(x) = f(x')) \Rightarrow (x = x')\)
		\item \(f\) est surjective si \(\forall y \in F, \exists x \in E, f(x) = y\)
		\item \(f\) est bijective si elle est \`a la fois injective et surjective
	\end{enumerate}
\end{defi}
\begin{itemize}
	\item Composition:
		\begin{itemize}
			\item \(f: E \rightarrow F,\ g: F \rightarrow G\)
				\begin{enumerate}
					\item (\(f\) et \(g\) injectives) \(\Rightarrow\) \(g \circ f\) injective
					\item (\(f\) et \(g\) surjectives) \(\Rightarrow\) \(g \circ f\) surjectives
					\item (\(f\) et \(g\) bijective) \(\Rightarrow\) \(g \circ f\) bijective
				\end{enumerate}
			\item Soit \(f: E \rightarrow F\) bijective, alors \(f^{-1}\) est l'unique application telle que: \\
				\(f \circ f^{-1} = I_{dF}\) \\
				\(f^{-1} \circ f = I_{dE}\)
			\item \(f: E \rightarrow F,\ g: F \rightarrow G\,\ h: G \rightarrow H\) (associativit\'e de \(\circ\)), \\
				\(h \circ (g \circ f) = (h \circ g) \circ f\)
		\end{itemize}
	\item Image directe, image r\'eciproque
		\begin{enumerate}
			%Pour \(f: E \rightarrow F\): \\
			\item \(A, A' \in \mathcal{P}(E)\) \\
				\(f(A \cup A') = f(A) \cup f(A')\)
			\item \(B, B' \in \mathcal{P}(F)\) \\
				\(f^{-1}(B \cup B') = f^{-1}(B) \cup f^{-1}(B')\) \\
				\(f^{-1}(B \cap B') = f^{-1}(B) \cap f^{-1}(B')\) \\
				\(f^{-1}(B^c) = (f^{-1}(B))^c\)
		\end{enumerate}
\end{itemize}

\section{Relations Binaires}
Soit \(R\) une relation binaire ser l'ensemble \(E\):
\begin{enumerate}
	\item \(R\) est r\'eflexive si \(\forall x \in E, xRx\)
	\item \(R\) est sym\'etrique si \(\forall x,y \in E, xRy \Rightarrow yRx\)
	\item \(R\) est antisym\'etrique si \(\forall x,y \in E, (xRy \land yRx) \Rightarrow (x=y)\)
	\item \(R\) est transitive si \(\forall x,y,z \in E, (xRy \land yRz) \Rightarrow (xRz)\)
\end{enumerate}

\subsection{Relations d'\'equivalence}
\begin{itemize}
	\item D\'ef: On dit que \(R\) est une relation d'\'equivalence si \(R\) est r\'eflexive, sym\'etrique et transitive
	\item La classe d'\'equivalence de \(x\) est: \\
		\(Cl_R(x) = \{y \in E |xRy\}\)
\end{itemize}

\subsection{Relations d'ordre}
\begin{itemize}
	\item
		\begin{itemize}
			\item D\'ef: On dit que \(R\) est une relation d'ordre si \(R\) est r\'eflexive, antisym\'etrique et transitive \\
				Dans ce cas, \((E,R)\) est un ordre ou un ensemble ordonn\'ee.
			\item D\'ef: L'ordre \((E,R)\) est totale si \(\forall x,y \in E, (xRy) \lor (yRx)\)
		\end{itemize}
	\item Maximums et minimums: \\
		Soit \((E, \preceq)\) un ordre. Soit \(A \subset E, A \neq \emptyset\). Un \'el\'ement \(a \in A\) est appel\'e:
		\begin{itemize}
			\item maximum de \(A\) si \(\forall x \in A, x \preceq a\)
			\item minimum de \(A\) si \(\forall x \in A, a \preceq x\)
		\end{itemize}
	\item Majorants et minorants:
		\begin{itemize}
			\item A est major\'ee s'il existe \(x \in E, \forall a \in A, a \preceq x\)
			\item A est minor\'ee s'il existe \(x \in E, \forall a \in A, x \preceq a\)
		\end{itemize}
	\item Bornes sup\'erieures et inf\'erieures:
		\begin{itemize}
			\item Soit \(A \subset E\), A major\'ee. \\
			En cas d'existence, la borne sup\'erieure de A est le minimum de l'ensemble B des majorants de A. Not\'ee \emph{sup A}
			\item Supposons A minor\'ee, en cas d'existence, la borne inf\'erieure de A est le maximum de l'ensemble B des minorants de A. Not\'ee \emph{inf A} \\
		\end{itemize}
\end{itemize}


\end{document}
