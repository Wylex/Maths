\documentclass[fleqn]{article}
\usepackage{amssymb}
\usepackage{amsmath}
\usepackage{amsthm}
\usepackage{verbatim}
\usepackage{booktabs}

\title{Entiers naturels}
\date{}

\theoremstyle{definition} \newtheorem*{defi}{D\'efinition}
\theoremstyle{definition} \newtheorem*{theo}{Th\'eor\`eme}
\theoremstyle{definition} \newtheorem*{coro}{Corollaire}
\theoremstyle{remark} \newtheorem*{rqs}{Remarques}
\theoremstyle{definition} \newtheorem*{prop}{Propri\'et\'e}
\newcommand{\ra}[1]{\renewcommand{\arraystretch}{#1}}
\ra{1.3}

\begin{document}
\maketitle

\section{Axiomatique de p\'eano}
Il existe un ensemble non vide, not\'e $\mathbb{N}$, muni de relation d'ordre totale $\leq$ telle que:
\begin{enumerate}
	\item Toute partie non vide de $\mathbb{N}$ admet un minimum
	\item Toute partie non vide major\'ee de $\mathbb{N}$ admet un maximum
	\item $\mathbb{N}$ n'est pas major\'ee
\end{enumerate}

\section{r\'ecurrence}
\begin{enumerate}
	\item r\'ecurrence simple:
		\begin{enumerate}
			\item $P(0)$ est vrai
			\item $\forall n \in \mathbb{N}, P(n) \Rightarrow P(n+1)$
		\end{enumerate}
	\item r\'ecurrence double:
		\begin{enumerate}
			\item $P(0)$ et $P(1)$ vrai
			\item $\forall n \in \mathbb{N}, (P(n)$ et $P(n+1)) \Rightarrow P(n+2)$
		\end{enumerate}
	\item r\'ecurrence forte:
		\begin{enumerate}
			\item $P(0)$ est vrai
			\item $\forall n \in \mathbb{N}, (\forall k \in [\![0,n]\!], P(k)) \Rightarrow P(n+1)$
		\end{enumerate}
\end{enumerate}

\section{Propri\'et\'ees}
\begin{theo} D\'ecomposition en nombres premiers \\
		Soit $n \in \mathbb{N}^*$, alors $n$ s'\'ecrit de fa\c{c}on unique:
		\[n = p_i \ldots p_r\] (avec $r \in \mathbb{N}, p_i$ premier $(1 \leq i \leq r)$)
\end{theo}
\begin{coro}
	Tout entier $n \in \mathbb{N}^*$ s'\'ecrit de fa\c{c}on unique:
	\[n = p_1^{\alpha_1} \ldots p_r^{\alpha_r}\] (avec $r \in \mathbb{N}$,
	$p_1 \ldots p_r$ premier distincts, $\alpha_1 \ldots \alpha_r \in {N}^*$)
\end{coro}

\begin{defi} Valuation p-adique \\
	Pour $n \in \mathbb{N}^*$ on pose $V_p(n)$ \\
		\[n = \prod_{p \in \mathcal{P}} p^{V_p(n)}\]
\end{defi}
\begin{prop} $ $
	\begin{itemize}
		\item [-] $V_p(nm) = V_p(n) + V_p(m)$
		\item [-] $n|m \Leftrightarrow \forall p \in \mathcal{P}, V_p(n) \leq V_p(m)$
	\end{itemize}
\end{prop}

\begin{defi} Division euclidienne dans $\mathbb{N}$ \\
	Soit $a \in \mathbb{N}, b \in \mathbb{N}^*$, il existe un unique couple $(a,b) \in \mathbb{N}^2$ tel que:
	\begin{enumerate}
		\item $a = bq + r$
		\item $r < b$
	\end{enumerate}
\end{defi}

\end{document}
