\documentclass[fleqn]{article}
\usepackage{amssymb}
\usepackage{amsmath}
\usepackage{amsthm}
\usepackage{verbatim}

\title{Continuit\'e}
\date{}

\theoremstyle{definition} \newtheorem*{defi}{D\'efinition}
\theoremstyle{definition} \newtheorem*{theo}{Th\'eor\`eme}
\theoremstyle{definition} \newtheorem*{coro}{Corollaire}
\theoremstyle{remark} \newtheorem*{rqs}{Remarques}

\begin{document}
\maketitle

\section{D\'efinitions}
\begin{defi}
	Soit $D \subset \mathbb{R}, f:D \rightarrow \mathbb{R}.\ x_0 \in D$. LASSE:
	\begin{enumerate}
		\item $\forall\ V \in V_\mathbb{R}(f(x_0)), \exists U \in V_\mathbb{R}(x_0), f(U \cap D) \subset V$ [def. topologique]
		\item $\forall \epsilon>0, \exists \alpha > 0, \forall x \in D, |x-x_0| \leq \alpha \Rightarrow |f(x) - f(x_0)| \leq \epsilon$
		\item Pour toute suite $(u_n)$ de points de $D$ qui converge vers $x_0$, la suite $(f(u_n))_{n\in\mathbb{N}}$ converge vers $f(x_0)$
		[def. s\'equentielle]
	\end{enumerate}
	On dit que $f$ est continue sur $D$ si pour tout $x_0 \in D$, f est continue en $x_0$
\end{defi}
\begin{enumerate}
	\item Fonctions lipschitziennes
		\begin{defi} $f:D \rightarrow \mathbb{R}$ est lipschitzienne s'il existe $k \in \mathbb{R}_+$ tel que:
			\[\forall x,y \in D, |f(x) - f(y)| \leq k|x-y|\]
		\end{defi}
	\item Fonctions uniform\'ement continues
		\begin{defi} $f:D \rightarrow \mathbb{R}$ est UC si:
		\[\forall \epsilon >0, \exists \alpha>0, \forall x,y \in D, |x-y| \leq \alpha \Rightarrow |f(x) - f(y)| \leq \epsilon\]
	\end{defi}
\end{enumerate}
\begin{theo} Heine: \\
	Soit $D \subset \mathbb{R}$ tel que $D$ ferm\'e, born\'e. Si $f:D \rightarrow \mathbb{R}$ est continue, alors $f$ est UC
\end{theo}
\begin{rqs} $ $
	\begin{enumerate}
		\item Soit $f: D \rightarrow \mathbb{R}$ continue en un point $x_0 \in D$. Si $a, b \in \mathbb{R}$ sont tels que $a < f(x_0) < b$
			alors: \\
			$ \exists \alpha > 0$ tel que $\forall x \in D \cap [x_0 - \alpha, x_0 + \alpha], a < f(x) < b$
		\item en particulier: \\
			Si $f(x_0) > 0$, soit $m \in ]0, f(x_0)[$ alors il existe $\alpha >0$ tel que:
			$\forall x \in D \cap [x_0 - \alpha, x_0 + \alpha], f(x) \geq m > 0$ \\
			$f$ est minor\'ee par une constante $m>0$ au voisinage de $x_0$ (analogue pour $f(x_0) < 0$)
	\end{enumerate}
\end{rqs}

\section{Th\'eor\`emes fondamentaux}
\begin{theo}
	Image d'un ferm\'e born\'e par une application continue: \\
	Soit $D \subset \mathbb{R}$, on suppose $D$ ferm\'e et born\'e, $f:D \rightarrow \mathbb{R}$ continue. Alors $f(D)$ est ferm\'e et born\'e
	\begin{coro}
		$f$ est born\'ee et atteint ses bornes
	\end{coro}
\end{theo}
\begin{theo}
	Valeurs interm\'ediaires: \\
	Soit $I$ un intervalle non vide de $\mathbb{R}$ et $f:I \rightarrow \mathbb{R}$ continue:
	\begin{enumerate}
		\item alors $f(I)$ est un intervalle de $\mathbb{R}$
		\item on suppose qu'il existent $a,b \in I, f(a) \geq 0$ et $f(b) \leq 0$ alors il exsite $c \in I$ tel que: $f(c) = 0$
		\item si $f$ prend les valeurs $\alpha \leq \beta$, alors $f$ prend toute valeur interm\'ediaire entre $\alpha$ et $\beta$
	\end{enumerate}
\end{theo}
\begin{theo}
	L'image d'un segment par une application continue est un segment
\end{theo}
\begin{theo}
	Soit $I$ un intervalle non trivial de $\mathbb{R}, f:I \rightarrow \mathbb{R}$ continue et injective, alors $f$ est strictement monotone
\end{theo}

\end{document}
