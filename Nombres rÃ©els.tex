\documentclass[fleqn]{article}
\usepackage{amssymb}
\usepackage{amsmath}

\title{Entiers naturels}
\date{}

\begin{document}
\maketitle

\section{Valeur absolue}
\begin{itemize}
	\item $|x| \leq x \leq a \Leftrightarrow a$ et $-x \leq a$
	\item $|xy| = |x||y|$
	\item In\'egalit\'es triangulaires: \\
		$|x+y| \leq |x| + |y|$ \\
		$|x+y| \geq ||x| - |y||$ \\
		$|x-y| \geq ||x| - |y||$
\end{itemize}

\section{Intervalles de $\mathbb{R}$}
Soit $A$ una partie non vide de $\mathbb{R}$. LASSE,
\begin{enumerate}
	\item $A$ est un intervalle
	\item $\forall x,y \in A, x \leq y \Rightarrow [x,y] \subset A$ ($A$ est convexe)
\end{enumerate}

\section{Partie entière, propri\'et\'e d'Archim\`ede, densit\'e de $\mathbb{Q}$ dans $\mathbb{R}$}
\begin{itemize}
	\item $\mathbb{N}$ n'est pas major\'ee dans $\mathbb{R}$
	\item Propri\'et\'e d'Archim\`ede: soit $x \in \mathbb{R}, \epsilon \in \mathbb{R}_+^*$, alors: \\
		Il existe $n \in \mathbb{N}, n\epsilon>x$
	\item Partie enti\`ere: soit $x \in \mathbb{R}$ alors, \\
		Il existe un unique $n \in \mathbb{Z}, n \leq x < n+1$ \\
		$n$ s'appelle la partie enti\`ere de $x$ et se note $E(x)$
	\item Frac$(x) = x - E(x)$
	\item Densit\'e de $\mathbb{Q}$ dans $\mathbb{R}$, LASSE:
		D\'efinition: Soit $A \subset \mathbb{R}$,
		\begin{enumerate}
			\item $\forall x \in \mathbb{R}, \forall \epsilon > 0, \exists a \in A, |x-a| \leq \epsilon$
			\item $\forall \alpha,\beta \in \mathbb{R}, (\alpha < \beta) \Rightarrow A \cap ]\alpha, \beta[ \neq \emptyset$
		\end{enumerate}
		Th\'eor\`eme: $\mathbb{Q}$ est dense dans $\mathbb{R}$ \\
		Corollaire: $\mathbb{R}/\mathbb{Q}$ est dense dans $\mathbb{R}$
\end{itemize}


\end{document}
