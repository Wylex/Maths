%Préamble {{{1
\documentclass[fleqn]{article}

\usepackage{amssymb}
\usepackage{amsmath}
\usepackage{amsthm}
\usepackage{verbatim}
\usepackage{booktabs}

\theoremstyle{definition} \newtheorem*{defi}{D\'efinition}
\theoremstyle{definition} \newtheorem*{theo}{Th\'eor\`eme}
\theoremstyle{definition} \newtheorem*{coro}{Corollaire}
\theoremstyle{remark} \newtheorem*{rqs}{Remarques}
\theoremstyle{definition} \newtheorem*{prop}{Propri\'et\'e}

\title{Nombres r\'eels}
\date{}

\begin{document}
\maketitle

%Valeur absolue {{{1
\section{Valeur absolue}
\begin{prop} $x,a \in \mathbb{R}$
	\begin{itemize}
		\item [-] $|x| \leq a \Leftrightarrow (x \leq a$ et $-x \leq a)$
		\item [-] $|xy| = |x||y|$
	\end{itemize}
\end{prop}
\begin{prop} In\'egalit\'es triangulaires:
	\begin{itemize}
		\item [-] $|x+y| \leq |x| + |y|$
		\item [-] $|x+y| \geq \big||x| - |y|\big|$
		\item [-] $|x-y| \geq \big||x| - |y|\big|$
	\end{itemize}
\end{prop}

%Intervalles de R {{{1
\section{Intervalles de $\mathbb{R}$}
\begin{defi} Intervalle \\
	Soit $A$ una partie non vide de $\mathbb{R}$. LASSE,
	\begin{enumerate}
		\item $A$ est un intervalle
		\item $\forall x,y \in A, x \leq y \Rightarrow [x,y] \subset A$ ($A$ est convexe)
	\end{enumerate}
\end{defi}
\begin{prop}
	Pour $a \in \mathbb{R}, \epsilon > 0$,
	\[|x-a| \leq \epsilon \Leftrightarrow x \in [a - \epsilon, a + \epsilon]\]
	\[|x-a| < \epsilon \Leftrightarrow x \in ]a - \epsilon, a + \epsilon[\]
\end{prop}

%Partie entière, propriété d'Archiméde, densité de Q dans R {{{1
\section{Partie entière, propri\'et\'e d'Archim\`ede, densit\'e de $\mathbb{Q}$ dans $\mathbb{R}$}
\begin{prop}
	$\mathbb{N}$ n'est pas major\'ee dans $\mathbb{R}$
\end{prop}
\begin{prop} Archim\`ede \\
	Soit $x \in \mathbb{R}, \epsilon \in \mathbb{R}_+^*$, alors il existe $n \in \mathbb{N},\ n\epsilon>x$
\end{prop}
\begin{defi} Partie enti\`ere
	\begin{itemize}
		\item [-] Soit $x \in \mathbb{R}$ alors, il existe un unique $n \in \mathbb{Z}, n \leq x < n+1$ \\
			$n$ s'appelle la partie enti\`ere de $x$ et se note $E(x)$
		\item [-] Frac$(x) = x - E(x)$
	\end{itemize}
\end{defi}
\begin{defi} Densit\'e \\
	Soit $A \subset \mathbb{R}$, $A$ est dense dans $\mathbb{R}$ si:
	\begin{enumerate}
		\item $\forall x \in \mathbb{R}, \forall \epsilon > 0, \exists\ a \in A,\ |x-a| \leq \epsilon$
		\item $\forall\ \alpha,\beta \in \mathbb{R},\ \alpha < \beta \Rightarrow A \cap ]\alpha, \beta[ \neq \emptyset$
	\end{enumerate}
	\begin{theo} $\mathbb{Q}$ est dense dans $\mathbb{R}$ \end{theo}
	\begin{coro} $\mathbb{R}/\mathbb{Q}$ est dense dans $\mathbb{R}$ \end{coro}
\end{defi}


\end{document}
