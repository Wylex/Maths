%Préamble {{{1
\documentclass[fleqn]{article}

\usepackage{amssymb}
\usepackage{amsmath}
\usepackage{amsthm}
\usepackage{verbatim}
\usepackage{booktabs}

\theoremstyle{definition} \newtheorem*{defi}{D\'efinition}
\theoremstyle{definition} \newtheorem*{theo}{Th\'eor\`eme}
\theoremstyle{definition} \newtheorem*{coro}{Corollaire}
\theoremstyle{remark} \newtheorem*{rqs}{Remarques}
\theoremstyle{definition} \newtheorem*{prop}{Propri\'et\'e}
\newcommand{\ra}[1]{\renewcommand{\arraystretch}{#1}}
\newcommand*{\bfrac}[2]{\genfrac{}{}{0pt}{}{#1}{#2}}
\ra{1.3}

\title{Nombres complexes}
\date{}

\begin{document}
\maketitle

%Ensemble C {{{1
\section{Ensemble $\mathbb{C}$}
Pour $z \in \mathbb{C}$,
\begin{enumerate}
	\item $z + \overline{z} = 2Re(z)$
	\item $z - \overline{z} = 2iIm(z)$
\end{enumerate}

\begin{prop}
	$z,z' \in \mathbb{C},\ zz' = 0 \Leftrightarrow z = 0$ ou $z' = 0$
\end{prop}

\subsection{Conjugaison}
$z,z' \in \mathbb{C}$
\begin{enumerate}
	\item $\overline{\overline{z}} = z$
	\item $\overline{z + z'} = \overline{z} + \overline{z'}$\\
		$\overline{zz'} = \overline{z} \overline{z'}$
	\item Pour tout $n \in \mathbb{Z}$, $\overline{z^n} = \overline{z}^n$
\end{enumerate}

\subsection{Module}
\begin{defi}
	Pour $z \in \mathbb{C},\ |z| = \sqrt{z\overline{z}} \in \mathbb{R}_+$ \\
	$(z\overline{z} = |z|^2 = Re^2(z) + Im^2(z))$
	\begin{rqs}
		Si $z$ est un r\'eel alors (valeur abs)$|z| = \sqrt{Re^2(z)} = \sqrt{z^2} = |z|$(module)
	\end{rqs}
\end{defi}

\begin{prop} $z,z' \in \mathbb{C}$
	\begin{enumerate}
		\item $|z| = 0 \Leftrightarrow z = 0$
		\item $|z| = |\overline{z}|$
		\item $|zz'| = |z||z'|$
		\item $n \in \mathbb{Z},\ |z^n| = |z|^n$
		\item $|z| = 1 \Leftrightarrow \frac{1}{z} = \overline{z}$ (tr\`es utile pour $\mathbb{U}_n$)
	\end{enumerate}
\end{prop}

\begin{prop} Soit $\omega \in \mathbb{C}$
	\[|Re(\omega)| \leq |\omega| \leq |Re(\omega)| + |Im(\omega)| \text{ et } |Im(\omega)| \leq |\omega| \leq |Re(\omega)| + |Im(\omega)|\]
\end{prop}

\begin{prop} `` Lemme magique `` \\
	$A \in \mathbb{R}, z \in \mathbb{C}$:
	\[|z| \leq A \Leftrightarrow \forall \omega \in \mathbb{C},\ Re(\omega z) \leq |\omega|A\]
\end{prop}

\begin{prop} $ $
	\begin{itemize}
		\item [-] Identit\'e de Lagrange \\
			$a,b,c,d \in \mathbb{R}$, $(ac - bd)^2 + (ad + bc)^2 = (a^2 + b^2)(c^2 + d^2)$
		\item [-] Cauchy-Schwarz \\
			$a,b,c,d \in \mathbb{R}$, $|ac + bd| \leq \sqrt{a^2 + b^2} \sqrt{c^2 + d^2}$
	\end{itemize}
\end{prop}

\begin{prop} In\'egalit\'e triangulaire
	\begin{itemize}
		\item [-] $z_1, z_2 \in \mathbb{C},\ \big||z_1| - |z_2|\big| \leq |z_1 \bfrac{+}{-} z_2| \leq |z_1| + |z_2|$ \\
			Cas \'egalit\'e: $\overline{z_1}z_2 \in \mathbb{R}_+$
		\item [-] G\'en\'eralisation: $|\sum z_k| \leq \sum |z_k|$ \\
			Cas \'egalit\'e: $\exists u \in \mathbb{C}^*, (t_1, \hdots, t_n) \in \mathbb{R}_+^n$ tels que $z_k = t_ku$
	\end{itemize}
\end{prop}

\subsection{Argument d'un complexe non nul}
Soit $z \in \mathbb{C}^*$. On appelle argument de $z$ et on note $\arg(z)$ l'ensemble des r\'eels $\theta$ tels que:
\[z = |z|(\cos \theta + i \sin \theta)\]
\begin{rqs}
	Si on conna\^it un argument de $z$ on les conna\^it tous (modulo $2\pi$)
\end{rqs}

\subsection{La notation $e^{i \theta}$}
\begin{defi}
	Pour tout r\'eel $\theta$, on note $e^{i\theta}$ le complexe $\cos \theta + i \sin \theta$\\
	On \'ecrit alors: $z = re^{i\theta}$ avec $r \in \mathbb{R},\ \theta \in \mathbb{R}$. C'est une \'ecriture tr\`es bien adapt\'ee aux produits
	et aux quotients
	\begin{rqs}
		$\overline{e^{i\theta}} = \cos \theta - i \sin \theta = e^{-i\theta}$
	\end{rqs}
\end{defi}
\begin{prop} $ $ \\
	$\arg (zz') = \arg z + \arg z'$\\
	$\arg \frac{z}{z'}  = \arg z  - \arg z'$
\end{prop}

\subsection{Exponentielle complexe}
Pour $z \in \mathbb{C}$ on pose $\exp z = e^{Re(z)}(\cos(Im(z)) + i\sin(Im(z)) =  e^{Re(z)} e^{iIm(z)}$
\begin{itemize}
	\item [-] $\exp (z + z') = \exp z \exp z'$
	\item [-] $\exp 0 = 1$
	\item [-] $|\exp z| = e^{Re(z)}$
	\item [-] $\exp z = 1 \Leftrightarrow z \in 2i\pi\mathbb{Z}$
\end{itemize}

%Équations algébriques dans C {{{1
\section{\'Equations alg\`ebriques dans $\mathbb{C}$}

\subsection{Racines n-i\`emes d'un complexe}
\begin{defi} Soit $z \in \mathbb{C}, n \in \mathbb{N}^*, \omega \in \mathbb{C}$ \\
	On dit que $\omega$ est une racine $n^e$ de $z$ si $\omega^n = z$
\end{defi}

\begin{theo} Soit $z \in \mathbb{C}^*, n \in \mathbb{N}^*,\ \theta \in arg(z)$. Alors $z$ admet $n$ racines distinctes:
	\[R_n(z) = \{\sqrt[n]{|z|}e^{i \frac{\theta + 2k\pi}{n}},\ k \in [\![0,n-1]\!] \}\]
\end{theo}

\begin{prop} Ensemble des racines $n^e$ de 1 \\
	Pour $n \in \mathbb{N}^*, R_n(1)$ se note $\mathbb{U}_n$:
	\[\mathbb{U}_n = \{z \in \mathbb{C},\ z^n = 1\} = \{e^{\frac{2ik\pi}{n}}, k \in [\![0,n-1]\!] \}\]
\end{prop}

\begin{rqs}
	Soit $z \in \mathbb{C}^*,\ n \in \mathbb{N}$ tel que $n \geq 2$. Soit $a \in \mathbb{C}^*$ telle que $a^n = z$ \\
	Alors $R_n(z) = \{ua,\ u \in \mathbb{U}_n\}$
\end{rqs}

\begin{defi} Racines primitives $n^e$ de 1 \\
	$u \in \mathbb{U}_n$. On dit que $u$ est racine $n^e$ primitive de 1 si $\mathbb{R}_n = \{1, u, u^2, \hdots, u^{n-1}\}$
\end{defi}

\subsection{R\'esolution dans $\mathbb{C}$ d'\'equations de degr\'e 2}
Soit $a \in \mathbb{C}^*,b,c \in \mathbb{C}$ et $P: z \in \mathbb{C} \rightarrow az^2 + bz + c$ \\
$\Delta = b^2 -4ac$. Soit $\delta$ une racine car\'ee de $\Delta$. Alors $P$ admet dans $\mathbb{C}$ deux racines qui sont: $r_1 =
\frac{-b+\delta}{2a}$ et $r_2 = \frac{-b-\delta}{2a}$

\subsubsection{Racines carr\'ees}
\begin{enumerate}
	\item Cas o\`u $z \in \mathbb{R}$: Si $z > 0$ alors $\sqrt{z}$ est racine carr\'ee. Si $z < 0$ alors $i\sqrt{-z}$
	\item Cas o\`u l'on conna\^it une \'ecriture trigo int\'eressante: $z = re^{i\theta}$ alors $\sqrt{r}e^{i\frac{\theta}{2}}$
	\item Cas g\'en\'eral: $z = a + ib$ \\
		$\omega$ est racine carr\'ee de $z \Leftrightarrow \omega^2 = z \Leftrightarrow \omega^2 = z$ et $|\omega|^2 = |z|$
\end{enumerate}

%Remarques {{{1
\section{Remarques}
\begin{enumerate}
	\item Penser \`a factoriser par le demi-angle
	\item Utiliser \'ecriture exponentielle
	\item Pour une somme, essayer de faire appara\^itre une somme g\'eom\'etrique
	\item Pour r\'esoudre des \'equations avec des complexes, penser \`a l'\'equation du cercle: $(x-h)^2 + (y-k)^2=r^2$ avec $h,k$ les
		coordonn\'ees du centre du cercle
\end{enumerate}

\end{document}
