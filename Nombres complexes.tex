\documentclass[fleqn]{article}
\usepackage{amssymb}
\usepackage{amsmath}
\usepackage{amsthm}
\usepackage{verbatim}
\usepackage{booktabs}

\title{Nombres complexes}
\date{}

\theoremstyle{definition} \newtheorem*{defi}{D\'efinition}
\theoremstyle{definition} \newtheorem*{theo}{Th\'eor\`eme}
\theoremstyle{definition} \newtheorem*{coro}{Corollaire}
\theoremstyle{remark} \newtheorem*{rqs}{Remarques}
\theoremstyle{definition} \newtheorem*{prop}{Propri\'et\'e}
\newcommand{\ra}[1]{\renewcommand{\arraystretch}{#1}}
\newcommand*{\bfrac}[2]{\genfrac{}{}{0pt}{}{#1}{#2}}
\ra{1.3}

\begin{document}
\maketitle

\section{Ensemble $\mathbb{C}$}
\begin{prop} Pour $z \in \mathbb{C}$,
	\begin{itemize}
		\item $z + \overline{z} = 2Re(z)$
		\item $z - \overline{z} = 2iIm(z)$
	\end{itemize}
\end{prop}

\begin{prop}
	$z,z' \in \mathbb{C},\ zz' = 0 \Leftrightarrow z = 0$ ou $z' = 0$
\end{prop}

\subsection{Conjugaison}
\begin{prop} $z,z' \in \mathbb{C}$
	\begin{itemize}
		\item $\overline{\overline{z}} = z$
		\item $\overline{z + z'} = \overline{z} + \overline{z'}$\\
			$\overline{zz'} = \overline{z} \overline{z'}$
		\item Pour tout $n \in \mathbb{Z}$, $\overline{z^n} = \overline{z}^n$
	\end{itemize}
\end{prop}

\subsection{Module}
\begin{defi}
	Pour $z \in \mathbb{C},\ |z| = \sqrt{z\overline{z}} \in \mathbb{R}_+$ \\
	$(z\overline{z} = Re^2(z) + Im^2(z))$
	\begin{rqs}
		Si $z$ est un r\'eel alors (valeur abs)$|z| = \sqrt{Re^2(z)} = \sqrt{z^2} = |z|$(module)
	\end{rqs}
\end{defi}


\begin{prop} $z,z' \in \mathbb{C}$
	\begin{itemize}
		\item $|z| = 0 \Leftrightarrow z = 0$
		\item $|z| = |\overline{z}|$
		\item $|zz'| = |z||z'|$
		\item $n \in \mathbb{Z},\ |z^n| = |z|^n$
	\end{itemize}
\end{prop}

\begin{prop} Identit\'e de Lagrange \\
	$a,b,c,d \in \mathbb{R}$, $(ac - bd)^2 + (ad + bc)^2 = (a^2 + b^2)(c^2 + d^2)$
\end{prop}

\begin{prop} Cauchy-Schwarz \\
	$a,b,c,d \in \mathbb{R}$, $|ac + bd| \leq \sqrt{a^2 + b^2} \sqrt{c^2 + d^2}$
\end{prop}

\begin{prop} In\'egalit\'e triangulaire \\
	$z_1, z_2 \in \mathbb{C},\ \big||z_1| - |z_2|\big| \leq |z_1 \bfrac{+}{-} z_2| \leq |z_1| + |z_2|$
\end{prop}

\subsection{Argument d'un complexe non nul}
\begin{defi}
	Soit $z \in \mathbb{C}^*$. On appelle argument de $z$ et on note $\arg(z)$ l'ensemble des r\'eels $\theta$ tels que:
	\[z = |z|(\cos \theta + i \sin \theta)\]
	\begin{rqs}
		Si on conna\^it un argument de $z$ on les conna\^it tous (modulo $2\pi$)
	\end{rqs}
\end{defi}

\subsection{La notation $e^{i \theta}$}
\begin{defi}
	Pour tout r\'eel $\theta$, on note $e^{i\theta}$ le complexe $\cos \theta + i \sin \theta$\\
	On \'ecrit alors: $z = re^{i\theta}$ avec $r \in \mathbb{R},\ \theta \in \mathbb{R}$. C'est une \'ecriture tr\`es bien adapt\'ee aux produits
	et aux quotients
	\begin{rqs}
		$\overline{e^{i\theta}} = \cos \theta - i \sin \theta$
	\end{rqs}
\end{defi}
\begin{prop} $ $ \\
	$\arg (zz') = \arg z + \arg z'$\\
	$\arg \frac{z}{z'}  = \arg z  - \arg z'$
\end{prop}

\end{document}
