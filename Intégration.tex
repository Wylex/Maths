\documentclass[fleqn]{article}
\usepackage{amssymb}
\usepackage{amsmath}
\usepackage{amsthm}
\usepackage{verbatim}
\usepackage{booktabs}

\title{Int\'egration}
\date{}

\theoremstyle{definition} \newtheorem*{defi}{D\'efinition}
\theoremstyle{definition} \newtheorem*{theo}{Th\'eor\`eme}
\theoremstyle{remark} \newtheorem*{rqs}{Remarques}
\theoremstyle{definition} \newtheorem*{prop}{Propri\'et\'e}
\newcommand{\ra}[1]{\renewcommand{\arraystretch}{#1}}
\ra{1.3}

\begin{document}
\maketitle

\section{Int\'egrales des fonctions en escalier}
\begin{defi} Subdivision d'un segment\\
	Soient $a, b\in \mathbb{R},\ a<b$\\
	Une subdivision du segment $[a,b]$ est une suite finie $(x_0, \hdots, x_n)$ avec: \\ $a=x_0 < x_1 < \hdots < x_{n-1} <  x_n = b$
\end{defi}

\begin{defi} Fonctions en escalier\\
	$f: [a,b] \rightarrow \mathbb{R}$ est dite en escalier sur $[a,b]$ s'il existe $\Gamma = (x_0, \hdots, x_n)$ une subdivision de $[a,b]$
	tel que $f$ est constante sur $]x_k, x_k+1[,\ (0 \leq k \leq n-1)$
\end{defi}

\begin{theo} $ $\\
	Soit $f \in E([a,b], \mathbb{R})$. Soit $\Gamma = (x_0, \hdots, x_n)$ une subdivision de $[a,b]$ adapt\'ee \`a $f$ et notons $\lambda_i$
	la valeur de $f$ sur $]x_i, x_{i+1}[$ \\
	Le nombre $\sum_{i = 0}^{n-1} \lambda_i (x_{i+1} - x_i)$ ne d\'epend pas de $\Gamma$. On l'appelle int\'egrale de $f$ et on le note $I(f)$
\end{theo}

\begin{prop} $ $
	\begin{enumerate}
		\item lin\'earit\'e: \\
			Soient $f,g \in E([a,b], \mathbb{R}),\ \alpha \in \mathbb{R}$ alors: $I(\alpha f + g) = \alpha I(f) + I(g)$
		\item Positivit\'e: \\
			Soit $f \in E([a,b], \mathbb{R})$ telle que $\forall x \in [a,b],\ f(x) \geq 0$ alors: $I(f) \geq 0$ \\

			Corollaires:
			\begin{enumerate}
				\item $f,g \in E([a,b], \mathbb{R})$ tel que $f \leq g$ alors $I(f) \leq I(g)$
				\item Soit $f \in E([a,b], \mathbb{R})$ alors $|I(f)| \leq I(|f|)$
			\end{enumerate}
		\item Chasles: \\
			Soit $f \in E([a,b], \mathbb{R})$ et $c \in ]a,b[$ alors $I(f) = I(f_{|[a,c]}) + I(f_{|[c,b]})$
	\end{enumerate}
\end{prop}

\section{Int\'egrale d'une fonction continue sur un segment}
\begin{theo} Densit\'e \\
	Soient $a,b \in \mathbb{R},\ a < b$. Soit $f \in C([a,b], \mathbb{R})$. Soit $\epsilon > 0$. \\
	Il existe alors $\varphi_{\epsilon}, \psi_{\epsilon} \in E([a,b], \mathbb{R})$ tels que $\varphi_{\epsilon} \leq f \leq \psi_{\epsilon}$
	et $\psi_{\epsilon} - \varphi_{\epsilon} \leq \epsilon$
\end{theo}

\begin{theo} Int\'egrale d'une fonction continue sur un segment \\
	Soit $a,b \in \mathbb{R},\ a<b$ et $f:[a,b] \rightarrow \mathbb{R}$, continue. On note $E_f^+$ l'ensemble des fonction $\psi$ en escalier
	tels que $\psi \geq f$ (resp. $E_f^-,\ \varphi$) \\
	$\Omega_+ = \{I(\psi) \backslash \psi \in E_f^+\} \subset \mathbb{R}$ \\
	$\Omega_- = \{I(\varphi) \backslash \varphi \in E_f^-\} \subset \mathbb{R}$ \\
	Alors $\Omega_+$ est minor\'ee, $\Omega_+$ est major\'ee et inf $\Omega_+ = \text{sup }\Omega_-$ \\
	Ce nombre est par d\'efinition l'int\'egrale de $f$ et se note $\int_a^b f$. On a donc\\
	$\forall \varphi \in E_f^-,\ I(\varphi) \leq \int_a^b f$\\
	$\forall \psi \in E_f^+,\ I(\psi) \geq \int_a^b f$
\end{theo}

\begin{rqs} M\^eme propri\'et\'ees que pour les fonction en escalier (lin\'earit\'e, positivit\'e, chasles)
\end{rqs}

\begin{prop} $f \in C([a,b], \mathbb{R}_+)$, $f \geq 0$ et $\int_a^b f = 0 \Leftrightarrow \forall t \in [a,b],\ f(t) = 0$
\end{prop}

\begin{theo} Fondamental \\
	Soit I un intervalle de $\mathbb{R},\ f:I\rightarrow \mathbb{R}$ continue:
	\begin{enumerate}
		\item $f$ admet des primitives sur I: \\
			Si $a \in I,\ x \in I \mapsto \int_a^x f(t)dt$ est une primitive de $f$
		\item Pour toute primitive $F$ de $f$ on a: \\
			$\forall a,b \in I,\ \int_a^b f = F(b) - F(a) = [F(t)]_a^b$
	\end{enumerate}
\end{theo}

\end{document}
