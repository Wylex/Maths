%Préamble {{{1
\documentclass[fleqn]{article}

\usepackage{amssymb}
\usepackage{amsmath}
\usepackage{amsthm}
\usepackage{verbatim}
\usepackage{booktabs}
\usepackage{mathrsfs}

\theoremstyle{definition} \newtheorem*{defi}{D\'efinition}
\theoremstyle{definition} \newtheorem*{theo}{Th\'eor\`eme}
\theoremstyle{definition} \newtheorem*{coro}{Corollaire}
\theoremstyle{remark} \newtheorem*{rqs}{Remarques}
\theoremstyle{definition} \newtheorem*{prop}{Propri\'et\'e}
\newcommand{\ra}[1]{\renewcommand{\arraystretch}{#1}}
\ra{1.3}

\title{Int\'egration}
\date{}

\begin{document}
\maketitle

%Intégrales de fonctions en escalier {{{1
\section{Int\'egrales de fonctions en escalier}
\begin{defi} $ $
\begin{enumerate}
	\item Subdivision d'un segment \\
		Soient $a, b\in \mathbb{R},\ a<b$. Une subdivision du segment $[a,b]$ est une suite finie $(x_0, \hdots, x_n)$ avec: $a=x_0 < x_1 < \hdots < x_{n-1} <  x_n = b$

	\item Fonctions en escalier\\
		$f: [a,b] \rightarrow \mathbb{R}$ est dite en escalier sur $[a,b]$ s'il existe $\Gamma = (x_0, \hdots, x_n)$ une subdivision de $[a,b]$
		tel que $f$ est constante sur $]x_k, x_{k+1}[\ (0 \leq k \leq n-1)$
\end{enumerate}
\end{defi}

\begin{theo} Int\'egrale des fonctions en escalier
	\begin{itemize}
	\item [-] Soit $f \in \mathcal{E}([a,b], \mathbb{R})$. Soit $\Gamma = (x_0, \hdots, x_n)$ une subdivision de $[a,b]$ adapt\'ee \`a $f$ et
		notons $\lambda_i$ la valeur de $f$ sur $]x_i, x_{i+1}[$ \\
		Le nombre $\sum_{i = 0}^{n-1} \lambda_i (x_{i+1} - x_i)$ ne d\'epend pas de $\Gamma$.
	\item [-] On l'appelle int\'egrale de $f$ et on le note $I(f)$
	\end{itemize}
\end{theo}

\begin{prop} $ $
	\begin{enumerate}
		\item lin\'earit\'e: \\
			Soient $f,g \in \mathcal{E}([a,b], \mathbb{R}),\ \alpha \in \mathbb{R}$ alors: $I(\alpha f + g) = \alpha I(f) + I(g)$
		\item Positivit\'e: \\
			Soit $f \in \mathcal{E}([a,b], \mathbb{R})$ telle que $\forall x \in [a,b],\ f(x) \geq 0$ alors: $I(f) \geq 0$
			\begin{enumerate}
				\item $f,g \in \mathcal{E}([a,b], \mathbb{R})$ tel que $f \leq g$ alors $I(f) \leq I(g)$
				\item Soit $f \in \mathcal{E}([a,b], \mathbb{R})$ alors $|I(f)| \leq I(|f|)$
			\end{enumerate}
		\item Chasles: \\
			Soit $f \in \mathcal{E}([a,b], \mathbb{R})$ et $c \in ]a,b[$ alors $I(f) = I(f_{|[a,c]}) + I(f_{|[c,b]})$
	\end{enumerate}
\end{prop}

%Intégrale d'une fonction continue sur un segment {{{1
\section{Int\'egrale d'une fonction continue sur un segment}
\begin{theo} Densit\'e \\
	Soient $a,b \in \mathbb{R},\ a < b$. Soit $f \in \mathscr{C}([a,b], \mathbb{R})$. Soit $\epsilon > 0$. \\
	Il existe alors $\varphi_{\epsilon}, \psi_{\epsilon} \in \mathcal{E}([a,b], \mathbb{R})$ tels que $\varphi_{\epsilon} \leq f
	\leq \psi_{\epsilon}$ et $\psi_{\epsilon} - \varphi_{\epsilon} \leq \epsilon$
\end{theo}

\begin{theo} Int\'egrale d'une fonction continue sur un segment
	\begin{itemize}
		\item [-] Soit $a,b \in \mathbb{R},\ a<b$ et $f:[a,b] \rightarrow \mathbb{R}$, continue. On note $\mathcal{E}_f^+$ l'ensemble des
		fonction $\psi$
			en escalier tels que $\psi \geq f$ (resp. $\mathcal{E}_f^-,\ \varphi$) \\
			$\Omega_+ = \{I(\psi) / \psi \in \mathcal{E}_f^+\} \subset \mathbb{R}$ \\
			$\Omega_- = \{I(\varphi) / \varphi \in \mathcal{E}_f^-\} \subset \mathbb{R}$ \\
			Alors $\Omega_+$ est minor\'ee, $\Omega_+$ est major\'ee et $\inf \Omega_+ = \sup \Omega_-$
		\item [-] Ce nombre est par d\'efinition l'int\'egrale de $f$ et se note $\int_a^b f$. On a donc\\
			$\forall \varphi \in \mathcal{E}_f^-,\ I(\varphi) \leq \int_a^b f$\\
			$\forall \psi \in \mathcal{E}_f^+,\ I(\psi) \geq \int_a^b f$
	\end{itemize}
\end{theo}

\begin{rqs} M\^eme propri\'et\'ees que pour les fonction en escalier (lin\'earit\'e, positivit\'e, chasles)
\end{rqs}

\begin{prop} $f \in \mathscr{C}([a,b], \mathbb{R}_+)$, $f \geq 0$ et $\int_a^b f = 0 \Leftrightarrow \forall t \in [a,b],\ f(t) = 0$
\end{prop}

\begin{theo} Fondamental \\
	Soit I un intervalle de $\mathbb{R},\ f:I\rightarrow \mathbb{R}$ continue:
	\begin{enumerate}
		\item Une primitive est une application $F: I \rightarrow \mathbb{R}$ d\'erivable telle que $F' = f$
		\item $f$ admet des primitives sur I: \\
			Si $a \in I,\ x \in I \mapsto \int_a^x f(t)dt$ est une primitive de $f$
		\item Pour toute primitive $F$ de $f$ on a: \\
			$\forall a,b \in I,\ \int_a^b f = F(b) - F(a) = [F(t)]_a^b$
	\end{enumerate}
\end{theo}

\begin{theo} Taylor avec reste int\'egral \\
	Soient $a,b \in \mathbb{R},\ n \in \mathbb{N}$. $f:[a,b] \rightarrow \mathbb{R}$ de classe $C^{n+1}$ alors:
	\[f(b) = T_{n,f,a}(b) + \int_a^b \frac{(b-t)^n}{n!} f^{(n+1)} (t)dt\]
	\begin{rqs}
		$TRI \Rightarrow ITL$
	\end{rqs}
\end{theo}

\begin{theo} Int\'egration par parties \\
	Soient $f,g: [a,b] \rightarrow \mathbb{R}$ de classe $C^1$. Alors
	\[\int_a^b f'g = [f(t)g(t)]_a^b - \int_a^b fg'\]
\end{theo}

\begin{theo} Changement de variable
	\begin{enumerate}
		\item $f:[a,b] \rightarrow \mathbb{R}$ continue, $\varphi:[\alpha, \beta] \rightarrow [\varphi(\alpha), \varphi(\beta)]$ de classe $C^1$:
			\[\int_\alpha^\beta \varphi '(t) f(\varphi(t)) dt = \int_{\varphi(\alpha)}^{\varphi(\beta)} f(u)du\]
		\item $\psi:[\alpha, \beta] \rightarrow [\psi(\alpha), \psi(\beta)]$ une bijection ($\psi ' \neq 0$) de de classe $C^1$:
			\[\int_\alpha^\beta f(\psi(t)) dt = \int_{\psi(\alpha)}^{\psi(\beta)} f(u)\frac{du}{\psi' \circ \psi^{-1} (u)}\]
	\end{enumerate}
\end{theo}

\begin{prop} Formule de Stirling
	\[n! \sim (\frac{n}{e})^n \sqrt{2\pi n}\]
\end{prop}

\subsection{Sommes de Darboux-Riemann}
\begin{defi}
	$a,b \in \mathbb{R},\ f:[a,b] \rightarrow \mathbb{R}\ C^0,\ \Gamma = (x_0 ,\hdots, x_n)$ une subdivision de $[a,b]$, $\omega = (y_0, \hdots, y_{n-1})$ un choix de
	points associ\'ee \`a $\Gamma$ \\
	On pose $S_{f,\Gamma, \omega} = \sum_{i=0}^{n-1} (x_{i+1} - x_i) f(y_i)$
\end{defi}
\begin{theo}
	Soit $\epsilon > 0,\ \exists \alpha > 0$ tel que pour toute subdivision de $[a,b]$ tel que $\delta(\Gamma) \leq \alpha$ alors
	pour tout choix de points associ\'e \`a $\Gamma$:
	\[|S_{f,\Gamma, \omega} - \int_a^b f| \leq \epsilon\]
\end{theo}

\begin{coro} Suites de Darboux-Riemann \\
	$f:[a,b] \rightarrow \mathbb{R},\ C^0,\ n \in \mathbb{N}^*$ \\
	$u_n = \frac{b-a}{n} \sum_{k= 0}^{n-1} f(a + k\frac{b-a}{n})$ \\
	$v_n = \frac{b-a}{n} \sum_{k= 1}^{n} f(a + k\frac{b-a}{n})$ \\
	Alors, $u_n, v_n \rightarrow \int_a^b f$
\end{coro}

\subsection{Cauchy-Schwarz}
\begin{theo} In\'egalit\'e de Cauchy-Schwarz \\
	Soient $a,b \in \mathbb{R}$, continues. Alors:
	\[\left|\int_a^b fg\right| \leq \left(\int_a^b f^2\right)^\frac{1}{2} \left(\int_a^b g^2\right)^\frac{1}{2}\]
	Si $f$ n'est pas nulle, il y \`a \'egalit\'e si et seulement si $g$ est proportinnielle \`a f
\end{theo}

%Fonctions continues par morceaux {{{1
\section{Fonctions continues par morceaux}
\begin{defi}
	$f:[a,b] \rightarrow \mathbb{R}$ est continue par morceaux s'il existe une subdivision de $[a,b]$ telle que $\forall i \in [\![0,n-1]\!]$
	\begin{enumerate}
		\item $f$ est continue sur $]x_i, x_{i+1}[$
		\item $f$ admet une limite finie \`a droite en $x_i$, gauche en $x_{i+1}$
	\end{enumerate}
\end{defi}

\begin{rqs}
	Le th\'eor\`eme de densit\'e reste valable et par cons\'equent les propri\'et\'ees aussi
\end{rqs}

%Approximation des intégrales {{{1
\section{Approximation des int\'egrales}
\begin{enumerate}
	\item M\'ethode des rectangles \\
		$f:[a,b] \rightarrow \mathbb{R}$ au moins $C^1,\ \Gamma_n$ une subdivision r\'eguli\`ere de $[a,b]$, $M$ majorant de $f'$ \\
		$\Delta_n = \int_\alpha^\beta f - \sum_{k=0}^{n-1} \frac{b-a}{n} f(a + k\frac{b-a}{n})$ \\
		$|\Delta_n| \leq \frac{M(b-a)^2}{2n} \in O(\frac{1}{n})$
	\item M\'ethode des rectangles m\'edians \\
		$f:[a,b] \rightarrow \mathbb{R},\ C^2,\ y_k = \frac{x_k + x_{k+1}}{2},$ $M$ majorant de $f''$\\
		$\Delta_n = \int_a^b f - \sum_{k=0}^{n-1} (x_{k+1} - x_k)f(y_k)$ \\
		$|\Delta_n| \leq \frac{M(b-a)^3}{24n^2} \in O(\frac{1}{n^2})$

\end{enumerate}

\end{document}
