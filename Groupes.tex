%Préamble {{{1
\documentclass[fleqn]{article}

\usepackage{amssymb}
\usepackage{amsmath}
\usepackage{amsthm}
\usepackage{verbatim}
\usepackage{booktabs}
\usepackage{mathrsfs}

\theoremstyle{definition} \newtheorem*{defi}{D\'efinition}
\theoremstyle{definition} \newtheorem*{theo}{Th\'eor\`eme}
\theoremstyle{definition} \newtheorem*{coro}{Corollaire}
\theoremstyle{remark} \newtheorem*{rqs}{Remarques}
\theoremstyle{definition} \newtheorem*{prop}{Propri\'et\'e}
\newcommand{\ra}[1]{\renewcommand{\arraystretch}{#1}}
\ra{1.3}

\title{Groupes}
\date{}

\begin{document}
\maketitle

%Généralités {{{1
\section{G\'en\'eralit\'es}
\begin{defi} LCI \\
Soit $F$ un ensemble. Une loi de composition interne sur $E$ est une application de $E\times E \rightarrow E$
\end{defi}

Soit $E$ un ensemble muni d'une LCI $*$:
\begin{enumerate}
	\item $*$ associative si: $a*(b*c) = (a*b)*c$
	\item $*$ est commutative si: $a*b = b*a$
	\item Un \'el\'ement $e \in E$ est neutre pour $*$ si: $\forall x \in E,\ x*e = e*x = x$
	\item On dit que $x$ est sym\'etrique (ou inversible) s'il existe $y \in E$ tel que: $x*y = y*x = e$
	\item $(E,*)$ est un mono\"ide si: $*$ est associative et admet un neutre
\end{enumerate}

\begin{prop}
	S'il existe un \'el\'ement neutre il en existe un et un seul
\end{prop}

\begin{prop} Inversibles dans un mono\"ide
	\begin{itemize}
		\item [-] Si $x$ est inversible il existe un unique $y \in E$ tel que $x*y = y*x = e$
		\item [-] $x^{-1}$ est inversible et $(x^{-1})^{-1} = x$
		\item [-] $x, x'$ inversibles $\Rightarrow$ $x*x'$ inversible et $(x*x')^{-1} = x'^{-1}*x^{-1}$
	\end{itemize}
\end{prop}

\begin{prop} Puissances \\
Soit $(E, *)$ un mono\"ide. On d\'efinit $x^n,\ n \in \mathbb{N}$ en posant:
	\begin{enumerate}
		\item $x^0 = e$
		\item $x^{k+1} = x^k * x$
	\end{enumerate}
	Propri\'et\'es:
	\begin{itemize}
		\item [-] $\forall m,n \in \mathbb{N},\ x^{m+n} = x^m * x^n$
		\item [-] $\forall m,n \in \mathbb{N}, (x^n)^m = x^{nm}$
		\item [-] Soient $x,y \in E$ tels que $x*y = y*x$ alors:
			\begin{enumerate}
				\item $\forall m,n \in \mathbb{N}, x^n * y^m = y^m * x^n$
				\item $\forall n \in \mathbb{N}, (x*y)^n = x^n * y^n$
			\end{enumerate}
		\item [-] Soit $x \in E$ inversible. Soit $n \in \mathbb{Z}_-^*$, alors $x^n = (x^{-1})^{-n}$\\
			$\forall n \in \mathbb{Z}, x^n$ est inversible d'inverse $x^{-n}$
	\end{itemize}
\end{prop}

\begin{theo} $(\frac{\mathbb{Z}}{n\mathbb{Z}}, \dot\times)$ inversibles
\[\overline{a} \text{ est inversible dans } (\frac{\mathbb{Z}}{n\mathbb{Z}}, \dot\times) \Leftrightarrow a \text{ et } n
	\text{ sont premiers entre eux}\]

\end{theo}

%Groupes {{{1
\section{Groupes}
\begin{defi}
	Un groupe est un mono\^ide dans lequel tout \'el\'ement est inversible. Si de plus la LCI est commutative, on parle de groupe commutatif
	ou ab\'elien.
\end{defi}

\begin{prop} Soit $(G, *)$ un groupe
	\begin{enumerate}
		\item $a \in G$ alors $x \in G \mapsto ax$ est bijective de $G$ dans $G$
		\item Si $G$ est commutatif fini de cardinal $n$, alors $\forall x \in G, x^n = e$
	\end{enumerate}
\end{prop}

%Sous Groupes {{{1
\section{Sous groupes}
\begin{defi} Soit $(G,*)$ un groupe, $H \subset G$.\\
	On dit que $H$ est un sous groupe lorsque:
	\begin{enumerate}
		\item $H \neq \emptyset$
		\item $H$ est stable par $*:\ \forall x,y \in H,\ x*y \in H$
		\item $H$ est stable par passage \`a l'inverse: $\forall x \in H,\ x^{-1} \in H$
	\end{enumerate}
\end{defi}

\begin{prop}
	Soit $(G, *)$ un groupe, $x \in G$, alors $\langle x\rangle = \{x^n,\ n \in \mathbb{Z}\}$ est un sous groupe de $G$
\end{prop}

\begin{theo}
Soit $H$ un sous groupe de $(\mathbb{Z}, +)$ \\
Il existe un et un seul $n \in \mathbb{N}$ tel que $H = n\mathbb{Z}$
\end{theo}

\begin{theo}
	Si $H$ est un sous groupe de $(\mathbb{R}, +)$ alors:
	\begin{enumerate}
		\item $\exists\ \alpha > 0$ tel que $H = \alpha \mathbb{Z}$
		\item $H$ est dense dans $\mathbb{R}$
	\end{enumerate}
	(1) et (2) s'excluent mutuellement
\end{theo}

\begin{prop} $(G, *)$ un groupe, $H,K$ deux sous groupes de $G$.\\
Alors $H \cap K$ est un sous groupe de $G$
\end{prop}

\subsection{Sous groupes engengr\'e par une partie}
\begin{defi} $(G, *)$ un groupe. Soit $S \subset G$
	\begin{itemize}
		\item [-] $gr(S)$ est l'intersection de tous les sous groupes $H$ de $G$ tels que $S \subset H$
		\item [-] $gr(S)$ est le sous groupe engendr\'e par G\\
			Si $gr(S) = G$ alors on dit que $S$ engendre $G$
	\end{itemize}
\end{defi}

\begin{prop} $gr(x) = \langle x \rangle$ \end{prop}

\begin{defi} Soit $(G, *)$ un groupe:
	\begin{itemize}
		\item [-] On dit que $G$ est monog\`ene s'il existe $x \in G$ tel que $G = gr(x) = \langle x \rangle$ \\
			$G$ monog\`ene $\Rightarrow G$ commutatif
		\item [-] $G$ est cyclique si $G$ est monog\`ene fini
	\end{itemize}
\end{defi}

\subsection{\'El\'ements de torsion}
\begin{defi}
$(G,*)$ un groupe, $x \in G$. \\
On dit que $x$ est un \'el\'ement de torsion s'il existe $m \in \mathbb{N}^{*}$ tel que $x^m = e$ \\
On d\'efinit alors l'ordre par:
	\[o(x) = \min\{k \in \mathbb{N}^* / x^k = e\}\]
\end{defi}

\begin{prop} $(G,*)$ un groupe, $x \in G$ un \'el\'ement de torsion
	\begin{enumerate}
		\item Pour $k \in \mathbb{Z},\ x^k = e \Leftrightarrow o(x) \text{ divise } k$
		\item $x$ de torsion $\Leftrightarrow gr(x)$ est fini, dans ce cas $o(x) = |gr(x)|$
		\item Si $G$ est fini:
			\begin{enumerate}
				\item $\forall x \in G,\ gr(x) \subset G \Rightarrow gr(x)$ est fini $\Rightarrow x$ est de torsion et $o(x) \leq |G|$
				\item $G$ cyclique $\Leftrightarrow \exists x \in G,\ o(x) = |G|$
			\end{enumerate}
	\end{enumerate}
\end{prop}

\begin{theo} Lagrange \\
	Soit $(G, *)$ un groupe fini, $H$ un sous groupe de $G$ alors $|H|$ divise $|G|$ \\
	Par cons\'equent, si $G$ est un groupe fini, $x\in G$ est \'el\'ement de torsion et $o(x)$ divise  $|G|$
\end{theo}

%Morphismes de groupe {{{1
\section{Morphismes de groupe}
\begin{defi}
$(G,.),\ (H, *)$ deux groupes, $f: G \rightarrow H$ \\
On dira que $f$ est un morphisme de groupes si:
\[\forall x,y \in G, f(x.y) = f(x) * f(y)\]
$f$ est un isomorphisme si $f$ est bijective
\end{defi}

\begin{prop} $f: (G,.) \rightarrow (H,*)$ un morphisme de groupe
	\begin{enumerate}
		\item $f(e_G) = e_H$
		\item Pour $x \in G,\ n \in \mathbb{Z},\ f(x^n) = (f(x))^n$
		\item Soit $G_1$ un sous groupe de $G$. Alors $f(G_1)$ est un sous groupe de $H$.\\ En particulier, $f(G)$ appel\'e image de $f$
			et se note $Im\ f$
		\item Soit $H_1$ un sous groupe de $H$. Alors $f^{-1}(H_1)$ est un sous groupe de $G$. \\ En particulier, $f^{-1}(\{e_H\})$ est
			appel\'e le noyau de $f$ et se note $Ker\ f$
		\begin{rqs} $f$ injective $\Leftrightarrow Ker\ f = \{e_G\}$ \\ $|G| = |Ker f| |Im f|$ \end{rqs}
		\item La compos\'e de deux morphismes est un morphisme.\\
			La r\'eciproque d'un isomorphisme est un isomorphisme.
	\end{enumerate}
\end{prop}

\end{document}

