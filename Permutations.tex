%Préamble {{{1
\documentclass[fleqn]{article}

\usepackage{amssymb}
\usepackage{amsmath}
\usepackage{amsthm}
\usepackage{verbatim}
\usepackage{booktabs}
\usepackage{mathrsfs}

\theoremstyle{definition} \newtheorem*{defi}{D\'efinition}
\theoremstyle{definition} \newtheorem*{theo}{Th\'eor\`eme}
\theoremstyle{definition} \newtheorem*{coro}{Corollaire}
\theoremstyle{definition} \newtheorem*{nota}{Notation}
\theoremstyle{definition} \newtheorem*{vocab}{Vocabulaire}
\theoremstyle{remark} \newtheorem*{rqs}{Remarques}
\theoremstyle{definition} \newtheorem*{prop}{Propri\'et\'e}
\newcommand{\ra}[1]{\renewcommand{\arraystretch}{#1}}
\newcommand*{\bfrac}[2]{\genfrac{}{}{0pt}{}{#1}{#2}}
\ra{1.3}

\title{Permutations}
\date{}

\begin{document}
\maketitle

%Généralitées {{{1
\section{G\'en\'eralit\'ees}
On note $S_n$ l'ensemble des permutations de $[\![1,n]\!]$. $(S_n, \circ)$ est un groupe.

\begin{theo}
	Tout groupe fini $G$ est isomorphe \`a un sous groupe de $S_n$.
\end{theo}

\begin{defi} Transposition\\
	Soit $n \geq 2,\ \sigma \in S_n$ est une transposition s'il existe $a,b \in [\![1,n]\!], a \neq b$ tel que $\sigma(a) = b,\ \sigma(b) = a$ et
	$\sigma(x) = x$ pour $x \neq \{a,b\}$. \\
	On note $\tau_{ab}$ la transpos\'e qui \'echange $a$ et $b$.
\end{defi}

\begin{theo}
	Soit $\sigma \in S_n$, alors $\sigma$ est produit de transpositions.
\end{theo}

%cycles dans Sn {{{1
\section{Cycles dans $S_n$}

\begin{defi} Cycle\\
	Soit $\sigma \in S_n$: on dit que $\sigma$ est un cycle s'il existe $p \in [\![2,n]\!],\ x_1, \hdots, x_p \in [\![1,n]\!]$ distincts avec
	$\sigma(x_1) = x_2,\ \sigma(x_2) = x_3, \hdots, \sigma(x_p) = x_1$ et $\sigma(x) = x$ pour \mbox{$x \notin \{x_1, \hdots, x_p\}$}
	\begin{rqs}
		Il y a $\sum_{p=2}^n \binom{n}{p}(p-1)!$ cycles dans $S_n$
	\end{rqs}
\end{defi}

\begin{prop}
	soit $p \in [\![2,n]\!], \sigma$ un p-cycle. Alors $\sigma$ est un \'el\'ement d'ordre $p$.
\end{prop}

\begin{prop}
	Soient $\sigma, \sigma'$ deux cycles de support $A$ et $B$ tels que $A \cap B = \emptyset$, alors $\sigma \circ \sigma' = \sigma'
	\circ \sigma$
\end{prop}

\begin{theo}
	Soit $\sigma \in S_n$, alors $\sigma$ s'\'ecrit comme produit commutatif de cycles dont les supports sont deux \`a deux disjoints
\end{theo}

\begin{prop} Ordre d'une permutation\\
	Soit $\sigma \in S_n,\ \sigma = \gamma_1 \circ \hdots \circ \gamma_r$, avec $\gamma_i$ cycle de longueur $p_i$ et de support $A_i$. Alors,
	l'ordre de $\sigma$ est $ppcm(p_1, \hdots, p_r)$
\end{prop}

%Signature d'une permutation {{{1
\section{Signature d'une permutation}
\begin{defi}
	Soit $\sigma \in S_n$ et $(i,j) \in [\![1,n]\!]^2$ tels que $i < j$. On dit que $\sigma$ pr\'esente une inversion en $(i,j)$ si $\sigma(i)
	> \sigma(j)$. On note $N(\sigma)$ \mbox{le nombre d'inversions de $\sigma$.}\\
	La signature de $\sigma$ est $\epsilon(\sigma) = (-1)^{N(\sigma)}$
\end{defi}

\begin{prop}
	La signature d'une transposition vaut $-1$, la signature d'un cycle de longueur $p$ vaut $(-1)^{p-1}$
\end{prop}

\begin{theo}
	Pour $\sigma, \tau \in S_n,\ \epsilon(\sigma \circ \tau) = \epsilon(\sigma) \epsilon(\tau)$
\end{theo}

\end{document}
