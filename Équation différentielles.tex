\documentclass[fleqn]{article}
\usepackage{amssymb}
\usepackage{amsmath}
\usepackage{amsthm}
\usepackage{verbatim}
\usepackage{booktabs}

\title{\'Equations diff\'erentielles}
\date{}

\theoremstyle{definition} \newtheorem*{defi}{D\'efinition}
\theoremstyle{definition} \newtheorem*{theo}{Th\'eor\`eme}
\theoremstyle{remark} \newtheorem*{rqs}{Remarques}
\newcommand{\ra}[1]{\renewcommand{\arraystretch}{#1}}
\ra{1.3}

\begin{document}
\maketitle

\section{G\'en\'eralit\'es}
Une \'equation diff\'erenielle est une \'ecriure du type:
\[F(t, y, y', \hdots, y^{(n)}) = 0\]
F une application de $I \times \Omega$ dans $\mathbb{R}$ avec I intervalle de $\mathbb{R}$, $\Omega$ partie de $\mathbb{R}^{n+1}$\\
Une solution est par d\'efinition une application continue $f:J\rightarrow \mathbb{R}$, n fois d\'erivable telle que:
\begin{enumerate}
	\item $\forall t \in J, (f(t), f'(t), \hdots, f^{(n)}(t)) \in \Omega$
	\item $\forall t \in J, F(t, f(t), \hdots, f^{(n)}(t)) = 0$
\end{enumerate}

\begin{rqs} $ $
	\begin{itemize}
		\item Il n'y a pas toujours de solution
		\item S'il y-en a, elle ne sont pas forcement d\'efinies sur tout l'intervalle
		\item S'il y-en a, en g\'eneral on ne peut pas les exprimer au moyen des fonctions usuelles.
	\end{itemize}
\end{rqs}

\section{\'Equations diff\'erentielles lin\'eaires}
Ce sont des \'equations du type:
\[a_0(x)y + a_1(x)y' + \hdots + a_n(x)y^{(n)} = b(x)\]
Avec $a_0, \hdots, a_n, b$ des fonctions d\'efinies sur I et continues.\\

\begin{theo} Principe de superposition \\
	On suppose que b s'\'ecrit $b = b_1 + \hdots + b_p (p \in \mathbb{N}^{*},\ b_i : I \rightarrow \mathbb{R}$,  continues)\\
	On consid\`ere $(E_i): a_0(x)y + \hdots + a_n(x) y^{(n)} = b_i$ \\
	Si $\varphi_i$ est solution de $E_i$ alors $\varphi = \varphi_1 + \hdots + \varphi_p$ est solution de E
\end{theo}

\begin{theo} $ $\\
	Supposons que $u$, solution de $(E_0)$, ne s'annule jamais. Alors, toute fonction $f: I \rightarrow \mathbb{R}$ s'\'ecrit:\\
	$f = \lambda u$ avec $\lambda \in D^n(I, \mathbb{R})$ \\
	$\lambda u$ solution de (E) $\Leftrightarrow \lambda '$ est solution d'une EDL d'ordre $\leq n-1$
\end{theo}

\subsection{EDL d'ordre 1}
Soit une \'equation du type (E): $\alpha(x)y' + \beta(x)y = \gamma(x)$

\begin{enumerate}
	\item Cas sympathique: (E) est r\'esolue en y' ($\alpha$ ne s'annule pas)
		\[y' + \frac{\beta(x)}{\alpha(x)}y = \frac{\gamma(x)}{\alpha(x)}\]
		Donnons nous $a, b : I \rightarrow \mathbb{R}$ continues:
		\[y' + a(x)y = b(x)\]
		\begin{enumerate}
			\item Solutions de $(E_0)$ (\'equation diff\'erentielle homog\`ene)\\
				Soit A une primitive de a sur I, $f: I \rightarrow \mathbb{R}$ d\'erivable\\
				f est solution de $(E_0) \Leftrightarrow \exists \lambda \in \mathbb{R}, \forall t \in I, f(t) = \lambda e^{-A(t)}$
			\item Solution de (E) (solution particuli\`ere)
				\begin{align*}
					\text{f solution de} (E) &\Leftrightarrow \underset{f'}{(\lambda'(x) e^{-A} - a \lambda(x) e^{-A})} +
					\underset{af}{a \lambda(x) e^{-A}} = b\\
					&\Leftrightarrow \lambda'(x) e^{-A} = b \\
					&\Leftrightarrow \lambda'(x) = be^A
				\end{align*}
				En notant B une primitive de $be^A$ on voit que $Be^A$ est solution de E. On d\'eduit que:\\
				\[S = Be^{-A} + S_0 = \{t\in I \mapsto (\lambda + B(t))e^{-A(t)}/ \lambda \in \mathbb{R}\}\]
		\end{enumerate}
		\begin{theo} Cauchy-Lipschitz\\
			Soit $t_0 \in I, y_0 \in \mathbb{R}$ \\
			Alors il existe une et une seule solution f de (E) sut I telle que $f(t_0) = y_0$
		\end{theo}
	\item Cas g\'en\'eral
		\begin{itemize}
			\item Dans ce cas on ne peut rien affirmer de g\'en\'eral sur (E)
			\item Le cas plus simple est celui o\`u $\alpha$ s'annule en un seul point $x_0 \in Int(I)$\\
				$J_+ = I \cap ]x_0, +\infty[$\\
				$J_- = I \cap ]-\infty, x_0[$
			\item $\alpha$ est continue sur $J_+$ et $J_-$ et ne s'annule pas. On sait donc trouver les solutions (E) sur ces intervalles.
			\item Pour trouver les solutions sur I: nous avons $\beta(x_0)f(x_0) = \gamma(x_0)$ ce qui peut imposer certaines conditions. Il faut
				ainsi tenter de raccorder ces solutions en $x_0$:
				\begin{itemize}
					\item f doit \^etre continue en $x_0$ donc $lim_{x_0^-} f = f(x_0) = lim_{x_0^+}$
					\item f doit \^etre d\'erivable en $x_0$ donc $lim_{x_0^-} \frac{f(x) - f(x_0)}{x - x_0} = lim_{x_0^+} \frac{f(x) -
						f(x_0)}{x - x_0} = l \in \mathbb{R}$
				\end{itemize}
		\end{itemize}

\end{enumerate}

\end{document}
