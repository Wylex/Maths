\documentclass[fleqn]{article}
\usepackage{amssymb}
\usepackage{amsmath}
\usepackage{amsthm}
\usepackage{verbatim}
\usepackage{booktabs}
\usepackage{mathrsfs}

\title{D\'enombrement}
\date{}

\theoremstyle{definition} \newtheorem*{defi}{D\'efinition}
\theoremstyle{definition} \newtheorem*{theo}{Th\'eor\`eme}
\theoremstyle{definition} \newtheorem*{coro}{Corollaire}
\theoremstyle{remark} \newtheorem*{rqs}{Remarques}
\theoremstyle{definition} \newtheorem*{prop}{Propri\'et\'e}
\newcommand{\ra}[1]{\renewcommand{\arraystretch}{#1}}
\ra{1.3}

\begin{document}
\maketitle

\section{Ensembles finis}
\begin{defi}
	Soit $E$ un ensemble. $E$ est dit fini s'il existe $n \in \mathbb{N}$ tel que $E$ est en bijection avec $[\![1,n]\!]$ \\
	Pour $n, p \in \mathbb{N},\ [\![1,n]\!]$ est bijection avec $[\![1,p]\!] \Leftrightarrow n = p$
\end{defi}

\begin{prop} Soit $E$ un ensemble fini, $F \subset E$
	\begin{enumerate}
		\item $F$ est fini et card $F$ $\leq$ card $E$
		\item si card $F$ $=$ card $E$ alors $F$ = $E$
	\end{enumerate}
\end{prop}

\section{Op\'erations sur les enmbles finis}
\subsection{R\'eunion}
\begin{itemize}
	\item [-] Soient $E, F$ deux ensembles finis tels que $E \cap F = \emptyset$ \\
		Alors $E \cup F$ est fini et $|E \cup F| = |E| + |F|$
	\item [-] Pour toute partie $A$ de $E$, $|A| = |E| - |\overline{A}|$
\end{itemize}
Soient $E, F$ deux ensembles finis \\
Alors $E\cup F$ est fini et $|E \cup F| = |E| + |F| - |E \cap F|$

\subsection{Ensemble des parties d'un ensemble fini}
Soit E un ensemble fini, alors $P(E)$ est fini et $|P(E)| = 2^{|E|}$ \\
$\gamma_k^n = \binom{n}{k}$ c'est le nombre de parties \`a $k$ \`el\`ements d'un ensemble fini \`a n \`el\`ements

\subsection{Produit cart\'esien}
Soient $E,F$ deux ensembles finis, alors $E\times F$ est fini et $|E\times F| = |E||F|$

\subsection{Applications}
$E,F$ deux ensembles finis, alors $\mathcal{F}(E,F)$ est fini et $|\mathcal{F}(E,F)| = |F|^{|E|}$

\section{Applications entre ensembles finis}
Soit $E$ un ensemble fini, $F$ quelconque et $f: E \rightarrow F$
\begin{enumerate}
	\item $f(E)$ est fini et $|f(E)| \leq |E|$
	\item $|f(E)| = |E| \Leftrightarrow f$ est injective
\end{enumerate}

Supposons que $F$ est fini
\begin{enumerate}
	\item $f(E) \leq |F|$
	\item $|f(E)| = |F| \Leftrightarrow f$ est surjective
\end{enumerate}

\begin{itemize}
	\item [-] Si $|E| > |F|$ alors $f$ ne peut pas \^etre injective
	\item [-] Si $|E| < |F|$ alors $f$ ne peut pas \^etre surjective
	\item [-] $E$ et $F$ sont en bijection $\Leftrightarrow |E| = |F|$
\end{itemize}

\begin{prop} $f: E \rightarrow F$ avec $|E| = |F|$, LASSE
	\begin{enumerate}
		\item $f$ est injective
		\item $f$ est surjective
		\item $f$ est bijective
	\end{enumerate}
\end{prop}

\subsection{Nombre d'injections, de permutations}
\begin{prop} $ $
	\begin{enumerate}
		\item $A_p^n = \frac{p!}{(p-n)!}$
		\item $|S(E)| = n!$
	\end{enumerate}
\end{prop}

\end{document}
