%Préamble {{{1
\documentclass[fleqn]{article}

\usepackage{amssymb}
\usepackage{amsmath}
\usepackage{amsthm}
\usepackage{verbatim}
\usepackage{booktabs}
\usepackage{mathrsfs}

\theoremstyle{definition} \newtheorem*{defi}{D\'efinition}
\theoremstyle{definition} \newtheorem*{theo}{Th\'eor\`eme}
\theoremstyle{definition} \newtheorem*{coro}{Corollaire}
\theoremstyle{remark} \newtheorem*{rqs}{Remarques}
\theoremstyle{definition} \newtheorem*{prop}{Propri\'et\'e}
\newcommand{\ra}[1]{\renewcommand{\arraystretch}{#1}}
\ra{1.3}

\title{D\'enombrement}
\date{}

\begin{document}
\maketitle

%Ensembles finis {{{1
\section{Ensembles finis}
\begin{defi}
	Soit $E$ un ensemble. $E$ est dit fini s'il existe $n \in \mathbb{N}$ tel que $E$ est en bijection avec $[\![1,n]\!]$ \\
	Pour $n, p \in \mathbb{N},\ [\![1,n]\!]$ est bijection avec $[\![1,p]\!] \Leftrightarrow n = p$
\end{defi}

\begin{prop} Soit $E$ un ensemble fini, $F \subset E$
	\begin{enumerate}
		\item $F$ est fini et $|F|$ $\leq$ $|E|$
		\item si $|F|$ $=$ $|E|$ alors $F$ = $E$
	\end{enumerate}
\end{prop}

%Opérations sur les ensembles finis {{{1
\section{Op\'erations sur les ensembles finis}
\begin{prop} Soient $E$ et $F$ deux ensembles finis
	\begin{enumerate}
		\item R\'eunion de deux ensembles
			\begin{enumerate}
				\item On suppose que $E \cap F = \emptyset$ \\
					Alors $E \cup F$ est fini et $|E \cup F| = |E| + |F|$
				\item Pour toute partie $A$ de $E$, $|A| = |E| - |\overline{A}|$
				\item Soient $E, F$ deux ensembles finis \\
					Alors $E\cup F$ est fini et $|E \cup F| = |E| + |F| - |E \cap F|$
			\end{enumerate}
		\item Ensemble des parties d'un ensemble fini
			\begin{enumerate}
			\item $P(E)$ est fini et $|P(E)| = 2^{|E|}$
			\item $\gamma_k^n = \binom{n}{k}$ c'est le nombre de parties \`a $k$ \`el\`ements d'un ensemble fini \`a n \`el\`ements
			\end{enumerate}
		\item Produit cart\'esien \\
			$E\times F$ est fini et $|E\times F| = |E||F|$
		\item Applications \\
			$\mathcal{F}(E,F)$ est fini et $|\mathcal{F}(E,F)| = |F|^{|E|}$
	\end{enumerate}
\end{prop}

%Applications entre ensembles finis {{{1
\section{Applications entre ensembles finis}
\begin{prop} $ $
	\begin{enumerate}
		\item Soit $E$ un ensemble fini, $F$ quelconque et $f: E \rightarrow F$
			\begin{enumerate}
				\item $f(E)$ est fini et $|f(E)| \leq |E|$
				\item $|f(E)| = |E| \Leftrightarrow f$ est injective
			\end{enumerate}

		\item Supposons que $F$ est fini
			\begin{enumerate}
				\item $|f(E)| \leq |F|$
				\item $|f(E)| = |F| \Leftrightarrow f$ est surjective
			\end{enumerate}
	\end{enumerate}
\end{prop}

\begin{coro} $ $
	\begin{itemize}
		\item [-] Si $|E| > |F|$ alors $f$ ne peut pas \^etre injective
		\item [-] Si $|E| < |F|$ alors $f$ ne peut pas \^etre surjective
		\item [-] $E$ et $F$ sont en bijection $\Leftrightarrow |E| = |F|$
	\end{itemize}
\end{coro}

\begin{prop} $f: E \rightarrow F$ avec $|E| = |F|$, LASSE
	\begin{enumerate}
		\item $f$ est injective
		\item $f$ est surjective
		\item $f$ est bijective
	\end{enumerate}
\end{prop}

\subsection{Nombre d'injections, de permutations}
\begin{enumerate}
	\item Nombre d'injections de $[\![1,n]\!]$ dans $[\![1,p]\!]$:
		\[A_p^n = \frac{p!}{(p-n)!}\]
	\item Nombre de permutations (bijections) ($p=n$):
		\[|S(E)| = n!\]
\end{enumerate}

%Remarques {{{1
\section{Remarques}
\begin{enumerate}
	\item Utiliser des ensembles compl\'ementaires
	\item D\'ecomposer en ensembles \`a $k$ \'el\'ements avec $\binom{n}{k}$ puis faire une somme pour revenir au premier ensemble
\end{enumerate}

\end{document}
