\documentclass[fleqn]{article}
\usepackage{amssymb}
\usepackage{amsmath}
\usepackage{amsthm}
\usepackage{verbatim}
\usepackage{booktabs}

\title{D\'eriv\'ees}
\date{}

\theoremstyle{definition} \newtheorem*{defi}{D\'efinition}
\theoremstyle{definition} \newtheorem*{theo}{Th\'eor\`eme}
\theoremstyle{remark} \newtheorem*{rqs}{Remarques}
\theoremstyle{definition} \newtheorem*{prop}{Propri\'et\'e}
\newcommand{\ra}[1]{\renewcommand{\arraystretch}{#1}}
\ra{1.3}

\begin{document}
\maketitle

\section{D\'efinition}
\begin{defi}
	Soit I un intervalle non trivial de $\mathbb{R},\ x_0 \in I, f:I \rightarrow \mathbb{R}$. On note $\Phi_{x_0, f}$ l'application
	\begin{align*}
		I\backslash\{x_0\} &\rightarrow \mathbb{R} \\
		x &\mapsto \frac{f(x) - f(x_0)}{x - x_0}
	\end{align*}
	On dit que $f$ est d\'erivable au point $x_0$ si $\Phi_{x_0, f}$ admet une limite finie en $x_0$
\end{defi}

\begin{rqs}
	$\frac{f(x) - f(x_0)}{x - x_0} \underset{x_0}{\rightarrow} l \Leftrightarrow \frac{f(x_0 + t) - f(x_0)}{t} \underset{0}{\rightarrow} l$
\end{rqs}

\begin{prop}
	$f$ d\'erivable en $x_0 \Rightarrow f$ continue en $x_0$
\end{prop}

\section{Th\'eor\`emes g\'en\'eraux}
\begin{theo}
	$f,g: I \rightarrow \mathbb{R}, x_0 \in I, \alpha \in \mathbb{R}$. On suppose $f$ et $g$ d\'erivables en $x_0$
	\begin{enumerate}
		\item $\alpha f + g$ est d\'erivable en $x_0$ et $(\alpha f + g)' = \alpha f' + g'$
		\item $fg$ est d\'erivable en $x_0$ et $(fg)' = f'g + fg'$
		\item On suppose que $g$ ne s'annule pas sur I:\\
			$\frac{1}{g}$ est d\'erivable en $x_0$ et $(\frac{1}{g})' = \frac{-g'}{g^2}$
	\end{enumerate}
\end{theo}

\begin{theo} Composition\\
	$I,J$ deux intervalles non triviaux de $\mathbb{R}$, $f:I\rightarrow J,\ g:J \rightarrow \mathbb{R}$ \\
	Si $f \in D(I, \mathbb{R}),\ g \in D(J, \mathbb{R})$ alors $g\circ f \in D(I, \mathbb{R})$ et $(g\circ f)' = f' \times g'\circ f$
\end{theo}

\begin{theo} D\'eriv\'ee d'une r\'eciproque\\
	Soit I un intervalle de $\mathbb{R}, x_0 \in I,\ f:I\rightarrow \mathbb{R}$ continue, strictement monotone. \\
	On sait que $f$ induit une bijection de I sur $J=f(I)$ dont on note $g$ la r\'eciproque. \\
	On suppose que $f$ est d\'erivable sur I et $f'$ ne s'annule pas. alors, $g$ est d\'erivable sur J et:
	\[\forall y \in J,\ g'(y) = \frac{1}{f'(g(y))}\]
\end{theo}

\section{Rolle, TAF}
\begin{theo} Rolle\\
	Soient $a, b \in \mathbb{R}$ tels que $a < b$, $f:[a,b] \rightarrow \mathbb{R}$ continue sur $[a,b]$, d\'erivable sur $]a,b[$ et telle que $f(a) = f(b)$. Alors il existe $c \in ]a,b[$
	tel que \[f'(c) = 0\]
\end{theo}

\begin{theo} TAF\\
	Soient $a, b \in \mathbb{R}$ tels que $a < b$,  $f:[a,b] \rightarrow \mathbb{R}$ continue sur $[a,b]$, d\'erivable sur $]a,b[$.
	Alors il existe $c \in ]a,b[$ tel que:
	\[f(b) - f(a) = (b-a)f'(c)\]
\end{theo}

\subsection{Cons\'equences du TAF}
\begin{theo} $ $\\
	Soit I un intervalle de $\mathbb{R}, f:I \rightarrow$ d\'erivable. Alors:
	\[f\text{ constante }\Leftrightarrow f' = 0\]
\end{theo}

\begin{theo} $ $ \\
	Soit I un intervalle de $\mathbb{R},\ f \in D(I, \mathbb{R})$
	\begin{enumerate}
		\item $f$ est croissante $\Leftrightarrow f' \geq 0$
		\item $f' > 0 \Rightarrow $ $f$ strictement croissante
		\item $f$ strictement croissante $\Leftrightarrow f' \geq 0$ et $Int(\{x\in I /f'(x) = 0\}) = \emptyset$
	\end{enumerate}
\end{theo}

\begin{theo} In\'egalit\'es des accroissements finis
	\begin{enumerate}
		\item $a<b, f:[a,b] \rightarrow \mathbb{R}$ continue sur $[a,b]$, d\'erivable sur $]a,b[$.\\
			On suppose $\exists m,M \in \mathbb{R}, \forall t \in ]a,b[, m \leq f'(t) \leq M$ Alors:
			\[m(b-a) \leq f(b) - f(a) \leq M(b-a)\]
		\item I un intervalle de $\mathbb{R}, f:I \rightarrow \mathbb{R}$ continue sur I, d\'erivable sur $Int(I)$.\\
			On suppose $\exists M\geq 0, \forall t \in Int(I), |f'(t)| \leq M$. Alors,
			\[\forall x,y \in I, |f(x) - f(y)| \leq M|x-y|\]
		\item $a,b \in \mathbb{R}, a<b. f,g:[a,b] \rightarrow \mathbb{R}$ continue sur $[a,b]$, d\'erivable sur $]a,b[$.\\
			On suppose $\forall t \in ]a,b[, |f'(t)| \leq g'(t)$ Alors,
			\[|f(b) - f(a)| \leq g(b) - g(a)\]
	\end{enumerate}
\end{theo}

\begin{theo} Limite de la d\'eriv\'ee \\
	Soit I un intervalle de $\mathbb{R}, x_0 \in I,\ f:I\rightarrow \mathbb{R}$ continue sur I, d\'erivable sur $I\backslash\{x_0\}$.\\
	On suppose que $f'(x) \underset{x \rightarrow x_0}{\rightarrow} l \in \mathbb{R}$ alors, $f$ est d\'erivable en $x_0$ et $f'(x_0) = l$
\end{theo}

\section{Ordre sup\'erieur}
\begin{defi}
	Soit I un intervalle de $\mathbb{R}, f:I \rightarrow \mathbb{R}, n \in \mathbb{N}^{*}$ \\
	On d\'efinit $f^{(n)}$ par:
	\begin{enumerate}
		\item si $n = 0,\ f^{(0)} = f$
		\item si $n \geq 1,\ f^{(n)}$ est d\'efinie si $f^{(n-1)}$ est bien d\'efinie et d\'erivable. Alors $f^{(n)} = (f^{(n-1)})'$
	\end{enumerate}
\end{defi}

\subsection{Th\'eor\`emes}
\begin{theo}
	Soit I un intervalle de $\mathbb{R}, n \in \mathbb{N}, f,g \in D^n(I, \mathbb{R})$ Alors:
	\begin{enumerate}
		\item Pour $\lambda \in \mathbb{R}$ \\
			$\lambda (f + g) \in D^n$ et $(\lambda f + g)^{(n)} = \lambda f^{(n)} + g^{(n)}$
		\item $fg \in D^n$ et $(fg)^{(n)} = \sum_{k=0}^{n} \binom{n}{k} f^{(k)} g ^{(n-k)}$
		\item On suppose de plus $g$ \`a valeurs non nulles. Alors: $\frac{f}{g} \in D^n$
	\end{enumerate}
\end{theo}

\begin{theo} Composition\\
	I,J deux intervalles de $\mathbb{R}, f \in D^n(I, J),\ g \in D^n(J, \mathbb{R})$\\
	Alors $g \circ f \in D^n(I, \mathbb{R})$
\end{theo}

\begin{theo} R\'eciproque\\
	Soit I un intervalle de $\mathbb{R}$, $f$ une application admettant une r\'eciproque g et telle que $f \in D^n$. Alors $g \in D^n$
\end{theo}

\begin{theo} Prolongement des fonctions de classe $C^n$\\
	Soit I un intervalle de $\mathbb{R}, n \in \mathbb{N}, x_0 \in I, f:I\backslash\{x_0\} \rightarrow \mathbb{R}$, de classe $C^n$ \\
	On suppose que $\forall k \in [\![0,n]\!], f^{(k)}$ admet en $x_0$ une limite finie $l_k$. Alors:
	\begin{align*}
		\tilde{f}: I &\rightarrow \mathbb{R}\\
		x_0 &\mapsto l_0\\
		x \neq x_0 &\mapsto f(x)
	\end{align*}
	est de classe $C^n$ sur I et $\forall k \in [\![0,n]\!],\ f^{(k)}(x_0) = l_k$
\end{theo}

\section{In\'egalit\'e de Taylor-Lagrange, Formule de Taylor-Young}
\begin{defi} Polyn\^omes de Taylor \\
	Soit I un intervalle de $\mathbb{R},\ x_0 \in I,\ f \in D^n(I, \mathbb{R})$. Pour $x \in \mathbb{R}$ on pose:
	\[T_{n,f,x_0}(x) = \sum_{k=0}^{n} \frac{f^{(k)}(x_0)}{k!}(x-x_0)^k\]
\end{defi}

\begin{theo} In\'egalit\'e de Taylor-Lagrange\\
	Soient $a,b \in \mathbb{R},\ f:[a,b] \rightarrow \mathbb{R}$ de classe $C^{n+1}$. Soit M un majorant de $|f^{(n+1)}|$ sur $[a,b]$. Alors:
	\[|f(b) - T_{n,f,a}(b)| \leq M\frac{|b-a|^{n+1}}{(n+1)!}\]
\end{theo}

\begin{theo} Formule de Taylor-Young\\
	Soit I un intervalle de $\mathbb{R},\ x_0 \in I, f:I\rightarrow \mathbb{R}$ de classe $C^n$. Alors:
	\[f(x) - T_{n,f,x_0}(x) \underset{x \rightarrow x_0}{=} o((x-x_0)^n)\]

	\begin{rqs}
		On peut utiliser ce th\'eor\`eme pour prouver qu'une fonction admet un DL
	\end{rqs}
\end{theo}

\end{document}
