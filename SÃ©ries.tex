%Préamble {{{1
\documentclass[fleqn]{article}

\usepackage{amssymb}
\usepackage{amsmath}
\usepackage{amsthm}
\usepackage{verbatim}
\usepackage{booktabs}
\usepackage{mathrsfs}

\theoremstyle{definition} \newtheorem*{defi}{D\'efinition}
\theoremstyle{definition} \newtheorem*{theo}{Th\'eor\`eme}
\theoremstyle{definition} \newtheorem*{coro}{Corollaire}
\theoremstyle{remark} \newtheorem*{rqs}{Remarques}
\theoremstyle{definition} \newtheorem*{prop}{Propri\'et\'e}
\newcommand{\ra}[1]{\renewcommand{\arraystretch}{#1}}
\ra{1.3}

\title{S\'eries}
\date{}

\begin{document}
\maketitle

%Définitions, premières propriétées {{{1
\section{D\'efinitions, premi\`eres propri\'et\'ees}

\begin{defi} Une s\'erie d'\'el\'ements de $\mathbb{K}$ est un couple $(u,S)$ o\`u $u = (u_n)_{n \in \mathbb{N}}$ est une suite d'\'el\'ements de
$\mathbb{K}$ et $S$ la suite d\'efinie par: $S_n = \sum_{k=0}^n u_k$\\
On dit que $S$ est la suite des sommes partielles de la s\'erie et $u$ le terme g\'en\'eral de la s\'erie. On note $\sum_{n \in \mathbb{N}} u_n$
cette s\'erie.
\end{defi}

\begin{defi} Convergence \\
Soit $u \in K^{\mathbb{N}}$. On dit que la s\'erie de terme g\'en\'eral $u$ converge lorsque la suite $S$ des sommes partielles de la s\'erie
est une suite convergente. En cas de convergence, la limite $L$ de la s\'erie $S$ s'appelle la somme de la s\'erie et se note
$\sum_{n=0}^{+\infty} u_n$
\end{defi}

\begin{defi} Suite des restes \\
	Soit $\sum_{n \in \mathbb{N}} u_n$ une s\'erie convergente de somme $L$. On note $(R_n)_{n \in \mathbb{N}}$ tel que $R_n = L - S_n$ la suite
	des restes de la s\'erie.\\
	On observe que $(R_n)$ v\'erifie toujours $R_n \rightarrow 0$

	\begin{rqs}
		$R_n = \sum_{k = n+1}^{+\infty} u_k$
	\end{rqs}
\end{defi}

\begin{prop} On consid\`ere une s\'erie $\sum_{n \in \mathbb{N}} u_n$
	\begin{enumerate}
		\item [-] $u_n = S_n - S_{n-1}$
		\item [-] $u_n = R_{n-1} - R_n$
	\end{enumerate}
\end{prop}

\begin{prop} Faits de base
	\begin{enumerate}
		\item [-] Soient $u,v \in \mathbb{K}^{\mathbb{N}},\ \alpha, \in K$. On suppose $\sum u_n$ et $\sum v_n$ convergentes:\\
			alors $\sum(\alpha u_n + v_n)$ converge et $\sum_{n=0}^{+\infty} (\alpha u_n + v_n) = \alpha \sum_{n=0}^{+\infty} u_n +
			\sum_{n=0}^{+\infty} v_n$
		\item [-] Soit $u \in \mathbb{K}^\mathbb{N}$. On suppose que $\sum u_n$ converge, alors $u$ converge vers $0$. Si $u$ ne converge
			pas vers $0$, on dit qu'il y a divergence grossi\`ere
		\item [-] $u \in \mathbb{K}^\mathbb{N}$ est convergente $\Leftrightarrow \sum_{n \in \mathbb{N}} (u_{n+1} - u_n)$ est convergente
	\end{enumerate}
\end{prop}

%Séries à termes positifs {{{1
\section{S\'eries \`a termes positifs}
\begin{defi}
Une s\'erie \`a termes positifs (SATP) est une s\'erie $\sum u_n$ o\`u $u$ est une suite r\'eelle positive (au moins \`a partir d'un
	certain rang).
\end{defi}

\begin{prop} Soit $u \in \mathbb{R}_{+}^{\mathbb{N}}$
	\begin{enumerate}
		\item [-] $\sum u_n$ convergente $\Leftrightarrow S$ major\'ee
		\item [-] $\sum u_n$ divergente $\Leftrightarrow S_n \rightarrow +\infty$
	\end{enumerate}
\end{prop}

\subsection{Principe de comparaison}
\begin{theo} Soient $u,b \in \mathbb{R}_+^{\mathbb{N}}$. On suppose $u_n \leq v_n$ APCR, alors:
\begin{enumerate}
	\item Si $\sum_{n \in \mathbb{N}} v_n < +\infty$ alors $\sum_{n \in \mathbb{N}} u_n < +\infty$
	\item Si $\sum_{n \in \mathbb{N}} u_n = +\infty$ alors $\sum_{n \in \mathbb{N}} v_n = +\infty$
\end{enumerate}
\end{theo}

\begin{coro} Soient $u,v \in \mathbb{R}_+^\mathbb{N}$
\begin{enumerate}
	\item [-] On suppose $u_n \underset{n \rightarrow +\infty}{=} O(v_n)$ alors:  $\sum_{n \in \mathbb{N}} v_n < +\infty \Rightarrow$
	$\sum_{n \in \mathbb{N}} u_n < +\infty$
	\item [-] Si $u_n \sim v_n$ alors $\sum_{n \in \mathbb{N}} u_n$ et $\sum_{n \in \mathbb{N}} v_n$ sont de m\^eme nature
\end{enumerate}
\end{coro}

\begin{theo} Soient $(a_n), (b_n) \in \mathbb{R}_+^{\mathbb{N}}$ avec $a_n \sim b_n$
	\begin{enumerate}
		\item Si $\sum a_n < +\infty$ alors $\sum b_n < +\infty$ et $\sum_{k=n}^{+\infty} a_k \sim \sum_{k=n}^{+\infty} b_k$
		\item Si $\sum a_n = +\infty$ alors $\sum b_n = +\infty$ et $\sum_{k=0}^{n} a_k \sim \sum_{k=0}^{n} b_k$
	\end{enumerate}
\end{theo}

\begin{theo} Crit\`ere de D'Alembert\\
	Soit $(a_n)$ une suite de r\'eels strictement positifs. On suppose $(\frac{a_{n+1}}{a_n})$ a une limite $l \in \mathbb{R} \cup +\infty$,
		alors: \begin{enumerate}
		\item si $l < 1$ on a $\sum a_n < +\infty$
		\item si $l > 1$ alors $a_n \rightarrow +\infty$
		\item si $l = 1$, on ne peut rien affirmer
	\end{enumerate}
\end{theo}


%Convergence absolue {{{1
\section{Convergence absolue}
\begin{defi} Soit $u \in \mathbb{C}^\mathbb{N}$. On dit que la s\'erie $\sum u_n$ est absolument convergente lorsque la SATP $\sum |u_n|$ est
convergente.
\end{defi}

\begin{theo} Toute s\'erie absolument convergente est convergente

\begin{prop} Convergence
	\begin{enumerate}
		\item [-] Crit\`ere $n^\alpha u_n$: Soit $u \in \mathbb{C}^\mathbb{N}$. On suppose qu'il existe $\alpha > 1,\ n^\alpha u_n \rightarrow 0$
			Alors $\sum u_n $ converge absolument
		\item [-] Soit $u \in \mathbb{C}^\mathbb{N}$. On suppose $|\frac{u_{n+1}}{u_n}| \rightarrow l \in [0,1[$ alors $\sum u_n$ converge
			absolument
	\end{enumerate}
\end{prop}

\end{theo}


%Séries alternées {{{1
\section{Séries altern\'ees}
\begin{theo}
	Soit $u$ une suite r\'eelle positive qui d\'ecroit vers $0$, alors $\sum (-1)^n u_n$ converge et de plus, $\forall n \in \mathbb{N},
	|\sum_{k=n+1}^{+\infty} (-1)^k u_k| \leq u_{n+1}$
\end{theo}


%Séries de référence {{{1
\section{S\'eries de r\'ef\'erence}
\begin{enumerate}
	\item S\'erie g\'eom\'etrique: $\sum q^n$ converge $\Leftrightarrow |q| < 1$
	\item S\'eries de Riemann: soit $\alpha \in \mathbb{R}$ alors,\\
		$\sum_{n \in \mathbb{N}^*} \frac{1}{n^\alpha} < +\infty \Leftrightarrow \alpha > 1$
	\item $H_n = \sum \frac{1}{k} \sim \ln n + \gamma + \frac{1}{2n}$
\end{enumerate}

%Remarques {{{1
\section{Remarques}
\begin{enumerate}
	\item Pour des sommes de la forme $\sum_k \frac{1}{\alpha k + \hdots}$ on peut essayer de se ramener \`a l'\'etude de $H_n$
	\item Pour d\'eterminer qu'une suite $u$ converge vers en certain r\'eel, on peut consid\'erer $u_{n+1} - u_n$
	\item Pour faire un developpement asymptotique, on peut utiliser $u_{n+1} - u_n$ puis l'\'equivalence des sommes
\end{enumerate}


\end{document}
