\documentclass[fleqn]{article}
\usepackage{amssymb}
\usepackage{booktabs}

\title{Trigonom\'etrie}
\date{}

\newcommand{\ra}[1]{\renewcommand{\arraystretch}{#1}}
\ra{1.3}

\begin{document}
\maketitle

\section{Fonctions hyperboliques}
\begin{itemize}
	\item D\'{e}finition:
		\[ \cosh t = \frac{e^t + e^{-t}}{2} \]
		\[ \sinh t = \frac{e^t - e^{-t}}{2} \]
		$\cosh$ est paire, $\sinh$ est impaire, $\tanh$ est impaire
	\item $ \cosh^2 t - \sinh^2 t = 1 $
	\item $\cosh$ et $\sinh$ sont d\'{e}rivables sur $\mathbb{R}$: \\
		$\cosh$' $ = \sinh$ \\
		$\sinh$' $ = \cosh$
	\newline
	\item Formules: \\
		$\cosh (a+b) = \cosh a\cosh b + \sinh a\sinh b$ \\
		$\cosh (a-b) = \cosh a\cosh b - \sinh a\sinh b$ \\
		$\sinh (a+b) = \sinh a\cosh b + \sinh b\cosh a$ \\
		$\sinh (a-b) = \sinh a\cosh b - \sinh b\cosh a$ \\
		$\cosh^2 a = \frac{\cosh 2a + 1}{2}$ \\
		$\sinh^2 a = \frac{\cosh 2a - 1}{2}$
\end{itemize}

\section{Fonctions cosinus, sinus, tangente}
\subsection{Cosinus et sinus}
\begin{itemize}
	\item valeurs remarquables:\\
		\begin{tabular}{@{}lrrrrrrrrr@{}}
			\toprule
			t & 0 & $\frac{\Pi}{6}$ & $\frac{\Pi}{4}$ & $\frac{\Pi}{3}$ & $\frac{\Pi}{2}$ & $\frac{2\Pi}{3}$ & $\frac{3\Pi}{4}$
				& $\frac{5\Pi}{6}$ & $\Pi$ \\
			\hline
			$\cos t$ & 1 & $\frac{\sqrt{3}}{2}$ & $\frac{\sqrt{2}}{2}$ & $\frac{1}{2}$ & 0 & $-\frac{1}{2}$ & $-\frac{\sqrt{2}}{2}$
				& $-\frac{\sqrt{3}}{2}$ & -1 \\
			$\sin t$ & 0 & $\frac{1}{2}$ & $\frac{\sqrt{2}}{2}$ & $\frac{\sqrt{3}}{2}$ & 1 & $\frac{\sqrt{3}}{2}$ & $\frac{\sqrt{2}}{2}$
				& $\frac{1}{2}$ & 0 \\
			\bottomrule
		\end{tabular}

	\item Formules: \\
		$\cos(\arcsin x) = \sin(\arccos x) = \sqrt{1 - x^2}$ \\
		$\cos (a+b) = \cos a\cos b - \sin a\sin b$ \\
		$\cos (a-b) = \cos a\cos b + \sin a\sin b$ \\
		$\sin (a+b) = \sin a\cos b + \sin b\cos a$ \\
		$\sin (a-b) = \sin a\cos b - \sin b\cos a$ \\
		$\cos^2 a = \frac{1}{1 + \tan^2a}$ \\
		$\cos^2 a = \frac{1 + \cos 2a}{2}$ \\
		$\sin^2 a = \frac{1 - \cos 2a}{2}$ \\
		\newline
		$\cos p + \cos q = 2\cos\frac{p+q}{2}\cos\frac{p-q}{2}$ \\
		$\cos p - \cos q = -2\sin\frac{p+q}{2}\sin\frac{p-q}{2}$ \\
		$\sin p + \sin q = 2\sin\frac{p+q}{2}\cos\frac{p-q}{2}$ \\
		$\sin p - \sin q = 2\sin\frac{p-q}{2}\cos\frac{p+q}{2}$ \\
\end{itemize}

\subsection{Tangente}
\begin{itemize}
	\item Valeurs remarquables: \\
	\begin{tabular}{lrrrr}
		\toprule
		t        & 0 & $\frac{\Pi}{6}$      & $\frac{\Pi}{4}$ & $\frac{\Pi}{3}$ \\
		\hline
		$\tan t$ & 0 & $\frac{1}{\sqrt{3}}$ & 1               & $\sqrt{3}$ \\
		\bottomrule
	\end{tabular}

	\item Formules: \\
	$\tan (\Pi-t) = -\tan t$ \\
	$\tan (\frac{\Pi}{2} - t) = \frac{1}{t} $ \\
	\newline
	$\tan$'$ t = \frac{1}{\cos^2 t} = 1 + \tan^2 t$ \\
	$\tan (a+b) = \frac{\tan a + \tan b}{1 - \tan a\tan b}$ \\
	$\tan (a-b) = \frac{\tan a - \tan b}{1 + \tan a\tan b}$ \\
\end{itemize}

\section{Fonctions trigonom\'etriques inverses}
Th\'eor\`eme: Soit \(I\) un intervalle de \(\mathbb{R}, f: I \rightarrow \mathbb{R}\) continue et strictement monotone, alors:
\begin{enumerate}
	\item \(J = f(I) = \{f(t)|t \in I\}\) est aussi un intervalle
	\item \(f\) induit une bijection de \(I\) sur \(J\)
	\item L'application r\'eciproque \(g = \tilde{f}^{-1}:J \rightarrow I\) est continue, de strictement m\^eme monotonie que \(f\)
\end{enumerate}
De plus, si \(f\) d\'erivable sur \(I\), \\
soit \(x_0 \in I\ et\ y_0 = f(x_0) \in J\). Si \(f'(x_0) \neq 0 \) alors \(g\) est d\'erivable en \(y_0\) et
\(g'(y_0) = \frac{1}{f'(x_0)} = \frac{1}{f'(g(y_0))}\) \\

\subsection{Arcsin}
\(Arcsin: [-1,1] \rightarrow [-\frac{\Pi}{2}, \frac{\Pi}{2}]\) \\
\(\arcsin\) est impaire, continue et strictement croissante \\
Pour \(y \in ]-1,1[, \arcsin'(y) = \frac{1}{\sqrt{1-y^2}}\) \\

\subsection{Arccos}
\(Arccos: [-1,1] \rightarrow [0, \Pi]\) \\
\(\arcsin\) est continue et strictement d\'{e}croissante \\
Pour \(y \in ]-1,1[, \arccos'(y) = -\frac{1}{\sqrt{1-y^2}}\) \\

\subsection{Arctan}
\(Arctan: [-\frac{\Pi}{2},\frac{\Pi}{2}] \rightarrow \mathbb{R}\) \\
\(\arctan\) est impaire continue et strictement croissante \\
Pour \(y \in \mathbb{R}, \arctan'(y) = \frac{1}{1+y^2}\) \\

\end{document}
