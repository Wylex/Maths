\documentclass[fleqn]{article}
\usepackage{amssymb}
\usepackage{amsthm}
\usepackage{booktabs}

\title{Trigonom\'etrie}
\date{}

\theoremstyle{definition} \newtheorem*{defi}{D\'efinition}
\theoremstyle{definition} \newtheorem*{theo}{Th\'eor\`eme}
\theoremstyle{definition} \newtheorem*{prop}{Propri\'et\'e}
\newcommand{\ra}[1]{\renewcommand{\arraystretch}{#1}}
\ra{1.3}

\begin{document}
\maketitle

\section{Fonctions hyperboliques}
\begin{defi} $\ $
		\[\cosh t = \frac{e^t + e^{-t}}{2},\ \mathbb{R} \rightarrow [1,+\infty]\]
		\[\sinh t = \frac{e^t - e^{-t}}{2},\ \mathbb{R} \rightarrow \mathbb{R} \]
		\[\tanh t = \frac{\sinh}{\cosh},\ \mathbb{R} \rightarrow [-1,+1]\]
\end{defi}

\begin{prop} $ $
	\begin{itemize}
		\item [-] $\cosh$ est paire, $\sinh$ est impaire, $\tanh$ est impaire
		\item [-] $ \cosh^2 t - \sinh^2 t = 1 $
		\item [-] $\cosh$ et $\sinh$ sont d\'{e}rivables sur $\mathbb{R}$: \\
			$\cosh'$ $ = \sinh$ \\
			$\sinh'$ $ = \cosh$
	\end{itemize}
\end{prop}
\begin{prop} Formules
	\begin{itemize}
			\item [-] $\cosh (a+b) = \cosh a\cosh b + \sinh a\sinh b$
			\item [-] $\sinh (a+b) = \sinh a\cosh b + \sinh b\cosh a$
			\item [-] $\cosh (a-b) = \cosh a\cosh b - \sinh a\sinh b$
			\item [-] $\sinh (a-b) = \sinh a\cosh b - \sinh b\cosh a$
			\item [-] $\cosh^2 a = \frac{\cosh 2a + 1}{2}$
			\item [-] $\sinh^2 a = \frac{\cosh 2a - 1}{2}$
	\end{itemize}
\end{prop}

\section{Fonctions cosinus, sinus, tangente}
\subsection{Cosinus et sinus}
\begin{prop} valeurs remarquables\\
	\begin{tabular}{@{}lrrrrrrrrr@{}}
		\toprule
		t & 0 & $\frac{\Pi}{6}$ & $\frac{\Pi}{4}$ & $\frac{\Pi}{3}$ & $\frac{\Pi}{2}$ & $\frac{2\Pi}{3}$ & $\frac{3\Pi}{4}$
			& $\frac{5\Pi}{6}$ & $\Pi$ \\
		\hline
		$\cos t$ & 1 & $\frac{\sqrt{3}}{2}$ & $\frac{\sqrt{2}}{2}$ & $\frac{1}{2}$ & 0 & $-\frac{1}{2}$ & $-\frac{\sqrt{2}}{2}$
			& $-\frac{\sqrt{3}}{2}$ & -1 \\
		$\sin t$ & 0 & $\frac{1}{2}$ & $\frac{\sqrt{2}}{2}$ & $\frac{\sqrt{3}}{2}$ & 1 & $\frac{\sqrt{3}}{2}$ & $\frac{\sqrt{2}}{2}$
			& $\frac{1}{2}$ & 0 \\
		\bottomrule
	\end{tabular}
\end{prop}

\begin{prop} Formules
	\begin{itemize}
		\item [-] $\cos(\arcsin x) = \sin(\arccos x) = \sqrt{1 - x^2}$
		\item [-] $\cos (a+b) = \cos a\cos b - \sin a\sin b$
		\item [-] $\cos (a-b) = \cos a\cos b + \sin a\sin b$
		\item [-] $\sin (a+b) = \sin a\cos b + \sin b\cos a$
		\item [-] $\sin (a-b) = \sin a\cos b - \sin b\cos a$
		\item [-] $\cos^2 a = \frac{1}{1 + \tan^2a} = \frac{1 + \cos 2a}{2}$
		\item [-] $\sin^2 a = \frac{1 - \cos 2a}{2}$
		\newline
		\item [-] $\cos p + \cos q = 2\cos\frac{p+q}{2}\cos\frac{p-q}{2}$
		\item [-] $\cos p - \cos q = -2\sin\frac{p+q}{2}\sin\frac{p-q}{2}$
		\item [-] $\sin p + \sin q = 2\sin\frac{p+q}{2}\cos\frac{p-q}{2}$
		\item [-] $\sin p - \sin q = 2\sin\frac{p-q}{2}\cos\frac{p+q}{2}$
	\end{itemize}
\end{prop}

\begin{prop} $u = \tan \frac{t}{2}$
	\[\sin t = \frac{2u}{1+u^2}\]
	\[\tan t = \frac{2u}{1-u^2}\]
	\[\cos t = \frac{1-u^2}{1+u^2}\]
\end{prop}

\subsection{Tangente}
\begin{prop} Valeurs remarquables\\
	\begin{tabular}{lrrrr}
		\toprule
		t        & 0 & $\frac{\Pi}{6}$      & $\frac{\Pi}{4}$ & $\frac{\Pi}{3}$ \\
		\hline
		$\tan t$ & 0 & $\frac{1}{\sqrt{3}}$ & 1               & $\sqrt{3}$ \\
		\bottomrule
	\end{tabular}
\end{prop}

\begin{prop} Formules
	\begin{itemize}
		\item [-] $\tan (\Pi-t) = -\tan t$
		\item [-] $\tan (\frac{\Pi}{2} - t) = \frac{1}{\tan t}$
		\newline
		\item [-] $\tan' t = \frac{1}{\cos^2 t} = 1 + \tan^2 t$
		\item [-] $\tan (a+b) = \frac{\tan a + \tan b}{1 - \tan a\tan b}$
		\item [-] $\tan (a-b) = \frac{\tan a - \tan b}{1 + \tan a\tan b}$
	\end{itemize}
\end{prop}

\section{Fonctions trigonom\'etriques inverses}
\begin{theo}
	Soit \(I\) un intervalle de \(\mathbb{R}, f: I \rightarrow \mathbb{R}\) continue et strictement monotone, alors:
	\begin{enumerate}
		\item \(J = f(I) = \{f(t)|t \in I\}\) est aussi un intervalle
		\item \(f\) induit une bijection de \(I\) sur \(J\)
		\item L'application r\'eciproque \(g = \tilde{f}^{-1}:J \rightarrow I\) est continue, de strictement m\^eme monotonie que \(f\)
	\end{enumerate}
\end{theo}
De plus, si \(f\) d\'erivable sur \(I\), \\
soit \(x_0 \in I\ et\ y_0 = f(x_0) \in J\). Si \(f'(x_0) \neq 0 \) alors \(g\) est d\'erivable en \(y_0\) et
\(g'(y_0) = \frac{1}{f'(x_0)} = \frac{1}{f'(g(y_0))}\) \\

\subsection{Arcsin}
\(Arcsin: [-1,1] \rightarrow [-\frac{\Pi}{2}, \frac{\Pi}{2}]\) \\
\(\arcsin\) est impaire, continue et strictement croissante \\
Pour \(y \in ]-1,1[, \arcsin'(y) = \frac{1}{\sqrt{1-y^2}}\) \\

\subsection{Arccos}
\(Arccos: [-1,1] \rightarrow [0, \Pi]\) \\
\(\arcsin\) est continue et strictement d\'{e}croissante \\
Pour \(y \in ]-1,1[, \arccos'(y) = -\frac{1}{\sqrt{1-y^2}}\) \\

\subsection{Arctan}
\(Arctan: [-\frac{\Pi}{2},\frac{\Pi}{2}] \rightarrow \mathbb{R}\) \\
\(\arctan\) est impaire continue et strictement croissante \\
Pour \(y \in \mathbb{R}, \arctan'(y) = \frac{1}{1+y^2}\) \\

\section{Astuces}
\begin{enumerate}
	\item Soient $a,b \in \mathbb{R}$ tels que $a^2 + b^2 = 1$\\
		Alors il existe $\theta \in \mathbb{R}$ tel que $a = \cos \theta$ et $b = \sin \theta$
	\item Aplitude/phase \\
		Soient $\lambda, \mu \in \mathbb{R}$, il existe $A \geq 0,\ \varphi \in \mathbb{R}$ tels que:
		\[\forall t \in \mathbb{R},\ \lambda \cos t + \mu \sin t = A\cos(t - \varphi)\]
\end{enumerate}

\end{document}
