\documentclass{article}
\usepackage{amssymb}

\title{Trigonom\'{e}trie}
\date{}


\begin{document}
\maketitle

\section{Fonctions hyperboliques}
\begin{itemize}
	\item D\'{e}finition:
	$$ \cosh t = \frac{e^t + e^{-t}}{2} $$
	$$ \sinh t = \frac{e^t - e^{-t}}{2} $$
	$\cosh$ est paire, $\sinh$ est impaire, $\tanh$ est impaire
	\item $ \cosh^2 t - \sinh^2 t = 1 $
	\item $\cosh$ et $\sinh$ sont d\'{e}rivables sur $\mathbb{R}$: \\
	$\cosh$' $ = \sinh$ \\
	$\sinh$' $ = \cosh$
	\newline
	\item Formules: \\
	$\cosh (a+b) = \cosh a\cosh b + \sinh a\sinh b$ \\
	$\cosh (a-b) = \cosh a\cosh b - \sinh a\sinh b$ \\
	$\sinh (a+b) = \sinh a\cosh b + \sinh b\cosh a$ \\
	$\sinh (a-b) = \sinh a\cosh b - \sinh b\cosh a$ \\
	$\cosh^2 a = \frac{\cosh 2a + 1}{2}$ \\
	$\sinh^2 a = \frac{\cosh 2a - 1}{2}$
\end{itemize}

\section{Fonctions cosinus, sinus, tangente}
\begin{itemize}
	\item valeurs remarquables:\\
	\begin{tabular}{l|c|c|c|c|c|c|c|c|c}
	t & 0 & $\frac{\Pi}{6}$ & $\frac{\Pi}{4}$ & $\frac{\Pi}{3}$ & $\frac{\Pi}{2}$ & $\frac{2\Pi}{3}$ & $\frac{3\Pi}{4}$ & $\frac{5\Pi}{6}$ & $\Pi$ \\
	\hline
	$\cos t$ & 1 & $\frac{\sqrt{3}}{2}$ & $\frac{\sqrt{2}}{2}$ & $\frac{1}{2}$ & 0 & $-\frac{1}{2}$ & $-\frac{\sqrt{2}}{2}$ & $-\frac{\sqrt{3}}{2}$ & -1 \\
	$\sin t$ & 0 & $\frac{1}{2}$ & $\frac{\sqrt{2}}{2}$ & $\frac{\sqrt{3}}{2}$ & 1 & $\frac{\sqrt{3}}{2}$ & $\frac{\sqrt{2}}{2}$ & $\frac{1}{2}$ & 0 \\
	\end{tabular}

	\item Formules: \\
	$\cos (a+b) = \cos a\cos b - \sin a\sin b$ \\
	$\cos (a-b) = \cos a\cos b + \sin a\sin b$ \\
	$\sin (a+b) = \sin a\cos b + \sin b\cos a$ \\
	$\sin (a-b) = \sin a\cos b - \sin b\cos a$ \\
	$\cos^2 a = \frac{1}{1 + \tan^2a}$ \\
	$\cos^2 a = \frac{1 + \cos 2a}{2}$ \\
	$\sin^2 a = \frac{1 - \cos 2a}{2}$ \\
	\newline
	$\cos p + \cos q = 2\cos\frac{p+q}{2}\cos\frac{p-q}{2}$ \\
	$\cos p - \cos q = -2\sin\frac{p+q}{2}\sin\frac{p-q}{2}$ \\
	$\sin p + \sin q = 2\sin\frac{p+q}{2}\cos\frac{p-q}{2}$ \\
	$\sin p - \sin q = 2\sin\frac{p-q}{2}\cos\frac{p+q}{2}$ \\

	\item Tangente: \\
	\begin{itemize}
		\item Valeurs remarquables: \\
		\begin{tabular}{l|c|c|c|c}
		t        & 0 & $\frac{\Pi}{6}$      & $\frac{\Pi}{4}$ & $\frac{\Pi}{3}$ \\
		\hline
		$\tan t$ & 0 & $\frac{1}{\sqrt{3}}$ & 1               & $\sqrt{3}$ \\
		\end{tabular}

		\item Formules: \\
		$\tan$'$ t = \frac{1}{\cos^2 t} = 1 + \tan^2 t$ \\
		$\tan (a+b) = \frac{\tan a + \tan b}{1 - \tan a\tan b}$ \\
		$\tan (a-b) = \frac{\tan a - \tan b}{1 + \tan a\tan b}$ \\
	\end{itemize}

\end{itemize}


\end{document}
